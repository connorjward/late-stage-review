\documentclass[a4,10pt]{article}

% packages
\usepackage{biblatex}
  \addbibresource{bibliography.bib}
\usepackage[acronym,nomain]{glossaries}
  \setacronymstyle{long-short}
  \newacronym[shortplural={DoFs},longplural={degrees-of-freedom}]
    {dof}{DoF}{degree-of-freedom}
  \newacronym{fem}{FEM}{finite element method}
  \newacronym{pde}{PDE}{partial differential equation}
  \newacronym[shortplural={FLOPs},longplural={floating-point operations}]
    {flop}{FLOP}{floating-point operation}
  \newacronym{ai}{AI}{arithmetic intensity}
  \newacronym{dag}{DAG}{directed acyclic graph}
  \newacronym{poset}{poset}{partially-ordered set}
  \newacronym{rcm}{RCM}{reverse Cuthill-McKee}
\usepackage{graphicx}
  \graphicspath{{figures}}
\usepackage{minted}
\usepackage{subcaption}
\usepackage{amsmath}
% source: https://tex.stackexchange.com/questions/650034/mathbb-font-for-lowercase-letters
\usepackage[bb=libus]{mathalpha}
\usepackage{pgf}
\usepackage{pgfplots}
\usepackage{tikz}
\usepackage{tkz-euclide}
  \usetikzlibrary{arrows,calc,graphs,graphdrawing,positioning,tikzmark,shapes.geometric,patterns.meta,decorations.pathreplacing}
  \usegdlibrary{trees}
  \pgfdeclarelayer{background}
  \pgfsetlayers{background,main}
  \tikzstyle{ptlabel} = [anchor=center, color=black, opacity=1]
  \tikzset{font={\small}}
  \tikzset{label style/.append style={font=\small}}
  % source: https://tex.stackexchange.com/questions/356564/macro-for-rounded-polygon-around-some-nodes
  \def\drawpolygon#1,#2;{
    \begin{pgfonlayer}{background}
        \filldraw[line width=28,join=round](#1.center)foreach\A in{#2}{--(\A.center)}--cycle;
        \filldraw[line width=27,join=round,white](#1.center)foreach\A in{#2}{--(\A.center)}--cycle;
    \end{pgfonlayer}
  }

% macros
\newcommand{\projectname}{\texttt{pyop3}}
% hacky way to get \pyop2 and \pyop3 as valid macros
% source: https://tex.stackexchange.com/questions/13290/how-to-define-macros-with-numbers-in-them
\def\pyop#1{\ifnum#1=2 {PyOP2}\else \ifnum#1=3 {\texttt{pyop3}}\fi \fi}
\newcommand{\basichasse}{%
  \begin{scope}[auto,every node/.style={circle,minimum size=20pt,draw,color=black}]
    \begin{scope}[yshift=0cm]
      \node (1) [xshift={1*\textwidth/3}] {1};
      \node (2) [xshift={2*\textwidth/3}] {2};
    \end{scope}

    \begin{scope}[yshift=2cm]
      \node (7) [xshift={1*\textwidth/6}] {7};
      \node (8) [xshift={2*\textwidth/6}] {8};
      \node (9) [xshift={3*\textwidth/6}] {9};
      \node (10) [xshift={4*\textwidth/6}] {10};
      \node (11) [xshift={5*\textwidth/6}] {11};
    \end{scope}

    \begin{scope}[yshift=4cm]
      \node (3) [xshift={1*\textwidth/5}] {3};
      \node (4) [xshift={2*\textwidth/5}] {4};
      \node (5) [xshift={3*\textwidth/5}] {5};
      \node (6) [xshift={4*\textwidth/5}] {6};
    \end{scope}

    \draw [-Stealth] (1) -- (7);
    \draw [-Stealth] (1) -- (8);
    \draw [-Stealth] (1) -- (9);
    \draw [-Stealth] (2) -- (9);
    \draw [-Stealth] (2) -- (10);
    \draw [-Stealth] (2) -- (11);
    \draw [-Stealth] (7) -- (3);
    \draw [-Stealth] (7) -- (5);
    \draw [-Stealth] (8) -- (3);
    \draw [-Stealth] (8) -- (4);
    \draw [-Stealth] (9) -- (4);
    \draw [-Stealth] (9) -- (5);
    \draw [-Stealth] (10) -- (4);
    \draw [-Stealth] (10) -- (6);
    \draw [-Stealth] (11) -- (5);
    \draw [-Stealth] (11) -- (6);
  \end{scope}
}

\newcommand{\py}{\mintinline{python}}
\newcommand{\clang}{\mintinline{c}}
\newcommand{\closure}{\mathbb{cl}}
\newcommand{\support}{\mathbb{supp}}
\newcommand{\plexstar}{\mathbb{st}}
\newcommand{\cone}{\mathbb{cone}}

% title
\title{High Performance Mesh Abstractions}
\author{Connor Ward}
\date{\today}

\begin{document}
  \maketitle

  \section{Introduction}

In scientific computing, the composition of appropriate software abstractions is essential for scientists to write portable and performant simulations in a productive way (the three P's).
Having suitable abstraction layers allows for a separation of concerns whereby numericists can reason about their problem from a purely mathematical point-of-view and computer scientists can focus on optimising performance.
Each discipline is presented with a particular interface from which the problems of interest can be expressed in the clearest possible way, facilitating rapid code development.

When it comes to writing software there are effectively three choices of approach.
For many problems, generic library interfaces introduce too much overhead to be viable options for writing programs.
Similarly, hand-written codes, though extremely fast, require a substantial effort to maintain and extend and the codebase can be very large.
Code generation is an appealing solution to these problems.
Given an appropriate abstraction, high-performance code can be automatically generated, compiled and run.
This offers an advantage over library interfaces because problem-specific information can be exploited to generate faster code (e.g. commonly used operations can be memoized for fast lookups), and the task of actually writing the code is offloaded to a compiler rather than being hand-written.
With a code generation framework, the key questions now become:
``What is an appropriate abstraction for capturing all of the behaviour I wish to model?", and
``What performance optimisations are nicely expressed at this layer of abstraction?"

% TODO: put in a list of some kind
In this work, we present \pyop3, a library for the fast execution of mesh-based computations over some local stencil.
In accordance with the principles described above, \pyop3 deals mainly in 3 abstractions:
Firstly, the user interface is motivated by the fact that many operations relating to the solution of \glspl{pde} can be expressed as the operation of some `local' kernel over a set of entities in the mesh where only functions with non-zero support on this entity are considered in the calculation.
A classic example of this sort of calculation occurs with finite element assembly where the cells of the mesh are the iteration set and the kernel uses \glspl{dof} from the cell and enclosing edges and vertices.
This first abstraction layer therefore presents an interface to the user where they may straightforwardly express the operation of local kernels within loops over mesh entities, specifying the requisite restrictions for the data.
The second abstraction, intended as an internal representation for the developer, describes the data layout and the third is a polyhedral loop model that is the target for the code generation.

The rest of this paper is laid out as follows:
In Section~\ref{sec:background} we review existing stencil libraries and mesh abstractions as well as strategies and details for code performance optimisation.
Section~\ref{sec:impl} discusses the design of \pyop3, Section~\ref{sec:future} then reviews some possible extensions that might be pursued in future.
Some concluding remarks are made in Section~\ref{sec:conclusions}.

  \section{Background}

In this chapter we review: existing software abstractions for mesh computations, common strategies for optimising performance, and mesh-specific optimisations.

\subsection{Existing software abstractions}

\subsection{Stencil languages}

Given the ubiquity of stencil operations in simulations, a number of libraries exist providing convenient interfaces for stencil applications.

% they are largely all inspired by the following pieces of info:
% - a mesh can be considered as sets of entities
% - one loops over a set, collects adjacent data, then calls some local kernel
% - the local kernel may be written in the same language as the stencil language, but it needn't be

Ebb, Simit and Liszt follow the approach of providing a domain-specific language for the expression of stencil problems...
They all provide high performance execution on GPUs as well as CPUs.

% i don't think simit works on MPI

OP2 is another approach~\cite{mudaligeOP2ActiveLibrary2012,mudaligeDesignInitialPerformance2013}.
Rather than using a domain-specific language, OP2 provides a simple API to the user for specifying the problem.
The key entities in the OP2 data model are: sets, data on sets, mappings between sets, and operations applied over these sets.
Having provided these inputs, the OP2 compiler is then called and transforms to source code to a high performance implementation of the traversal for a specific architecture.

% then PyOP2
% describe libCEED - main benefit is that it is fast.

% it is noteworthy that Ebb, Simit, Liszt have custom implementations of unstructured meshes.
% whereas (Py)OP2 and libCEED have an abstraction that does not specifically include a mesh.
% the latter is favourable as one can reuse existing solutions to other mesh problems like
% refinement, adaptivity, I/O, decomposition...
% however, lack of topological information can make code unreadable, hard to write as
% topological info is lost, e.g. composability is not really possible.

% this leads very nicely into discussion of DMPlex
% note: we don't have a nice solution to orientations yet - but no one does

Another library for the application of stencil operations, specifically high-order matrix-free kernels for the \gls{fem}, is libCEED.
% libceed - actually offloaded to PETSc for element restriction (the bit PyOP2 does)
% it's nice that libCEED and PyOP2 use DMPlex as the burden of other mesh operations (e.g. distribution, I/O, refinement) are very non-trivial

PyOP2 is a domain-specific language for expressing computations over unstructured meshes.
It is the direct precursor, and inspiration for, \projectname.

It distinguishes itself from OP2, a library depending on the same abstractions, by using run-time code generation instead of static analysis and transformation of the source code.

PyOP2's mesh abstraction is formed out of the following main components:

% \begin{itemize}
%   \item \mintinline{python}{set}
% \end{itemize}

In PyOP2, data is defined on \textit{sets} and these are related to one another using \textit{mappings}.
Importantly, this abstraction does not contain any concept of the underlying mesh and instead all of the required information is encoded in the maps.

  Computations over the mesh are expressed as the execution of some local kernel over all entities of some iteration set via a construct called a parallel loop, or \textit{parloop}.
  The kernel is written using Loopy, a library for expressing array-based computations in a platform-generic language~\cite{klocknerLooPyTransformationbased2014}.
  This intermediate representation allows for interplay between the local kernels enabling optimisations such as inter-element vectorisation~\cite{sunStudyVectorizationMatrixfree2020}.

  At present, PyOP2 only works on distributed memory, CPU-only systems (although some work has been done to permit execution on GPUs~\cite{fenics2021-kulkarni}).
  During the execution of a parloop, each rank works independently on some partition of the mesh.
  To avoid excessive communication between ranks, each rank has a narrow \textit{halo} region that overlaps with neighbouring ranks that is executed redundantly.
  The halos are split into \textit{owned}, \textit{exec}, and \textit{non-exec} regions to indicate the data's origin and the communication direction between the neighbouring processes.


% Liszt, Simit - notably associates data information with topological entities

\subsubsection{Mesh representations}

In software, a mesh is typically represented by a collection of sets of entities (e.g. cells or faces), coupled with adjacency relations between these sets.
Possible abstractions capturing this behaviour include databases (ebb, moab) or hypergraphs (simit).
In this work we focus on DMPlex, the unstructured mesh abstraction used in PETSc.
In contrast with Ebb, Simit or Liszt, DMPlex is a more general purpose mesh abstraction and so has a more substantial feature set.z

In DMPlex, the mesh is represented as a \textit{CW-complex}, an object from algebraic topology that describes some topological space.
In such a complex, all topological entities (e.g. cells, vertices) are simply referred to as \textit{points} and the connectivity of the mesh can be expressed as the edges of a \gls{dag} with the vertices being the points of the mesh.
More specifically, the points and relations form a \gls{poset} such that the mesh can be visualised using a Hasse diagram (Figure~\ref{fig:hasse_diagram}).

It is important to note that DMPlex works for arbitrary dimension.

\begin{figure}
  \begin{subfigure}{0.45\textwidth}
    \begin{tikzpicture}
      \draw (0,2) -- (2,0);
      \draw (0,2) -- (2,4);
      \draw (2,0) -- (2,4);
      \draw (2,0) -- (4,2);
      \draw (2,4) -- (4,2);

      \node[at={(1,2)}, xshift=.2cm] {1};
      \node[at={(3,2)}, xshift=-.2cm] {2};
      \node[at={(0,2)}, xshift=-.2cm] {3};
      \node[at={(2,4)}, yshift=.2cm] {4};
      \node[at={(2,0)}, yshift=-.2cm] {5};
      \node[at={(4,2)}, xshift=.2cm] {6};
      \node[at={(1,1)}, xshift=-.2cm, yshift=-.2cm] {7};
      \node[at={(1,3)}, xshift=-.2cm, yshift=.2cm] {8};
      \node[at={(2,2)}, xshift=-.2cm] {9};
      \node[at={(3,3)}, xshift=.2cm, yshift=.2cm] {10};
      \node[at={(3,1)}, xshift=.2cm, yshift=-.2cm] {11};
    \end{tikzpicture}
  \end{subfigure}
  %
  \begin{subfigure}{0.45\textwidth}
    \centering
    \begin{tikzpicture}
      \basichasse
    \end{tikzpicture}
  \end{subfigure}

  \caption{...}
  \label{fig:hasse_diagram}
\end{figure}

% stencil operations next - closure, star, etc

\begin{figure}
  \centering
  \begin{subfigure}{0.45\textwidth}
    \centering
    \begin{tikzpicture}
      \basichasse
      \drawpolygon 7,9;
    \end{tikzpicture}
    \caption{$cone(1)$}
  \end{subfigure}
  \begin{subfigure}{0.45\textwidth}
    \centering
    \begin{tikzpicture}
      \basichasse
      \drawpolygon 1,2;
    \end{tikzpicture}
    \caption{$supp(9)$}
  \end{subfigure}
  \begin{subfigure}{0.45\textwidth}
    \centering
    \begin{tikzpicture}
      \basichasse
      \drawpolygon 1,7,3,5,9;
    \end{tikzpicture}
    \caption{$cl(1)$}
  \end{subfigure}
  \begin{subfigure}{0.45\textwidth}
    \centering
    \begin{tikzpicture}
      \basichasse
      \drawpolygon 4,10,2,1,8;
    \end{tikzpicture}
    \caption{$star(1)$}
  \end{subfigure}

  \caption{...}
  \label{fig:plex_restrictions}
\end{figure}


Stencil queries are natural to express at this level.
For instance, the classical finite element request of ``give me all of the \glspl{dof} that have local support" is simply expressed as the closure of a given cell.
Another example useful for finite volume calculations: ``what are my neighbouring cells?" is supp(cone(c)).
One can also do clever patch things.


\subsection{Performance optimisation}

\subsubsection{Roofline model}

\begin{figure}
  %% Creator: Matplotlib, PGF backend
%%
%% To include the figure in your LaTeX document, write
%%   \input{<filename>.pgf}
%%
%% Make sure the required packages are loaded in your preamble
%%   \usepackage{pgf}
%%
%% Also ensure that all the required font packages are loaded; for instance,
%% the lmodern package is sometimes necessary when using math font.
%%   \usepackage{lmodern}
%%
%% Figures using additional raster images can only be included by \input if
%% they are in the same directory as the main LaTeX file. For loading figures
%% from other directories you can use the `import` package
%%   \usepackage{import}
%%
%% and then include the figures with
%%   \import{<path to file>}{<filename>.pgf}
%%
%% Matplotlib used the following preamble
%%   
%%   \usepackage{fontspec}
%%   \setmainfont{DejaVuSerif.ttf}[Path=\detokenize{/home/connor/Code/late-stage-review/venv/lib/python3.10/site-packages/matplotlib/mpl-data/fonts/ttf/}]
%%   \setsansfont{DejaVuSans.ttf}[Path=\detokenize{/home/connor/Code/late-stage-review/venv/lib/python3.10/site-packages/matplotlib/mpl-data/fonts/ttf/}]
%%   \setmonofont{DejaVuSansMono.ttf}[Path=\detokenize{/home/connor/Code/late-stage-review/venv/lib/python3.10/site-packages/matplotlib/mpl-data/fonts/ttf/}]
%%   \makeatletter\@ifpackageloaded{underscore}{}{\usepackage[strings]{underscore}}\makeatother
%%
\begingroup%
\makeatletter%
\begin{pgfpicture}%
\pgfpathrectangle{\pgfpointorigin}{\pgfqpoint{4.541667in}{3.638889in}}%
\pgfusepath{use as bounding box, clip}%
\begin{pgfscope}%
\pgfsetbuttcap%
\pgfsetmiterjoin%
\definecolor{currentfill}{rgb}{1.000000,1.000000,1.000000}%
\pgfsetfillcolor{currentfill}%
\pgfsetlinewidth{0.000000pt}%
\definecolor{currentstroke}{rgb}{1.000000,1.000000,1.000000}%
\pgfsetstrokecolor{currentstroke}%
\pgfsetdash{}{0pt}%
\pgfpathmoveto{\pgfqpoint{0.000000in}{0.000000in}}%
\pgfpathlineto{\pgfqpoint{4.541667in}{0.000000in}}%
\pgfpathlineto{\pgfqpoint{4.541667in}{3.638889in}}%
\pgfpathlineto{\pgfqpoint{0.000000in}{3.638889in}}%
\pgfpathlineto{\pgfqpoint{0.000000in}{0.000000in}}%
\pgfpathclose%
\pgfusepath{fill}%
\end{pgfscope}%
\begin{pgfscope}%
\pgfsetbuttcap%
\pgfsetmiterjoin%
\definecolor{currentfill}{rgb}{1.000000,1.000000,1.000000}%
\pgfsetfillcolor{currentfill}%
\pgfsetlinewidth{0.000000pt}%
\definecolor{currentstroke}{rgb}{0.000000,0.000000,0.000000}%
\pgfsetstrokecolor{currentstroke}%
\pgfsetstrokeopacity{0.000000}%
\pgfsetdash{}{0pt}%
\pgfpathmoveto{\pgfqpoint{0.567708in}{0.400278in}}%
\pgfpathlineto{\pgfqpoint{4.087500in}{0.400278in}}%
\pgfpathlineto{\pgfqpoint{4.087500in}{3.202222in}}%
\pgfpathlineto{\pgfqpoint{0.567708in}{3.202222in}}%
\pgfpathlineto{\pgfqpoint{0.567708in}{0.400278in}}%
\pgfpathclose%
\pgfusepath{fill}%
\end{pgfscope}%
\begin{pgfscope}%
\pgfpathrectangle{\pgfqpoint{0.567708in}{0.400278in}}{\pgfqpoint{3.519792in}{2.801944in}}%
\pgfusepath{clip}%
\pgfsetbuttcap%
\pgfsetroundjoin%
\definecolor{currentfill}{rgb}{1.000000,0.000000,0.000000}%
\pgfsetfillcolor{currentfill}%
\pgfsetlinewidth{1.003750pt}%
\definecolor{currentstroke}{rgb}{1.000000,0.000000,0.000000}%
\pgfsetstrokecolor{currentstroke}%
\pgfsetdash{}{0pt}%
\pgfsys@defobject{currentmarker}{\pgfqpoint{-0.041667in}{-0.041667in}}{\pgfqpoint{0.041667in}{0.041667in}}{%
\pgfpathmoveto{\pgfqpoint{0.000000in}{-0.041667in}}%
\pgfpathcurveto{\pgfqpoint{0.011050in}{-0.041667in}}{\pgfqpoint{0.021649in}{-0.037276in}}{\pgfqpoint{0.029463in}{-0.029463in}}%
\pgfpathcurveto{\pgfqpoint{0.037276in}{-0.021649in}}{\pgfqpoint{0.041667in}{-0.011050in}}{\pgfqpoint{0.041667in}{0.000000in}}%
\pgfpathcurveto{\pgfqpoint{0.041667in}{0.011050in}}{\pgfqpoint{0.037276in}{0.021649in}}{\pgfqpoint{0.029463in}{0.029463in}}%
\pgfpathcurveto{\pgfqpoint{0.021649in}{0.037276in}}{\pgfqpoint{0.011050in}{0.041667in}}{\pgfqpoint{0.000000in}{0.041667in}}%
\pgfpathcurveto{\pgfqpoint{-0.011050in}{0.041667in}}{\pgfqpoint{-0.021649in}{0.037276in}}{\pgfqpoint{-0.029463in}{0.029463in}}%
\pgfpathcurveto{\pgfqpoint{-0.037276in}{0.021649in}}{\pgfqpoint{-0.041667in}{0.011050in}}{\pgfqpoint{-0.041667in}{0.000000in}}%
\pgfpathcurveto{\pgfqpoint{-0.041667in}{-0.011050in}}{\pgfqpoint{-0.037276in}{-0.021649in}}{\pgfqpoint{-0.029463in}{-0.029463in}}%
\pgfpathcurveto{\pgfqpoint{-0.021649in}{-0.037276in}}{\pgfqpoint{-0.011050in}{-0.041667in}}{\pgfqpoint{0.000000in}{-0.041667in}}%
\pgfpathlineto{\pgfqpoint{0.000000in}{-0.041667in}}%
\pgfpathclose%
\pgfusepath{stroke,fill}%
}%
\begin{pgfscope}%
\pgfsys@transformshift{1.082402in}{0.902053in}%
\pgfsys@useobject{currentmarker}{}%
\end{pgfscope}%
\end{pgfscope}%
\begin{pgfscope}%
\pgfpathrectangle{\pgfqpoint{0.567708in}{0.400278in}}{\pgfqpoint{3.519792in}{2.801944in}}%
\pgfusepath{clip}%
\pgfsetbuttcap%
\pgfsetroundjoin%
\definecolor{currentfill}{rgb}{1.000000,0.000000,0.000000}%
\pgfsetfillcolor{currentfill}%
\pgfsetlinewidth{1.003750pt}%
\definecolor{currentstroke}{rgb}{1.000000,0.000000,0.000000}%
\pgfsetstrokecolor{currentstroke}%
\pgfsetdash{}{0pt}%
\pgfsys@defobject{currentmarker}{\pgfqpoint{-0.041667in}{-0.041667in}}{\pgfqpoint{0.041667in}{0.041667in}}{%
\pgfpathmoveto{\pgfqpoint{0.000000in}{-0.041667in}}%
\pgfpathcurveto{\pgfqpoint{0.011050in}{-0.041667in}}{\pgfqpoint{0.021649in}{-0.037276in}}{\pgfqpoint{0.029463in}{-0.029463in}}%
\pgfpathcurveto{\pgfqpoint{0.037276in}{-0.021649in}}{\pgfqpoint{0.041667in}{-0.011050in}}{\pgfqpoint{0.041667in}{0.000000in}}%
\pgfpathcurveto{\pgfqpoint{0.041667in}{0.011050in}}{\pgfqpoint{0.037276in}{0.021649in}}{\pgfqpoint{0.029463in}{0.029463in}}%
\pgfpathcurveto{\pgfqpoint{0.021649in}{0.037276in}}{\pgfqpoint{0.011050in}{0.041667in}}{\pgfqpoint{0.000000in}{0.041667in}}%
\pgfpathcurveto{\pgfqpoint{-0.011050in}{0.041667in}}{\pgfqpoint{-0.021649in}{0.037276in}}{\pgfqpoint{-0.029463in}{0.029463in}}%
\pgfpathcurveto{\pgfqpoint{-0.037276in}{0.021649in}}{\pgfqpoint{-0.041667in}{0.011050in}}{\pgfqpoint{-0.041667in}{0.000000in}}%
\pgfpathcurveto{\pgfqpoint{-0.041667in}{-0.011050in}}{\pgfqpoint{-0.037276in}{-0.021649in}}{\pgfqpoint{-0.029463in}{-0.029463in}}%
\pgfpathcurveto{\pgfqpoint{-0.021649in}{-0.037276in}}{\pgfqpoint{-0.011050in}{-0.041667in}}{\pgfqpoint{0.000000in}{-0.041667in}}%
\pgfpathlineto{\pgfqpoint{0.000000in}{-0.041667in}}%
\pgfpathclose%
\pgfusepath{stroke,fill}%
}%
\begin{pgfscope}%
\pgfsys@transformshift{3.085801in}{1.187726in}%
\pgfsys@useobject{currentmarker}{}%
\end{pgfscope}%
\end{pgfscope}%
\begin{pgfscope}%
\pgfpathrectangle{\pgfqpoint{0.567708in}{0.400278in}}{\pgfqpoint{3.519792in}{2.801944in}}%
\pgfusepath{clip}%
\pgfsetbuttcap%
\pgfsetroundjoin%
\definecolor{currentfill}{rgb}{1.000000,0.000000,0.000000}%
\pgfsetfillcolor{currentfill}%
\pgfsetlinewidth{1.003750pt}%
\definecolor{currentstroke}{rgb}{1.000000,0.000000,0.000000}%
\pgfsetstrokecolor{currentstroke}%
\pgfsetdash{}{0pt}%
\pgfsys@defobject{currentmarker}{\pgfqpoint{-0.041667in}{-0.041667in}}{\pgfqpoint{0.041667in}{0.041667in}}{%
\pgfpathmoveto{\pgfqpoint{0.000000in}{-0.041667in}}%
\pgfpathcurveto{\pgfqpoint{0.011050in}{-0.041667in}}{\pgfqpoint{0.021649in}{-0.037276in}}{\pgfqpoint{0.029463in}{-0.029463in}}%
\pgfpathcurveto{\pgfqpoint{0.037276in}{-0.021649in}}{\pgfqpoint{0.041667in}{-0.011050in}}{\pgfqpoint{0.041667in}{0.000000in}}%
\pgfpathcurveto{\pgfqpoint{0.041667in}{0.011050in}}{\pgfqpoint{0.037276in}{0.021649in}}{\pgfqpoint{0.029463in}{0.029463in}}%
\pgfpathcurveto{\pgfqpoint{0.021649in}{0.037276in}}{\pgfqpoint{0.011050in}{0.041667in}}{\pgfqpoint{0.000000in}{0.041667in}}%
\pgfpathcurveto{\pgfqpoint{-0.011050in}{0.041667in}}{\pgfqpoint{-0.021649in}{0.037276in}}{\pgfqpoint{-0.029463in}{0.029463in}}%
\pgfpathcurveto{\pgfqpoint{-0.037276in}{0.021649in}}{\pgfqpoint{-0.041667in}{0.011050in}}{\pgfqpoint{-0.041667in}{0.000000in}}%
\pgfpathcurveto{\pgfqpoint{-0.041667in}{-0.011050in}}{\pgfqpoint{-0.037276in}{-0.021649in}}{\pgfqpoint{-0.029463in}{-0.029463in}}%
\pgfpathcurveto{\pgfqpoint{-0.021649in}{-0.037276in}}{\pgfqpoint{-0.011050in}{-0.041667in}}{\pgfqpoint{0.000000in}{-0.041667in}}%
\pgfpathlineto{\pgfqpoint{0.000000in}{-0.041667in}}%
\pgfpathclose%
\pgfusepath{stroke,fill}%
}%
\begin{pgfscope}%
\pgfsys@transformshift{3.667269in}{2.521065in}%
\pgfsys@useobject{currentmarker}{}%
\end{pgfscope}%
\end{pgfscope}%
\begin{pgfscope}%
\pgfsetbuttcap%
\pgfsetroundjoin%
\definecolor{currentfill}{rgb}{0.000000,0.000000,0.000000}%
\pgfsetfillcolor{currentfill}%
\pgfsetlinewidth{0.803000pt}%
\definecolor{currentstroke}{rgb}{0.000000,0.000000,0.000000}%
\pgfsetstrokecolor{currentstroke}%
\pgfsetdash{}{0pt}%
\pgfsys@defobject{currentmarker}{\pgfqpoint{0.000000in}{-0.048611in}}{\pgfqpoint{0.000000in}{0.000000in}}{%
\pgfpathmoveto{\pgfqpoint{0.000000in}{0.000000in}}%
\pgfpathlineto{\pgfqpoint{0.000000in}{-0.048611in}}%
\pgfusepath{stroke,fill}%
}%
\begin{pgfscope}%
\pgfsys@transformshift{0.590121in}{0.400278in}%
\pgfsys@useobject{currentmarker}{}%
\end{pgfscope}%
\end{pgfscope}%
\begin{pgfscope}%
\pgfsetbuttcap%
\pgfsetroundjoin%
\definecolor{currentfill}{rgb}{0.000000,0.000000,0.000000}%
\pgfsetfillcolor{currentfill}%
\pgfsetlinewidth{0.803000pt}%
\definecolor{currentstroke}{rgb}{0.000000,0.000000,0.000000}%
\pgfsetstrokecolor{currentstroke}%
\pgfsetdash{}{0pt}%
\pgfsys@defobject{currentmarker}{\pgfqpoint{0.000000in}{-0.048611in}}{\pgfqpoint{0.000000in}{0.000000in}}{%
\pgfpathmoveto{\pgfqpoint{0.000000in}{0.000000in}}%
\pgfpathlineto{\pgfqpoint{0.000000in}{-0.048611in}}%
\pgfusepath{stroke,fill}%
}%
\begin{pgfscope}%
\pgfsys@transformshift{1.422014in}{0.400278in}%
\pgfsys@useobject{currentmarker}{}%
\end{pgfscope}%
\end{pgfscope}%
\begin{pgfscope}%
\pgfsetbuttcap%
\pgfsetroundjoin%
\definecolor{currentfill}{rgb}{0.000000,0.000000,0.000000}%
\pgfsetfillcolor{currentfill}%
\pgfsetlinewidth{0.803000pt}%
\definecolor{currentstroke}{rgb}{0.000000,0.000000,0.000000}%
\pgfsetstrokecolor{currentstroke}%
\pgfsetdash{}{0pt}%
\pgfsys@defobject{currentmarker}{\pgfqpoint{0.000000in}{-0.048611in}}{\pgfqpoint{0.000000in}{0.000000in}}{%
\pgfpathmoveto{\pgfqpoint{0.000000in}{0.000000in}}%
\pgfpathlineto{\pgfqpoint{0.000000in}{-0.048611in}}%
\pgfusepath{stroke,fill}%
}%
\begin{pgfscope}%
\pgfsys@transformshift{2.253907in}{0.400278in}%
\pgfsys@useobject{currentmarker}{}%
\end{pgfscope}%
\end{pgfscope}%
\begin{pgfscope}%
\pgfsetbuttcap%
\pgfsetroundjoin%
\definecolor{currentfill}{rgb}{0.000000,0.000000,0.000000}%
\pgfsetfillcolor{currentfill}%
\pgfsetlinewidth{0.803000pt}%
\definecolor{currentstroke}{rgb}{0.000000,0.000000,0.000000}%
\pgfsetstrokecolor{currentstroke}%
\pgfsetdash{}{0pt}%
\pgfsys@defobject{currentmarker}{\pgfqpoint{0.000000in}{-0.048611in}}{\pgfqpoint{0.000000in}{0.000000in}}{%
\pgfpathmoveto{\pgfqpoint{0.000000in}{0.000000in}}%
\pgfpathlineto{\pgfqpoint{0.000000in}{-0.048611in}}%
\pgfusepath{stroke,fill}%
}%
\begin{pgfscope}%
\pgfsys@transformshift{3.085801in}{0.400278in}%
\pgfsys@useobject{currentmarker}{}%
\end{pgfscope}%
\end{pgfscope}%
\begin{pgfscope}%
\pgfsetbuttcap%
\pgfsetroundjoin%
\definecolor{currentfill}{rgb}{0.000000,0.000000,0.000000}%
\pgfsetfillcolor{currentfill}%
\pgfsetlinewidth{0.803000pt}%
\definecolor{currentstroke}{rgb}{0.000000,0.000000,0.000000}%
\pgfsetstrokecolor{currentstroke}%
\pgfsetdash{}{0pt}%
\pgfsys@defobject{currentmarker}{\pgfqpoint{0.000000in}{-0.048611in}}{\pgfqpoint{0.000000in}{0.000000in}}{%
\pgfpathmoveto{\pgfqpoint{0.000000in}{0.000000in}}%
\pgfpathlineto{\pgfqpoint{0.000000in}{-0.048611in}}%
\pgfusepath{stroke,fill}%
}%
\begin{pgfscope}%
\pgfsys@transformshift{3.917694in}{0.400278in}%
\pgfsys@useobject{currentmarker}{}%
\end{pgfscope}%
\end{pgfscope}%
\begin{pgfscope}%
\pgfsetbuttcap%
\pgfsetroundjoin%
\definecolor{currentfill}{rgb}{0.000000,0.000000,0.000000}%
\pgfsetfillcolor{currentfill}%
\pgfsetlinewidth{0.602250pt}%
\definecolor{currentstroke}{rgb}{0.000000,0.000000,0.000000}%
\pgfsetstrokecolor{currentstroke}%
\pgfsetdash{}{0pt}%
\pgfsys@defobject{currentmarker}{\pgfqpoint{0.000000in}{-0.027778in}}{\pgfqpoint{0.000000in}{0.000000in}}{%
\pgfpathmoveto{\pgfqpoint{0.000000in}{0.000000in}}%
\pgfpathlineto{\pgfqpoint{0.000000in}{-0.027778in}}%
\pgfusepath{stroke,fill}%
}%
\begin{pgfscope}%
\pgfsys@transformshift{0.840546in}{0.400278in}%
\pgfsys@useobject{currentmarker}{}%
\end{pgfscope}%
\end{pgfscope}%
\begin{pgfscope}%
\pgfsetbuttcap%
\pgfsetroundjoin%
\definecolor{currentfill}{rgb}{0.000000,0.000000,0.000000}%
\pgfsetfillcolor{currentfill}%
\pgfsetlinewidth{0.602250pt}%
\definecolor{currentstroke}{rgb}{0.000000,0.000000,0.000000}%
\pgfsetstrokecolor{currentstroke}%
\pgfsetdash{}{0pt}%
\pgfsys@defobject{currentmarker}{\pgfqpoint{0.000000in}{-0.027778in}}{\pgfqpoint{0.000000in}{0.000000in}}{%
\pgfpathmoveto{\pgfqpoint{0.000000in}{0.000000in}}%
\pgfpathlineto{\pgfqpoint{0.000000in}{-0.027778in}}%
\pgfusepath{stroke,fill}%
}%
\begin{pgfscope}%
\pgfsys@transformshift{0.987035in}{0.400278in}%
\pgfsys@useobject{currentmarker}{}%
\end{pgfscope}%
\end{pgfscope}%
\begin{pgfscope}%
\pgfsetbuttcap%
\pgfsetroundjoin%
\definecolor{currentfill}{rgb}{0.000000,0.000000,0.000000}%
\pgfsetfillcolor{currentfill}%
\pgfsetlinewidth{0.602250pt}%
\definecolor{currentstroke}{rgb}{0.000000,0.000000,0.000000}%
\pgfsetstrokecolor{currentstroke}%
\pgfsetdash{}{0pt}%
\pgfsys@defobject{currentmarker}{\pgfqpoint{0.000000in}{-0.027778in}}{\pgfqpoint{0.000000in}{0.000000in}}{%
\pgfpathmoveto{\pgfqpoint{0.000000in}{0.000000in}}%
\pgfpathlineto{\pgfqpoint{0.000000in}{-0.027778in}}%
\pgfusepath{stroke,fill}%
}%
\begin{pgfscope}%
\pgfsys@transformshift{1.090971in}{0.400278in}%
\pgfsys@useobject{currentmarker}{}%
\end{pgfscope}%
\end{pgfscope}%
\begin{pgfscope}%
\pgfsetbuttcap%
\pgfsetroundjoin%
\definecolor{currentfill}{rgb}{0.000000,0.000000,0.000000}%
\pgfsetfillcolor{currentfill}%
\pgfsetlinewidth{0.602250pt}%
\definecolor{currentstroke}{rgb}{0.000000,0.000000,0.000000}%
\pgfsetstrokecolor{currentstroke}%
\pgfsetdash{}{0pt}%
\pgfsys@defobject{currentmarker}{\pgfqpoint{0.000000in}{-0.027778in}}{\pgfqpoint{0.000000in}{0.000000in}}{%
\pgfpathmoveto{\pgfqpoint{0.000000in}{0.000000in}}%
\pgfpathlineto{\pgfqpoint{0.000000in}{-0.027778in}}%
\pgfusepath{stroke,fill}%
}%
\begin{pgfscope}%
\pgfsys@transformshift{1.171589in}{0.400278in}%
\pgfsys@useobject{currentmarker}{}%
\end{pgfscope}%
\end{pgfscope}%
\begin{pgfscope}%
\pgfsetbuttcap%
\pgfsetroundjoin%
\definecolor{currentfill}{rgb}{0.000000,0.000000,0.000000}%
\pgfsetfillcolor{currentfill}%
\pgfsetlinewidth{0.602250pt}%
\definecolor{currentstroke}{rgb}{0.000000,0.000000,0.000000}%
\pgfsetstrokecolor{currentstroke}%
\pgfsetdash{}{0pt}%
\pgfsys@defobject{currentmarker}{\pgfqpoint{0.000000in}{-0.027778in}}{\pgfqpoint{0.000000in}{0.000000in}}{%
\pgfpathmoveto{\pgfqpoint{0.000000in}{0.000000in}}%
\pgfpathlineto{\pgfqpoint{0.000000in}{-0.027778in}}%
\pgfusepath{stroke,fill}%
}%
\begin{pgfscope}%
\pgfsys@transformshift{1.237460in}{0.400278in}%
\pgfsys@useobject{currentmarker}{}%
\end{pgfscope}%
\end{pgfscope}%
\begin{pgfscope}%
\pgfsetbuttcap%
\pgfsetroundjoin%
\definecolor{currentfill}{rgb}{0.000000,0.000000,0.000000}%
\pgfsetfillcolor{currentfill}%
\pgfsetlinewidth{0.602250pt}%
\definecolor{currentstroke}{rgb}{0.000000,0.000000,0.000000}%
\pgfsetstrokecolor{currentstroke}%
\pgfsetdash{}{0pt}%
\pgfsys@defobject{currentmarker}{\pgfqpoint{0.000000in}{-0.027778in}}{\pgfqpoint{0.000000in}{0.000000in}}{%
\pgfpathmoveto{\pgfqpoint{0.000000in}{0.000000in}}%
\pgfpathlineto{\pgfqpoint{0.000000in}{-0.027778in}}%
\pgfusepath{stroke,fill}%
}%
\begin{pgfscope}%
\pgfsys@transformshift{1.293152in}{0.400278in}%
\pgfsys@useobject{currentmarker}{}%
\end{pgfscope}%
\end{pgfscope}%
\begin{pgfscope}%
\pgfsetbuttcap%
\pgfsetroundjoin%
\definecolor{currentfill}{rgb}{0.000000,0.000000,0.000000}%
\pgfsetfillcolor{currentfill}%
\pgfsetlinewidth{0.602250pt}%
\definecolor{currentstroke}{rgb}{0.000000,0.000000,0.000000}%
\pgfsetstrokecolor{currentstroke}%
\pgfsetdash{}{0pt}%
\pgfsys@defobject{currentmarker}{\pgfqpoint{0.000000in}{-0.027778in}}{\pgfqpoint{0.000000in}{0.000000in}}{%
\pgfpathmoveto{\pgfqpoint{0.000000in}{0.000000in}}%
\pgfpathlineto{\pgfqpoint{0.000000in}{-0.027778in}}%
\pgfusepath{stroke,fill}%
}%
\begin{pgfscope}%
\pgfsys@transformshift{1.341395in}{0.400278in}%
\pgfsys@useobject{currentmarker}{}%
\end{pgfscope}%
\end{pgfscope}%
\begin{pgfscope}%
\pgfsetbuttcap%
\pgfsetroundjoin%
\definecolor{currentfill}{rgb}{0.000000,0.000000,0.000000}%
\pgfsetfillcolor{currentfill}%
\pgfsetlinewidth{0.602250pt}%
\definecolor{currentstroke}{rgb}{0.000000,0.000000,0.000000}%
\pgfsetstrokecolor{currentstroke}%
\pgfsetdash{}{0pt}%
\pgfsys@defobject{currentmarker}{\pgfqpoint{0.000000in}{-0.027778in}}{\pgfqpoint{0.000000in}{0.000000in}}{%
\pgfpathmoveto{\pgfqpoint{0.000000in}{0.000000in}}%
\pgfpathlineto{\pgfqpoint{0.000000in}{-0.027778in}}%
\pgfusepath{stroke,fill}%
}%
\begin{pgfscope}%
\pgfsys@transformshift{1.383949in}{0.400278in}%
\pgfsys@useobject{currentmarker}{}%
\end{pgfscope}%
\end{pgfscope}%
\begin{pgfscope}%
\pgfsetbuttcap%
\pgfsetroundjoin%
\definecolor{currentfill}{rgb}{0.000000,0.000000,0.000000}%
\pgfsetfillcolor{currentfill}%
\pgfsetlinewidth{0.602250pt}%
\definecolor{currentstroke}{rgb}{0.000000,0.000000,0.000000}%
\pgfsetstrokecolor{currentstroke}%
\pgfsetdash{}{0pt}%
\pgfsys@defobject{currentmarker}{\pgfqpoint{0.000000in}{-0.027778in}}{\pgfqpoint{0.000000in}{0.000000in}}{%
\pgfpathmoveto{\pgfqpoint{0.000000in}{0.000000in}}%
\pgfpathlineto{\pgfqpoint{0.000000in}{-0.027778in}}%
\pgfusepath{stroke,fill}%
}%
\begin{pgfscope}%
\pgfsys@transformshift{1.672439in}{0.400278in}%
\pgfsys@useobject{currentmarker}{}%
\end{pgfscope}%
\end{pgfscope}%
\begin{pgfscope}%
\pgfsetbuttcap%
\pgfsetroundjoin%
\definecolor{currentfill}{rgb}{0.000000,0.000000,0.000000}%
\pgfsetfillcolor{currentfill}%
\pgfsetlinewidth{0.602250pt}%
\definecolor{currentstroke}{rgb}{0.000000,0.000000,0.000000}%
\pgfsetstrokecolor{currentstroke}%
\pgfsetdash{}{0pt}%
\pgfsys@defobject{currentmarker}{\pgfqpoint{0.000000in}{-0.027778in}}{\pgfqpoint{0.000000in}{0.000000in}}{%
\pgfpathmoveto{\pgfqpoint{0.000000in}{0.000000in}}%
\pgfpathlineto{\pgfqpoint{0.000000in}{-0.027778in}}%
\pgfusepath{stroke,fill}%
}%
\begin{pgfscope}%
\pgfsys@transformshift{1.818928in}{0.400278in}%
\pgfsys@useobject{currentmarker}{}%
\end{pgfscope}%
\end{pgfscope}%
\begin{pgfscope}%
\pgfsetbuttcap%
\pgfsetroundjoin%
\definecolor{currentfill}{rgb}{0.000000,0.000000,0.000000}%
\pgfsetfillcolor{currentfill}%
\pgfsetlinewidth{0.602250pt}%
\definecolor{currentstroke}{rgb}{0.000000,0.000000,0.000000}%
\pgfsetstrokecolor{currentstroke}%
\pgfsetdash{}{0pt}%
\pgfsys@defobject{currentmarker}{\pgfqpoint{0.000000in}{-0.027778in}}{\pgfqpoint{0.000000in}{0.000000in}}{%
\pgfpathmoveto{\pgfqpoint{0.000000in}{0.000000in}}%
\pgfpathlineto{\pgfqpoint{0.000000in}{-0.027778in}}%
\pgfusepath{stroke,fill}%
}%
\begin{pgfscope}%
\pgfsys@transformshift{1.922864in}{0.400278in}%
\pgfsys@useobject{currentmarker}{}%
\end{pgfscope}%
\end{pgfscope}%
\begin{pgfscope}%
\pgfsetbuttcap%
\pgfsetroundjoin%
\definecolor{currentfill}{rgb}{0.000000,0.000000,0.000000}%
\pgfsetfillcolor{currentfill}%
\pgfsetlinewidth{0.602250pt}%
\definecolor{currentstroke}{rgb}{0.000000,0.000000,0.000000}%
\pgfsetstrokecolor{currentstroke}%
\pgfsetdash{}{0pt}%
\pgfsys@defobject{currentmarker}{\pgfqpoint{0.000000in}{-0.027778in}}{\pgfqpoint{0.000000in}{0.000000in}}{%
\pgfpathmoveto{\pgfqpoint{0.000000in}{0.000000in}}%
\pgfpathlineto{\pgfqpoint{0.000000in}{-0.027778in}}%
\pgfusepath{stroke,fill}%
}%
\begin{pgfscope}%
\pgfsys@transformshift{2.003483in}{0.400278in}%
\pgfsys@useobject{currentmarker}{}%
\end{pgfscope}%
\end{pgfscope}%
\begin{pgfscope}%
\pgfsetbuttcap%
\pgfsetroundjoin%
\definecolor{currentfill}{rgb}{0.000000,0.000000,0.000000}%
\pgfsetfillcolor{currentfill}%
\pgfsetlinewidth{0.602250pt}%
\definecolor{currentstroke}{rgb}{0.000000,0.000000,0.000000}%
\pgfsetstrokecolor{currentstroke}%
\pgfsetdash{}{0pt}%
\pgfsys@defobject{currentmarker}{\pgfqpoint{0.000000in}{-0.027778in}}{\pgfqpoint{0.000000in}{0.000000in}}{%
\pgfpathmoveto{\pgfqpoint{0.000000in}{0.000000in}}%
\pgfpathlineto{\pgfqpoint{0.000000in}{-0.027778in}}%
\pgfusepath{stroke,fill}%
}%
\begin{pgfscope}%
\pgfsys@transformshift{2.069353in}{0.400278in}%
\pgfsys@useobject{currentmarker}{}%
\end{pgfscope}%
\end{pgfscope}%
\begin{pgfscope}%
\pgfsetbuttcap%
\pgfsetroundjoin%
\definecolor{currentfill}{rgb}{0.000000,0.000000,0.000000}%
\pgfsetfillcolor{currentfill}%
\pgfsetlinewidth{0.602250pt}%
\definecolor{currentstroke}{rgb}{0.000000,0.000000,0.000000}%
\pgfsetstrokecolor{currentstroke}%
\pgfsetdash{}{0pt}%
\pgfsys@defobject{currentmarker}{\pgfqpoint{0.000000in}{-0.027778in}}{\pgfqpoint{0.000000in}{0.000000in}}{%
\pgfpathmoveto{\pgfqpoint{0.000000in}{0.000000in}}%
\pgfpathlineto{\pgfqpoint{0.000000in}{-0.027778in}}%
\pgfusepath{stroke,fill}%
}%
\begin{pgfscope}%
\pgfsys@transformshift{2.125046in}{0.400278in}%
\pgfsys@useobject{currentmarker}{}%
\end{pgfscope}%
\end{pgfscope}%
\begin{pgfscope}%
\pgfsetbuttcap%
\pgfsetroundjoin%
\definecolor{currentfill}{rgb}{0.000000,0.000000,0.000000}%
\pgfsetfillcolor{currentfill}%
\pgfsetlinewidth{0.602250pt}%
\definecolor{currentstroke}{rgb}{0.000000,0.000000,0.000000}%
\pgfsetstrokecolor{currentstroke}%
\pgfsetdash{}{0pt}%
\pgfsys@defobject{currentmarker}{\pgfqpoint{0.000000in}{-0.027778in}}{\pgfqpoint{0.000000in}{0.000000in}}{%
\pgfpathmoveto{\pgfqpoint{0.000000in}{0.000000in}}%
\pgfpathlineto{\pgfqpoint{0.000000in}{-0.027778in}}%
\pgfusepath{stroke,fill}%
}%
\begin{pgfscope}%
\pgfsys@transformshift{2.173289in}{0.400278in}%
\pgfsys@useobject{currentmarker}{}%
\end{pgfscope}%
\end{pgfscope}%
\begin{pgfscope}%
\pgfsetbuttcap%
\pgfsetroundjoin%
\definecolor{currentfill}{rgb}{0.000000,0.000000,0.000000}%
\pgfsetfillcolor{currentfill}%
\pgfsetlinewidth{0.602250pt}%
\definecolor{currentstroke}{rgb}{0.000000,0.000000,0.000000}%
\pgfsetstrokecolor{currentstroke}%
\pgfsetdash{}{0pt}%
\pgfsys@defobject{currentmarker}{\pgfqpoint{0.000000in}{-0.027778in}}{\pgfqpoint{0.000000in}{0.000000in}}{%
\pgfpathmoveto{\pgfqpoint{0.000000in}{0.000000in}}%
\pgfpathlineto{\pgfqpoint{0.000000in}{-0.027778in}}%
\pgfusepath{stroke,fill}%
}%
\begin{pgfscope}%
\pgfsys@transformshift{2.215842in}{0.400278in}%
\pgfsys@useobject{currentmarker}{}%
\end{pgfscope}%
\end{pgfscope}%
\begin{pgfscope}%
\pgfsetbuttcap%
\pgfsetroundjoin%
\definecolor{currentfill}{rgb}{0.000000,0.000000,0.000000}%
\pgfsetfillcolor{currentfill}%
\pgfsetlinewidth{0.602250pt}%
\definecolor{currentstroke}{rgb}{0.000000,0.000000,0.000000}%
\pgfsetstrokecolor{currentstroke}%
\pgfsetdash{}{0pt}%
\pgfsys@defobject{currentmarker}{\pgfqpoint{0.000000in}{-0.027778in}}{\pgfqpoint{0.000000in}{0.000000in}}{%
\pgfpathmoveto{\pgfqpoint{0.000000in}{0.000000in}}%
\pgfpathlineto{\pgfqpoint{0.000000in}{-0.027778in}}%
\pgfusepath{stroke,fill}%
}%
\begin{pgfscope}%
\pgfsys@transformshift{2.504332in}{0.400278in}%
\pgfsys@useobject{currentmarker}{}%
\end{pgfscope}%
\end{pgfscope}%
\begin{pgfscope}%
\pgfsetbuttcap%
\pgfsetroundjoin%
\definecolor{currentfill}{rgb}{0.000000,0.000000,0.000000}%
\pgfsetfillcolor{currentfill}%
\pgfsetlinewidth{0.602250pt}%
\definecolor{currentstroke}{rgb}{0.000000,0.000000,0.000000}%
\pgfsetstrokecolor{currentstroke}%
\pgfsetdash{}{0pt}%
\pgfsys@defobject{currentmarker}{\pgfqpoint{0.000000in}{-0.027778in}}{\pgfqpoint{0.000000in}{0.000000in}}{%
\pgfpathmoveto{\pgfqpoint{0.000000in}{0.000000in}}%
\pgfpathlineto{\pgfqpoint{0.000000in}{-0.027778in}}%
\pgfusepath{stroke,fill}%
}%
\begin{pgfscope}%
\pgfsys@transformshift{2.650821in}{0.400278in}%
\pgfsys@useobject{currentmarker}{}%
\end{pgfscope}%
\end{pgfscope}%
\begin{pgfscope}%
\pgfsetbuttcap%
\pgfsetroundjoin%
\definecolor{currentfill}{rgb}{0.000000,0.000000,0.000000}%
\pgfsetfillcolor{currentfill}%
\pgfsetlinewidth{0.602250pt}%
\definecolor{currentstroke}{rgb}{0.000000,0.000000,0.000000}%
\pgfsetstrokecolor{currentstroke}%
\pgfsetdash{}{0pt}%
\pgfsys@defobject{currentmarker}{\pgfqpoint{0.000000in}{-0.027778in}}{\pgfqpoint{0.000000in}{0.000000in}}{%
\pgfpathmoveto{\pgfqpoint{0.000000in}{0.000000in}}%
\pgfpathlineto{\pgfqpoint{0.000000in}{-0.027778in}}%
\pgfusepath{stroke,fill}%
}%
\begin{pgfscope}%
\pgfsys@transformshift{2.754757in}{0.400278in}%
\pgfsys@useobject{currentmarker}{}%
\end{pgfscope}%
\end{pgfscope}%
\begin{pgfscope}%
\pgfsetbuttcap%
\pgfsetroundjoin%
\definecolor{currentfill}{rgb}{0.000000,0.000000,0.000000}%
\pgfsetfillcolor{currentfill}%
\pgfsetlinewidth{0.602250pt}%
\definecolor{currentstroke}{rgb}{0.000000,0.000000,0.000000}%
\pgfsetstrokecolor{currentstroke}%
\pgfsetdash{}{0pt}%
\pgfsys@defobject{currentmarker}{\pgfqpoint{0.000000in}{-0.027778in}}{\pgfqpoint{0.000000in}{0.000000in}}{%
\pgfpathmoveto{\pgfqpoint{0.000000in}{0.000000in}}%
\pgfpathlineto{\pgfqpoint{0.000000in}{-0.027778in}}%
\pgfusepath{stroke,fill}%
}%
\begin{pgfscope}%
\pgfsys@transformshift{2.835376in}{0.400278in}%
\pgfsys@useobject{currentmarker}{}%
\end{pgfscope}%
\end{pgfscope}%
\begin{pgfscope}%
\pgfsetbuttcap%
\pgfsetroundjoin%
\definecolor{currentfill}{rgb}{0.000000,0.000000,0.000000}%
\pgfsetfillcolor{currentfill}%
\pgfsetlinewidth{0.602250pt}%
\definecolor{currentstroke}{rgb}{0.000000,0.000000,0.000000}%
\pgfsetstrokecolor{currentstroke}%
\pgfsetdash{}{0pt}%
\pgfsys@defobject{currentmarker}{\pgfqpoint{0.000000in}{-0.027778in}}{\pgfqpoint{0.000000in}{0.000000in}}{%
\pgfpathmoveto{\pgfqpoint{0.000000in}{0.000000in}}%
\pgfpathlineto{\pgfqpoint{0.000000in}{-0.027778in}}%
\pgfusepath{stroke,fill}%
}%
\begin{pgfscope}%
\pgfsys@transformshift{2.901246in}{0.400278in}%
\pgfsys@useobject{currentmarker}{}%
\end{pgfscope}%
\end{pgfscope}%
\begin{pgfscope}%
\pgfsetbuttcap%
\pgfsetroundjoin%
\definecolor{currentfill}{rgb}{0.000000,0.000000,0.000000}%
\pgfsetfillcolor{currentfill}%
\pgfsetlinewidth{0.602250pt}%
\definecolor{currentstroke}{rgb}{0.000000,0.000000,0.000000}%
\pgfsetstrokecolor{currentstroke}%
\pgfsetdash{}{0pt}%
\pgfsys@defobject{currentmarker}{\pgfqpoint{0.000000in}{-0.027778in}}{\pgfqpoint{0.000000in}{0.000000in}}{%
\pgfpathmoveto{\pgfqpoint{0.000000in}{0.000000in}}%
\pgfpathlineto{\pgfqpoint{0.000000in}{-0.027778in}}%
\pgfusepath{stroke,fill}%
}%
\begin{pgfscope}%
\pgfsys@transformshift{2.956939in}{0.400278in}%
\pgfsys@useobject{currentmarker}{}%
\end{pgfscope}%
\end{pgfscope}%
\begin{pgfscope}%
\pgfsetbuttcap%
\pgfsetroundjoin%
\definecolor{currentfill}{rgb}{0.000000,0.000000,0.000000}%
\pgfsetfillcolor{currentfill}%
\pgfsetlinewidth{0.602250pt}%
\definecolor{currentstroke}{rgb}{0.000000,0.000000,0.000000}%
\pgfsetstrokecolor{currentstroke}%
\pgfsetdash{}{0pt}%
\pgfsys@defobject{currentmarker}{\pgfqpoint{0.000000in}{-0.027778in}}{\pgfqpoint{0.000000in}{0.000000in}}{%
\pgfpathmoveto{\pgfqpoint{0.000000in}{0.000000in}}%
\pgfpathlineto{\pgfqpoint{0.000000in}{-0.027778in}}%
\pgfusepath{stroke,fill}%
}%
\begin{pgfscope}%
\pgfsys@transformshift{3.005182in}{0.400278in}%
\pgfsys@useobject{currentmarker}{}%
\end{pgfscope}%
\end{pgfscope}%
\begin{pgfscope}%
\pgfsetbuttcap%
\pgfsetroundjoin%
\definecolor{currentfill}{rgb}{0.000000,0.000000,0.000000}%
\pgfsetfillcolor{currentfill}%
\pgfsetlinewidth{0.602250pt}%
\definecolor{currentstroke}{rgb}{0.000000,0.000000,0.000000}%
\pgfsetstrokecolor{currentstroke}%
\pgfsetdash{}{0pt}%
\pgfsys@defobject{currentmarker}{\pgfqpoint{0.000000in}{-0.027778in}}{\pgfqpoint{0.000000in}{0.000000in}}{%
\pgfpathmoveto{\pgfqpoint{0.000000in}{0.000000in}}%
\pgfpathlineto{\pgfqpoint{0.000000in}{-0.027778in}}%
\pgfusepath{stroke,fill}%
}%
\begin{pgfscope}%
\pgfsys@transformshift{3.047735in}{0.400278in}%
\pgfsys@useobject{currentmarker}{}%
\end{pgfscope}%
\end{pgfscope}%
\begin{pgfscope}%
\pgfsetbuttcap%
\pgfsetroundjoin%
\definecolor{currentfill}{rgb}{0.000000,0.000000,0.000000}%
\pgfsetfillcolor{currentfill}%
\pgfsetlinewidth{0.602250pt}%
\definecolor{currentstroke}{rgb}{0.000000,0.000000,0.000000}%
\pgfsetstrokecolor{currentstroke}%
\pgfsetdash{}{0pt}%
\pgfsys@defobject{currentmarker}{\pgfqpoint{0.000000in}{-0.027778in}}{\pgfqpoint{0.000000in}{0.000000in}}{%
\pgfpathmoveto{\pgfqpoint{0.000000in}{0.000000in}}%
\pgfpathlineto{\pgfqpoint{0.000000in}{-0.027778in}}%
\pgfusepath{stroke,fill}%
}%
\begin{pgfscope}%
\pgfsys@transformshift{3.336226in}{0.400278in}%
\pgfsys@useobject{currentmarker}{}%
\end{pgfscope}%
\end{pgfscope}%
\begin{pgfscope}%
\pgfsetbuttcap%
\pgfsetroundjoin%
\definecolor{currentfill}{rgb}{0.000000,0.000000,0.000000}%
\pgfsetfillcolor{currentfill}%
\pgfsetlinewidth{0.602250pt}%
\definecolor{currentstroke}{rgb}{0.000000,0.000000,0.000000}%
\pgfsetstrokecolor{currentstroke}%
\pgfsetdash{}{0pt}%
\pgfsys@defobject{currentmarker}{\pgfqpoint{0.000000in}{-0.027778in}}{\pgfqpoint{0.000000in}{0.000000in}}{%
\pgfpathmoveto{\pgfqpoint{0.000000in}{0.000000in}}%
\pgfpathlineto{\pgfqpoint{0.000000in}{-0.027778in}}%
\pgfusepath{stroke,fill}%
}%
\begin{pgfscope}%
\pgfsys@transformshift{3.482715in}{0.400278in}%
\pgfsys@useobject{currentmarker}{}%
\end{pgfscope}%
\end{pgfscope}%
\begin{pgfscope}%
\pgfsetbuttcap%
\pgfsetroundjoin%
\definecolor{currentfill}{rgb}{0.000000,0.000000,0.000000}%
\pgfsetfillcolor{currentfill}%
\pgfsetlinewidth{0.602250pt}%
\definecolor{currentstroke}{rgb}{0.000000,0.000000,0.000000}%
\pgfsetstrokecolor{currentstroke}%
\pgfsetdash{}{0pt}%
\pgfsys@defobject{currentmarker}{\pgfqpoint{0.000000in}{-0.027778in}}{\pgfqpoint{0.000000in}{0.000000in}}{%
\pgfpathmoveto{\pgfqpoint{0.000000in}{0.000000in}}%
\pgfpathlineto{\pgfqpoint{0.000000in}{-0.027778in}}%
\pgfusepath{stroke,fill}%
}%
\begin{pgfscope}%
\pgfsys@transformshift{3.586650in}{0.400278in}%
\pgfsys@useobject{currentmarker}{}%
\end{pgfscope}%
\end{pgfscope}%
\begin{pgfscope}%
\pgfsetbuttcap%
\pgfsetroundjoin%
\definecolor{currentfill}{rgb}{0.000000,0.000000,0.000000}%
\pgfsetfillcolor{currentfill}%
\pgfsetlinewidth{0.602250pt}%
\definecolor{currentstroke}{rgb}{0.000000,0.000000,0.000000}%
\pgfsetstrokecolor{currentstroke}%
\pgfsetdash{}{0pt}%
\pgfsys@defobject{currentmarker}{\pgfqpoint{0.000000in}{-0.027778in}}{\pgfqpoint{0.000000in}{0.000000in}}{%
\pgfpathmoveto{\pgfqpoint{0.000000in}{0.000000in}}%
\pgfpathlineto{\pgfqpoint{0.000000in}{-0.027778in}}%
\pgfusepath{stroke,fill}%
}%
\begin{pgfscope}%
\pgfsys@transformshift{3.667269in}{0.400278in}%
\pgfsys@useobject{currentmarker}{}%
\end{pgfscope}%
\end{pgfscope}%
\begin{pgfscope}%
\pgfsetbuttcap%
\pgfsetroundjoin%
\definecolor{currentfill}{rgb}{0.000000,0.000000,0.000000}%
\pgfsetfillcolor{currentfill}%
\pgfsetlinewidth{0.602250pt}%
\definecolor{currentstroke}{rgb}{0.000000,0.000000,0.000000}%
\pgfsetstrokecolor{currentstroke}%
\pgfsetdash{}{0pt}%
\pgfsys@defobject{currentmarker}{\pgfqpoint{0.000000in}{-0.027778in}}{\pgfqpoint{0.000000in}{0.000000in}}{%
\pgfpathmoveto{\pgfqpoint{0.000000in}{0.000000in}}%
\pgfpathlineto{\pgfqpoint{0.000000in}{-0.027778in}}%
\pgfusepath{stroke,fill}%
}%
\begin{pgfscope}%
\pgfsys@transformshift{3.733139in}{0.400278in}%
\pgfsys@useobject{currentmarker}{}%
\end{pgfscope}%
\end{pgfscope}%
\begin{pgfscope}%
\pgfsetbuttcap%
\pgfsetroundjoin%
\definecolor{currentfill}{rgb}{0.000000,0.000000,0.000000}%
\pgfsetfillcolor{currentfill}%
\pgfsetlinewidth{0.602250pt}%
\definecolor{currentstroke}{rgb}{0.000000,0.000000,0.000000}%
\pgfsetstrokecolor{currentstroke}%
\pgfsetdash{}{0pt}%
\pgfsys@defobject{currentmarker}{\pgfqpoint{0.000000in}{-0.027778in}}{\pgfqpoint{0.000000in}{0.000000in}}{%
\pgfpathmoveto{\pgfqpoint{0.000000in}{0.000000in}}%
\pgfpathlineto{\pgfqpoint{0.000000in}{-0.027778in}}%
\pgfusepath{stroke,fill}%
}%
\begin{pgfscope}%
\pgfsys@transformshift{3.788832in}{0.400278in}%
\pgfsys@useobject{currentmarker}{}%
\end{pgfscope}%
\end{pgfscope}%
\begin{pgfscope}%
\pgfsetbuttcap%
\pgfsetroundjoin%
\definecolor{currentfill}{rgb}{0.000000,0.000000,0.000000}%
\pgfsetfillcolor{currentfill}%
\pgfsetlinewidth{0.602250pt}%
\definecolor{currentstroke}{rgb}{0.000000,0.000000,0.000000}%
\pgfsetstrokecolor{currentstroke}%
\pgfsetdash{}{0pt}%
\pgfsys@defobject{currentmarker}{\pgfqpoint{0.000000in}{-0.027778in}}{\pgfqpoint{0.000000in}{0.000000in}}{%
\pgfpathmoveto{\pgfqpoint{0.000000in}{0.000000in}}%
\pgfpathlineto{\pgfqpoint{0.000000in}{-0.027778in}}%
\pgfusepath{stroke,fill}%
}%
\begin{pgfscope}%
\pgfsys@transformshift{3.837075in}{0.400278in}%
\pgfsys@useobject{currentmarker}{}%
\end{pgfscope}%
\end{pgfscope}%
\begin{pgfscope}%
\pgfsetbuttcap%
\pgfsetroundjoin%
\definecolor{currentfill}{rgb}{0.000000,0.000000,0.000000}%
\pgfsetfillcolor{currentfill}%
\pgfsetlinewidth{0.602250pt}%
\definecolor{currentstroke}{rgb}{0.000000,0.000000,0.000000}%
\pgfsetstrokecolor{currentstroke}%
\pgfsetdash{}{0pt}%
\pgfsys@defobject{currentmarker}{\pgfqpoint{0.000000in}{-0.027778in}}{\pgfqpoint{0.000000in}{0.000000in}}{%
\pgfpathmoveto{\pgfqpoint{0.000000in}{0.000000in}}%
\pgfpathlineto{\pgfqpoint{0.000000in}{-0.027778in}}%
\pgfusepath{stroke,fill}%
}%
\begin{pgfscope}%
\pgfsys@transformshift{3.879629in}{0.400278in}%
\pgfsys@useobject{currentmarker}{}%
\end{pgfscope}%
\end{pgfscope}%
\begin{pgfscope}%
\definecolor{textcolor}{rgb}{0.000000,0.000000,0.000000}%
\pgfsetstrokecolor{textcolor}%
\pgfsetfillcolor{textcolor}%
\pgftext[x=2.327604in,y=0.247500in,,top]{\color{textcolor}\rmfamily\fontsize{8.000000}{9.600000}\selectfont Arithmetic intensity (FLOPs/byte)}%
\end{pgfscope}%
\begin{pgfscope}%
\pgfsetbuttcap%
\pgfsetroundjoin%
\definecolor{currentfill}{rgb}{0.000000,0.000000,0.000000}%
\pgfsetfillcolor{currentfill}%
\pgfsetlinewidth{0.803000pt}%
\definecolor{currentstroke}{rgb}{0.000000,0.000000,0.000000}%
\pgfsetstrokecolor{currentstroke}%
\pgfsetdash{}{0pt}%
\pgfsys@defobject{currentmarker}{\pgfqpoint{-0.048611in}{0.000000in}}{\pgfqpoint{-0.000000in}{0.000000in}}{%
\pgfpathmoveto{\pgfqpoint{-0.000000in}{0.000000in}}%
\pgfpathlineto{\pgfqpoint{-0.048611in}{0.000000in}}%
\pgfusepath{stroke,fill}%
}%
\begin{pgfscope}%
\pgfsys@transformshift{0.567708in}{0.400278in}%
\pgfsys@useobject{currentmarker}{}%
\end{pgfscope}%
\end{pgfscope}%
\begin{pgfscope}%
\pgfsetbuttcap%
\pgfsetroundjoin%
\definecolor{currentfill}{rgb}{0.000000,0.000000,0.000000}%
\pgfsetfillcolor{currentfill}%
\pgfsetlinewidth{0.803000pt}%
\definecolor{currentstroke}{rgb}{0.000000,0.000000,0.000000}%
\pgfsetstrokecolor{currentstroke}%
\pgfsetdash{}{0pt}%
\pgfsys@defobject{currentmarker}{\pgfqpoint{-0.048611in}{0.000000in}}{\pgfqpoint{-0.000000in}{0.000000in}}{%
\pgfpathmoveto{\pgfqpoint{-0.000000in}{0.000000in}}%
\pgfpathlineto{\pgfqpoint{-0.048611in}{0.000000in}}%
\pgfusepath{stroke,fill}%
}%
\begin{pgfscope}%
\pgfsys@transformshift{0.567708in}{1.118156in}%
\pgfsys@useobject{currentmarker}{}%
\end{pgfscope}%
\end{pgfscope}%
\begin{pgfscope}%
\pgfsetbuttcap%
\pgfsetroundjoin%
\definecolor{currentfill}{rgb}{0.000000,0.000000,0.000000}%
\pgfsetfillcolor{currentfill}%
\pgfsetlinewidth{0.803000pt}%
\definecolor{currentstroke}{rgb}{0.000000,0.000000,0.000000}%
\pgfsetstrokecolor{currentstroke}%
\pgfsetdash{}{0pt}%
\pgfsys@defobject{currentmarker}{\pgfqpoint{-0.048611in}{0.000000in}}{\pgfqpoint{-0.000000in}{0.000000in}}{%
\pgfpathmoveto{\pgfqpoint{-0.000000in}{0.000000in}}%
\pgfpathlineto{\pgfqpoint{-0.048611in}{0.000000in}}%
\pgfusepath{stroke,fill}%
}%
\begin{pgfscope}%
\pgfsys@transformshift{0.567708in}{1.836035in}%
\pgfsys@useobject{currentmarker}{}%
\end{pgfscope}%
\end{pgfscope}%
\begin{pgfscope}%
\pgfsetbuttcap%
\pgfsetroundjoin%
\definecolor{currentfill}{rgb}{0.000000,0.000000,0.000000}%
\pgfsetfillcolor{currentfill}%
\pgfsetlinewidth{0.803000pt}%
\definecolor{currentstroke}{rgb}{0.000000,0.000000,0.000000}%
\pgfsetstrokecolor{currentstroke}%
\pgfsetdash{}{0pt}%
\pgfsys@defobject{currentmarker}{\pgfqpoint{-0.048611in}{0.000000in}}{\pgfqpoint{-0.000000in}{0.000000in}}{%
\pgfpathmoveto{\pgfqpoint{-0.000000in}{0.000000in}}%
\pgfpathlineto{\pgfqpoint{-0.048611in}{0.000000in}}%
\pgfusepath{stroke,fill}%
}%
\begin{pgfscope}%
\pgfsys@transformshift{0.567708in}{2.553913in}%
\pgfsys@useobject{currentmarker}{}%
\end{pgfscope}%
\end{pgfscope}%
\begin{pgfscope}%
\pgfsetbuttcap%
\pgfsetroundjoin%
\definecolor{currentfill}{rgb}{0.000000,0.000000,0.000000}%
\pgfsetfillcolor{currentfill}%
\pgfsetlinewidth{0.602250pt}%
\definecolor{currentstroke}{rgb}{0.000000,0.000000,0.000000}%
\pgfsetstrokecolor{currentstroke}%
\pgfsetdash{}{0pt}%
\pgfsys@defobject{currentmarker}{\pgfqpoint{-0.027778in}{0.000000in}}{\pgfqpoint{-0.000000in}{0.000000in}}{%
\pgfpathmoveto{\pgfqpoint{-0.000000in}{0.000000in}}%
\pgfpathlineto{\pgfqpoint{-0.027778in}{0.000000in}}%
\pgfusepath{stroke,fill}%
}%
\begin{pgfscope}%
\pgfsys@transformshift{0.567708in}{0.616381in}%
\pgfsys@useobject{currentmarker}{}%
\end{pgfscope}%
\end{pgfscope}%
\begin{pgfscope}%
\pgfsetbuttcap%
\pgfsetroundjoin%
\definecolor{currentfill}{rgb}{0.000000,0.000000,0.000000}%
\pgfsetfillcolor{currentfill}%
\pgfsetlinewidth{0.602250pt}%
\definecolor{currentstroke}{rgb}{0.000000,0.000000,0.000000}%
\pgfsetstrokecolor{currentstroke}%
\pgfsetdash{}{0pt}%
\pgfsys@defobject{currentmarker}{\pgfqpoint{-0.027778in}{0.000000in}}{\pgfqpoint{-0.000000in}{0.000000in}}{%
\pgfpathmoveto{\pgfqpoint{-0.000000in}{0.000000in}}%
\pgfpathlineto{\pgfqpoint{-0.027778in}{0.000000in}}%
\pgfusepath{stroke,fill}%
}%
\begin{pgfscope}%
\pgfsys@transformshift{0.567708in}{0.742793in}%
\pgfsys@useobject{currentmarker}{}%
\end{pgfscope}%
\end{pgfscope}%
\begin{pgfscope}%
\pgfsetbuttcap%
\pgfsetroundjoin%
\definecolor{currentfill}{rgb}{0.000000,0.000000,0.000000}%
\pgfsetfillcolor{currentfill}%
\pgfsetlinewidth{0.602250pt}%
\definecolor{currentstroke}{rgb}{0.000000,0.000000,0.000000}%
\pgfsetstrokecolor{currentstroke}%
\pgfsetdash{}{0pt}%
\pgfsys@defobject{currentmarker}{\pgfqpoint{-0.027778in}{0.000000in}}{\pgfqpoint{-0.000000in}{0.000000in}}{%
\pgfpathmoveto{\pgfqpoint{-0.000000in}{0.000000in}}%
\pgfpathlineto{\pgfqpoint{-0.027778in}{0.000000in}}%
\pgfusepath{stroke,fill}%
}%
\begin{pgfscope}%
\pgfsys@transformshift{0.567708in}{0.832484in}%
\pgfsys@useobject{currentmarker}{}%
\end{pgfscope}%
\end{pgfscope}%
\begin{pgfscope}%
\pgfsetbuttcap%
\pgfsetroundjoin%
\definecolor{currentfill}{rgb}{0.000000,0.000000,0.000000}%
\pgfsetfillcolor{currentfill}%
\pgfsetlinewidth{0.602250pt}%
\definecolor{currentstroke}{rgb}{0.000000,0.000000,0.000000}%
\pgfsetstrokecolor{currentstroke}%
\pgfsetdash{}{0pt}%
\pgfsys@defobject{currentmarker}{\pgfqpoint{-0.027778in}{0.000000in}}{\pgfqpoint{-0.000000in}{0.000000in}}{%
\pgfpathmoveto{\pgfqpoint{-0.000000in}{0.000000in}}%
\pgfpathlineto{\pgfqpoint{-0.027778in}{0.000000in}}%
\pgfusepath{stroke,fill}%
}%
\begin{pgfscope}%
\pgfsys@transformshift{0.567708in}{0.902053in}%
\pgfsys@useobject{currentmarker}{}%
\end{pgfscope}%
\end{pgfscope}%
\begin{pgfscope}%
\pgfsetbuttcap%
\pgfsetroundjoin%
\definecolor{currentfill}{rgb}{0.000000,0.000000,0.000000}%
\pgfsetfillcolor{currentfill}%
\pgfsetlinewidth{0.602250pt}%
\definecolor{currentstroke}{rgb}{0.000000,0.000000,0.000000}%
\pgfsetstrokecolor{currentstroke}%
\pgfsetdash{}{0pt}%
\pgfsys@defobject{currentmarker}{\pgfqpoint{-0.027778in}{0.000000in}}{\pgfqpoint{-0.000000in}{0.000000in}}{%
\pgfpathmoveto{\pgfqpoint{-0.000000in}{0.000000in}}%
\pgfpathlineto{\pgfqpoint{-0.027778in}{0.000000in}}%
\pgfusepath{stroke,fill}%
}%
\begin{pgfscope}%
\pgfsys@transformshift{0.567708in}{0.958896in}%
\pgfsys@useobject{currentmarker}{}%
\end{pgfscope}%
\end{pgfscope}%
\begin{pgfscope}%
\pgfsetbuttcap%
\pgfsetroundjoin%
\definecolor{currentfill}{rgb}{0.000000,0.000000,0.000000}%
\pgfsetfillcolor{currentfill}%
\pgfsetlinewidth{0.602250pt}%
\definecolor{currentstroke}{rgb}{0.000000,0.000000,0.000000}%
\pgfsetstrokecolor{currentstroke}%
\pgfsetdash{}{0pt}%
\pgfsys@defobject{currentmarker}{\pgfqpoint{-0.027778in}{0.000000in}}{\pgfqpoint{-0.000000in}{0.000000in}}{%
\pgfpathmoveto{\pgfqpoint{-0.000000in}{0.000000in}}%
\pgfpathlineto{\pgfqpoint{-0.027778in}{0.000000in}}%
\pgfusepath{stroke,fill}%
}%
\begin{pgfscope}%
\pgfsys@transformshift{0.567708in}{1.006956in}%
\pgfsys@useobject{currentmarker}{}%
\end{pgfscope}%
\end{pgfscope}%
\begin{pgfscope}%
\pgfsetbuttcap%
\pgfsetroundjoin%
\definecolor{currentfill}{rgb}{0.000000,0.000000,0.000000}%
\pgfsetfillcolor{currentfill}%
\pgfsetlinewidth{0.602250pt}%
\definecolor{currentstroke}{rgb}{0.000000,0.000000,0.000000}%
\pgfsetstrokecolor{currentstroke}%
\pgfsetdash{}{0pt}%
\pgfsys@defobject{currentmarker}{\pgfqpoint{-0.027778in}{0.000000in}}{\pgfqpoint{-0.000000in}{0.000000in}}{%
\pgfpathmoveto{\pgfqpoint{-0.000000in}{0.000000in}}%
\pgfpathlineto{\pgfqpoint{-0.027778in}{0.000000in}}%
\pgfusepath{stroke,fill}%
}%
\begin{pgfscope}%
\pgfsys@transformshift{0.567708in}{1.048587in}%
\pgfsys@useobject{currentmarker}{}%
\end{pgfscope}%
\end{pgfscope}%
\begin{pgfscope}%
\pgfsetbuttcap%
\pgfsetroundjoin%
\definecolor{currentfill}{rgb}{0.000000,0.000000,0.000000}%
\pgfsetfillcolor{currentfill}%
\pgfsetlinewidth{0.602250pt}%
\definecolor{currentstroke}{rgb}{0.000000,0.000000,0.000000}%
\pgfsetstrokecolor{currentstroke}%
\pgfsetdash{}{0pt}%
\pgfsys@defobject{currentmarker}{\pgfqpoint{-0.027778in}{0.000000in}}{\pgfqpoint{-0.000000in}{0.000000in}}{%
\pgfpathmoveto{\pgfqpoint{-0.000000in}{0.000000in}}%
\pgfpathlineto{\pgfqpoint{-0.027778in}{0.000000in}}%
\pgfusepath{stroke,fill}%
}%
\begin{pgfscope}%
\pgfsys@transformshift{0.567708in}{1.085308in}%
\pgfsys@useobject{currentmarker}{}%
\end{pgfscope}%
\end{pgfscope}%
\begin{pgfscope}%
\pgfsetbuttcap%
\pgfsetroundjoin%
\definecolor{currentfill}{rgb}{0.000000,0.000000,0.000000}%
\pgfsetfillcolor{currentfill}%
\pgfsetlinewidth{0.602250pt}%
\definecolor{currentstroke}{rgb}{0.000000,0.000000,0.000000}%
\pgfsetstrokecolor{currentstroke}%
\pgfsetdash{}{0pt}%
\pgfsys@defobject{currentmarker}{\pgfqpoint{-0.027778in}{0.000000in}}{\pgfqpoint{-0.000000in}{0.000000in}}{%
\pgfpathmoveto{\pgfqpoint{-0.000000in}{0.000000in}}%
\pgfpathlineto{\pgfqpoint{-0.027778in}{0.000000in}}%
\pgfusepath{stroke,fill}%
}%
\begin{pgfscope}%
\pgfsys@transformshift{0.567708in}{1.334259in}%
\pgfsys@useobject{currentmarker}{}%
\end{pgfscope}%
\end{pgfscope}%
\begin{pgfscope}%
\pgfsetbuttcap%
\pgfsetroundjoin%
\definecolor{currentfill}{rgb}{0.000000,0.000000,0.000000}%
\pgfsetfillcolor{currentfill}%
\pgfsetlinewidth{0.602250pt}%
\definecolor{currentstroke}{rgb}{0.000000,0.000000,0.000000}%
\pgfsetstrokecolor{currentstroke}%
\pgfsetdash{}{0pt}%
\pgfsys@defobject{currentmarker}{\pgfqpoint{-0.027778in}{0.000000in}}{\pgfqpoint{-0.000000in}{0.000000in}}{%
\pgfpathmoveto{\pgfqpoint{-0.000000in}{0.000000in}}%
\pgfpathlineto{\pgfqpoint{-0.027778in}{0.000000in}}%
\pgfusepath{stroke,fill}%
}%
\begin{pgfscope}%
\pgfsys@transformshift{0.567708in}{1.460671in}%
\pgfsys@useobject{currentmarker}{}%
\end{pgfscope}%
\end{pgfscope}%
\begin{pgfscope}%
\pgfsetbuttcap%
\pgfsetroundjoin%
\definecolor{currentfill}{rgb}{0.000000,0.000000,0.000000}%
\pgfsetfillcolor{currentfill}%
\pgfsetlinewidth{0.602250pt}%
\definecolor{currentstroke}{rgb}{0.000000,0.000000,0.000000}%
\pgfsetstrokecolor{currentstroke}%
\pgfsetdash{}{0pt}%
\pgfsys@defobject{currentmarker}{\pgfqpoint{-0.027778in}{0.000000in}}{\pgfqpoint{-0.000000in}{0.000000in}}{%
\pgfpathmoveto{\pgfqpoint{-0.000000in}{0.000000in}}%
\pgfpathlineto{\pgfqpoint{-0.027778in}{0.000000in}}%
\pgfusepath{stroke,fill}%
}%
\begin{pgfscope}%
\pgfsys@transformshift{0.567708in}{1.550362in}%
\pgfsys@useobject{currentmarker}{}%
\end{pgfscope}%
\end{pgfscope}%
\begin{pgfscope}%
\pgfsetbuttcap%
\pgfsetroundjoin%
\definecolor{currentfill}{rgb}{0.000000,0.000000,0.000000}%
\pgfsetfillcolor{currentfill}%
\pgfsetlinewidth{0.602250pt}%
\definecolor{currentstroke}{rgb}{0.000000,0.000000,0.000000}%
\pgfsetstrokecolor{currentstroke}%
\pgfsetdash{}{0pt}%
\pgfsys@defobject{currentmarker}{\pgfqpoint{-0.027778in}{0.000000in}}{\pgfqpoint{-0.000000in}{0.000000in}}{%
\pgfpathmoveto{\pgfqpoint{-0.000000in}{0.000000in}}%
\pgfpathlineto{\pgfqpoint{-0.027778in}{0.000000in}}%
\pgfusepath{stroke,fill}%
}%
\begin{pgfscope}%
\pgfsys@transformshift{0.567708in}{1.619932in}%
\pgfsys@useobject{currentmarker}{}%
\end{pgfscope}%
\end{pgfscope}%
\begin{pgfscope}%
\pgfsetbuttcap%
\pgfsetroundjoin%
\definecolor{currentfill}{rgb}{0.000000,0.000000,0.000000}%
\pgfsetfillcolor{currentfill}%
\pgfsetlinewidth{0.602250pt}%
\definecolor{currentstroke}{rgb}{0.000000,0.000000,0.000000}%
\pgfsetstrokecolor{currentstroke}%
\pgfsetdash{}{0pt}%
\pgfsys@defobject{currentmarker}{\pgfqpoint{-0.027778in}{0.000000in}}{\pgfqpoint{-0.000000in}{0.000000in}}{%
\pgfpathmoveto{\pgfqpoint{-0.000000in}{0.000000in}}%
\pgfpathlineto{\pgfqpoint{-0.027778in}{0.000000in}}%
\pgfusepath{stroke,fill}%
}%
\begin{pgfscope}%
\pgfsys@transformshift{0.567708in}{1.676774in}%
\pgfsys@useobject{currentmarker}{}%
\end{pgfscope}%
\end{pgfscope}%
\begin{pgfscope}%
\pgfsetbuttcap%
\pgfsetroundjoin%
\definecolor{currentfill}{rgb}{0.000000,0.000000,0.000000}%
\pgfsetfillcolor{currentfill}%
\pgfsetlinewidth{0.602250pt}%
\definecolor{currentstroke}{rgb}{0.000000,0.000000,0.000000}%
\pgfsetstrokecolor{currentstroke}%
\pgfsetdash{}{0pt}%
\pgfsys@defobject{currentmarker}{\pgfqpoint{-0.027778in}{0.000000in}}{\pgfqpoint{-0.000000in}{0.000000in}}{%
\pgfpathmoveto{\pgfqpoint{-0.000000in}{0.000000in}}%
\pgfpathlineto{\pgfqpoint{-0.027778in}{0.000000in}}%
\pgfusepath{stroke,fill}%
}%
\begin{pgfscope}%
\pgfsys@transformshift{0.567708in}{1.724834in}%
\pgfsys@useobject{currentmarker}{}%
\end{pgfscope}%
\end{pgfscope}%
\begin{pgfscope}%
\pgfsetbuttcap%
\pgfsetroundjoin%
\definecolor{currentfill}{rgb}{0.000000,0.000000,0.000000}%
\pgfsetfillcolor{currentfill}%
\pgfsetlinewidth{0.602250pt}%
\definecolor{currentstroke}{rgb}{0.000000,0.000000,0.000000}%
\pgfsetstrokecolor{currentstroke}%
\pgfsetdash{}{0pt}%
\pgfsys@defobject{currentmarker}{\pgfqpoint{-0.027778in}{0.000000in}}{\pgfqpoint{-0.000000in}{0.000000in}}{%
\pgfpathmoveto{\pgfqpoint{-0.000000in}{0.000000in}}%
\pgfpathlineto{\pgfqpoint{-0.027778in}{0.000000in}}%
\pgfusepath{stroke,fill}%
}%
\begin{pgfscope}%
\pgfsys@transformshift{0.567708in}{1.766465in}%
\pgfsys@useobject{currentmarker}{}%
\end{pgfscope}%
\end{pgfscope}%
\begin{pgfscope}%
\pgfsetbuttcap%
\pgfsetroundjoin%
\definecolor{currentfill}{rgb}{0.000000,0.000000,0.000000}%
\pgfsetfillcolor{currentfill}%
\pgfsetlinewidth{0.602250pt}%
\definecolor{currentstroke}{rgb}{0.000000,0.000000,0.000000}%
\pgfsetstrokecolor{currentstroke}%
\pgfsetdash{}{0pt}%
\pgfsys@defobject{currentmarker}{\pgfqpoint{-0.027778in}{0.000000in}}{\pgfqpoint{-0.000000in}{0.000000in}}{%
\pgfpathmoveto{\pgfqpoint{-0.000000in}{0.000000in}}%
\pgfpathlineto{\pgfqpoint{-0.027778in}{0.000000in}}%
\pgfusepath{stroke,fill}%
}%
\begin{pgfscope}%
\pgfsys@transformshift{0.567708in}{1.803186in}%
\pgfsys@useobject{currentmarker}{}%
\end{pgfscope}%
\end{pgfscope}%
\begin{pgfscope}%
\pgfsetbuttcap%
\pgfsetroundjoin%
\definecolor{currentfill}{rgb}{0.000000,0.000000,0.000000}%
\pgfsetfillcolor{currentfill}%
\pgfsetlinewidth{0.602250pt}%
\definecolor{currentstroke}{rgb}{0.000000,0.000000,0.000000}%
\pgfsetstrokecolor{currentstroke}%
\pgfsetdash{}{0pt}%
\pgfsys@defobject{currentmarker}{\pgfqpoint{-0.027778in}{0.000000in}}{\pgfqpoint{-0.000000in}{0.000000in}}{%
\pgfpathmoveto{\pgfqpoint{-0.000000in}{0.000000in}}%
\pgfpathlineto{\pgfqpoint{-0.027778in}{0.000000in}}%
\pgfusepath{stroke,fill}%
}%
\begin{pgfscope}%
\pgfsys@transformshift{0.567708in}{2.052138in}%
\pgfsys@useobject{currentmarker}{}%
\end{pgfscope}%
\end{pgfscope}%
\begin{pgfscope}%
\pgfsetbuttcap%
\pgfsetroundjoin%
\definecolor{currentfill}{rgb}{0.000000,0.000000,0.000000}%
\pgfsetfillcolor{currentfill}%
\pgfsetlinewidth{0.602250pt}%
\definecolor{currentstroke}{rgb}{0.000000,0.000000,0.000000}%
\pgfsetstrokecolor{currentstroke}%
\pgfsetdash{}{0pt}%
\pgfsys@defobject{currentmarker}{\pgfqpoint{-0.027778in}{0.000000in}}{\pgfqpoint{-0.000000in}{0.000000in}}{%
\pgfpathmoveto{\pgfqpoint{-0.000000in}{0.000000in}}%
\pgfpathlineto{\pgfqpoint{-0.027778in}{0.000000in}}%
\pgfusepath{stroke,fill}%
}%
\begin{pgfscope}%
\pgfsys@transformshift{0.567708in}{2.178550in}%
\pgfsys@useobject{currentmarker}{}%
\end{pgfscope}%
\end{pgfscope}%
\begin{pgfscope}%
\pgfsetbuttcap%
\pgfsetroundjoin%
\definecolor{currentfill}{rgb}{0.000000,0.000000,0.000000}%
\pgfsetfillcolor{currentfill}%
\pgfsetlinewidth{0.602250pt}%
\definecolor{currentstroke}{rgb}{0.000000,0.000000,0.000000}%
\pgfsetstrokecolor{currentstroke}%
\pgfsetdash{}{0pt}%
\pgfsys@defobject{currentmarker}{\pgfqpoint{-0.027778in}{0.000000in}}{\pgfqpoint{-0.000000in}{0.000000in}}{%
\pgfpathmoveto{\pgfqpoint{-0.000000in}{0.000000in}}%
\pgfpathlineto{\pgfqpoint{-0.027778in}{0.000000in}}%
\pgfusepath{stroke,fill}%
}%
\begin{pgfscope}%
\pgfsys@transformshift{0.567708in}{2.268241in}%
\pgfsys@useobject{currentmarker}{}%
\end{pgfscope}%
\end{pgfscope}%
\begin{pgfscope}%
\pgfsetbuttcap%
\pgfsetroundjoin%
\definecolor{currentfill}{rgb}{0.000000,0.000000,0.000000}%
\pgfsetfillcolor{currentfill}%
\pgfsetlinewidth{0.602250pt}%
\definecolor{currentstroke}{rgb}{0.000000,0.000000,0.000000}%
\pgfsetstrokecolor{currentstroke}%
\pgfsetdash{}{0pt}%
\pgfsys@defobject{currentmarker}{\pgfqpoint{-0.027778in}{0.000000in}}{\pgfqpoint{-0.000000in}{0.000000in}}{%
\pgfpathmoveto{\pgfqpoint{-0.000000in}{0.000000in}}%
\pgfpathlineto{\pgfqpoint{-0.027778in}{0.000000in}}%
\pgfusepath{stroke,fill}%
}%
\begin{pgfscope}%
\pgfsys@transformshift{0.567708in}{2.337810in}%
\pgfsys@useobject{currentmarker}{}%
\end{pgfscope}%
\end{pgfscope}%
\begin{pgfscope}%
\pgfsetbuttcap%
\pgfsetroundjoin%
\definecolor{currentfill}{rgb}{0.000000,0.000000,0.000000}%
\pgfsetfillcolor{currentfill}%
\pgfsetlinewidth{0.602250pt}%
\definecolor{currentstroke}{rgb}{0.000000,0.000000,0.000000}%
\pgfsetstrokecolor{currentstroke}%
\pgfsetdash{}{0pt}%
\pgfsys@defobject{currentmarker}{\pgfqpoint{-0.027778in}{0.000000in}}{\pgfqpoint{-0.000000in}{0.000000in}}{%
\pgfpathmoveto{\pgfqpoint{-0.000000in}{0.000000in}}%
\pgfpathlineto{\pgfqpoint{-0.027778in}{0.000000in}}%
\pgfusepath{stroke,fill}%
}%
\begin{pgfscope}%
\pgfsys@transformshift{0.567708in}{2.394653in}%
\pgfsys@useobject{currentmarker}{}%
\end{pgfscope}%
\end{pgfscope}%
\begin{pgfscope}%
\pgfsetbuttcap%
\pgfsetroundjoin%
\definecolor{currentfill}{rgb}{0.000000,0.000000,0.000000}%
\pgfsetfillcolor{currentfill}%
\pgfsetlinewidth{0.602250pt}%
\definecolor{currentstroke}{rgb}{0.000000,0.000000,0.000000}%
\pgfsetstrokecolor{currentstroke}%
\pgfsetdash{}{0pt}%
\pgfsys@defobject{currentmarker}{\pgfqpoint{-0.027778in}{0.000000in}}{\pgfqpoint{-0.000000in}{0.000000in}}{%
\pgfpathmoveto{\pgfqpoint{-0.000000in}{0.000000in}}%
\pgfpathlineto{\pgfqpoint{-0.027778in}{0.000000in}}%
\pgfusepath{stroke,fill}%
}%
\begin{pgfscope}%
\pgfsys@transformshift{0.567708in}{2.442713in}%
\pgfsys@useobject{currentmarker}{}%
\end{pgfscope}%
\end{pgfscope}%
\begin{pgfscope}%
\pgfsetbuttcap%
\pgfsetroundjoin%
\definecolor{currentfill}{rgb}{0.000000,0.000000,0.000000}%
\pgfsetfillcolor{currentfill}%
\pgfsetlinewidth{0.602250pt}%
\definecolor{currentstroke}{rgb}{0.000000,0.000000,0.000000}%
\pgfsetstrokecolor{currentstroke}%
\pgfsetdash{}{0pt}%
\pgfsys@defobject{currentmarker}{\pgfqpoint{-0.027778in}{0.000000in}}{\pgfqpoint{-0.000000in}{0.000000in}}{%
\pgfpathmoveto{\pgfqpoint{-0.000000in}{0.000000in}}%
\pgfpathlineto{\pgfqpoint{-0.027778in}{0.000000in}}%
\pgfusepath{stroke,fill}%
}%
\begin{pgfscope}%
\pgfsys@transformshift{0.567708in}{2.484344in}%
\pgfsys@useobject{currentmarker}{}%
\end{pgfscope}%
\end{pgfscope}%
\begin{pgfscope}%
\pgfsetbuttcap%
\pgfsetroundjoin%
\definecolor{currentfill}{rgb}{0.000000,0.000000,0.000000}%
\pgfsetfillcolor{currentfill}%
\pgfsetlinewidth{0.602250pt}%
\definecolor{currentstroke}{rgb}{0.000000,0.000000,0.000000}%
\pgfsetstrokecolor{currentstroke}%
\pgfsetdash{}{0pt}%
\pgfsys@defobject{currentmarker}{\pgfqpoint{-0.027778in}{0.000000in}}{\pgfqpoint{-0.000000in}{0.000000in}}{%
\pgfpathmoveto{\pgfqpoint{-0.000000in}{0.000000in}}%
\pgfpathlineto{\pgfqpoint{-0.027778in}{0.000000in}}%
\pgfusepath{stroke,fill}%
}%
\begin{pgfscope}%
\pgfsys@transformshift{0.567708in}{2.521065in}%
\pgfsys@useobject{currentmarker}{}%
\end{pgfscope}%
\end{pgfscope}%
\begin{pgfscope}%
\pgfsetbuttcap%
\pgfsetroundjoin%
\definecolor{currentfill}{rgb}{0.000000,0.000000,0.000000}%
\pgfsetfillcolor{currentfill}%
\pgfsetlinewidth{0.602250pt}%
\definecolor{currentstroke}{rgb}{0.000000,0.000000,0.000000}%
\pgfsetstrokecolor{currentstroke}%
\pgfsetdash{}{0pt}%
\pgfsys@defobject{currentmarker}{\pgfqpoint{-0.027778in}{0.000000in}}{\pgfqpoint{-0.000000in}{0.000000in}}{%
\pgfpathmoveto{\pgfqpoint{-0.000000in}{0.000000in}}%
\pgfpathlineto{\pgfqpoint{-0.027778in}{0.000000in}}%
\pgfusepath{stroke,fill}%
}%
\begin{pgfscope}%
\pgfsys@transformshift{0.567708in}{2.770016in}%
\pgfsys@useobject{currentmarker}{}%
\end{pgfscope}%
\end{pgfscope}%
\begin{pgfscope}%
\pgfsetbuttcap%
\pgfsetroundjoin%
\definecolor{currentfill}{rgb}{0.000000,0.000000,0.000000}%
\pgfsetfillcolor{currentfill}%
\pgfsetlinewidth{0.602250pt}%
\definecolor{currentstroke}{rgb}{0.000000,0.000000,0.000000}%
\pgfsetstrokecolor{currentstroke}%
\pgfsetdash{}{0pt}%
\pgfsys@defobject{currentmarker}{\pgfqpoint{-0.027778in}{0.000000in}}{\pgfqpoint{-0.000000in}{0.000000in}}{%
\pgfpathmoveto{\pgfqpoint{-0.000000in}{0.000000in}}%
\pgfpathlineto{\pgfqpoint{-0.027778in}{0.000000in}}%
\pgfusepath{stroke,fill}%
}%
\begin{pgfscope}%
\pgfsys@transformshift{0.567708in}{2.896428in}%
\pgfsys@useobject{currentmarker}{}%
\end{pgfscope}%
\end{pgfscope}%
\begin{pgfscope}%
\pgfsetbuttcap%
\pgfsetroundjoin%
\definecolor{currentfill}{rgb}{0.000000,0.000000,0.000000}%
\pgfsetfillcolor{currentfill}%
\pgfsetlinewidth{0.602250pt}%
\definecolor{currentstroke}{rgb}{0.000000,0.000000,0.000000}%
\pgfsetstrokecolor{currentstroke}%
\pgfsetdash{}{0pt}%
\pgfsys@defobject{currentmarker}{\pgfqpoint{-0.027778in}{0.000000in}}{\pgfqpoint{-0.000000in}{0.000000in}}{%
\pgfpathmoveto{\pgfqpoint{-0.000000in}{0.000000in}}%
\pgfpathlineto{\pgfqpoint{-0.027778in}{0.000000in}}%
\pgfusepath{stroke,fill}%
}%
\begin{pgfscope}%
\pgfsys@transformshift{0.567708in}{2.986119in}%
\pgfsys@useobject{currentmarker}{}%
\end{pgfscope}%
\end{pgfscope}%
\begin{pgfscope}%
\pgfsetbuttcap%
\pgfsetroundjoin%
\definecolor{currentfill}{rgb}{0.000000,0.000000,0.000000}%
\pgfsetfillcolor{currentfill}%
\pgfsetlinewidth{0.602250pt}%
\definecolor{currentstroke}{rgb}{0.000000,0.000000,0.000000}%
\pgfsetstrokecolor{currentstroke}%
\pgfsetdash{}{0pt}%
\pgfsys@defobject{currentmarker}{\pgfqpoint{-0.027778in}{0.000000in}}{\pgfqpoint{-0.000000in}{0.000000in}}{%
\pgfpathmoveto{\pgfqpoint{-0.000000in}{0.000000in}}%
\pgfpathlineto{\pgfqpoint{-0.027778in}{0.000000in}}%
\pgfusepath{stroke,fill}%
}%
\begin{pgfscope}%
\pgfsys@transformshift{0.567708in}{3.055689in}%
\pgfsys@useobject{currentmarker}{}%
\end{pgfscope}%
\end{pgfscope}%
\begin{pgfscope}%
\pgfsetbuttcap%
\pgfsetroundjoin%
\definecolor{currentfill}{rgb}{0.000000,0.000000,0.000000}%
\pgfsetfillcolor{currentfill}%
\pgfsetlinewidth{0.602250pt}%
\definecolor{currentstroke}{rgb}{0.000000,0.000000,0.000000}%
\pgfsetstrokecolor{currentstroke}%
\pgfsetdash{}{0pt}%
\pgfsys@defobject{currentmarker}{\pgfqpoint{-0.027778in}{0.000000in}}{\pgfqpoint{-0.000000in}{0.000000in}}{%
\pgfpathmoveto{\pgfqpoint{-0.000000in}{0.000000in}}%
\pgfpathlineto{\pgfqpoint{-0.027778in}{0.000000in}}%
\pgfusepath{stroke,fill}%
}%
\begin{pgfscope}%
\pgfsys@transformshift{0.567708in}{3.112531in}%
\pgfsys@useobject{currentmarker}{}%
\end{pgfscope}%
\end{pgfscope}%
\begin{pgfscope}%
\pgfsetbuttcap%
\pgfsetroundjoin%
\definecolor{currentfill}{rgb}{0.000000,0.000000,0.000000}%
\pgfsetfillcolor{currentfill}%
\pgfsetlinewidth{0.602250pt}%
\definecolor{currentstroke}{rgb}{0.000000,0.000000,0.000000}%
\pgfsetstrokecolor{currentstroke}%
\pgfsetdash{}{0pt}%
\pgfsys@defobject{currentmarker}{\pgfqpoint{-0.027778in}{0.000000in}}{\pgfqpoint{-0.000000in}{0.000000in}}{%
\pgfpathmoveto{\pgfqpoint{-0.000000in}{0.000000in}}%
\pgfpathlineto{\pgfqpoint{-0.027778in}{0.000000in}}%
\pgfusepath{stroke,fill}%
}%
\begin{pgfscope}%
\pgfsys@transformshift{0.567708in}{3.160591in}%
\pgfsys@useobject{currentmarker}{}%
\end{pgfscope}%
\end{pgfscope}%
\begin{pgfscope}%
\pgfsetbuttcap%
\pgfsetroundjoin%
\definecolor{currentfill}{rgb}{0.000000,0.000000,0.000000}%
\pgfsetfillcolor{currentfill}%
\pgfsetlinewidth{0.602250pt}%
\definecolor{currentstroke}{rgb}{0.000000,0.000000,0.000000}%
\pgfsetstrokecolor{currentstroke}%
\pgfsetdash{}{0pt}%
\pgfsys@defobject{currentmarker}{\pgfqpoint{-0.027778in}{0.000000in}}{\pgfqpoint{-0.000000in}{0.000000in}}{%
\pgfpathmoveto{\pgfqpoint{-0.000000in}{0.000000in}}%
\pgfpathlineto{\pgfqpoint{-0.027778in}{0.000000in}}%
\pgfusepath{stroke,fill}%
}%
\begin{pgfscope}%
\pgfsys@transformshift{0.567708in}{3.202222in}%
\pgfsys@useobject{currentmarker}{}%
\end{pgfscope}%
\end{pgfscope}%
\begin{pgfscope}%
\definecolor{textcolor}{rgb}{0.000000,0.000000,0.000000}%
\pgfsetstrokecolor{textcolor}%
\pgfsetfillcolor{textcolor}%
\pgftext[x=0.414931in,y=1.801250in,,bottom,rotate=90.000000]{\color{textcolor}\rmfamily\fontsize{8.000000}{9.600000}\selectfont Arithmetic throughput (FLOPs/sec)}%
\end{pgfscope}%
\begin{pgfscope}%
\pgfpathrectangle{\pgfqpoint{0.567708in}{0.400278in}}{\pgfqpoint{3.519792in}{2.801944in}}%
\pgfusepath{clip}%
\pgfsetroundcap%
\pgfsetroundjoin%
\pgfsetlinewidth{1.505625pt}%
\definecolor{currentstroke}{rgb}{0.000000,0.000000,0.000000}%
\pgfsetstrokecolor{currentstroke}%
\pgfsetdash{}{0pt}%
\pgfpathmoveto{\pgfqpoint{0.567708in}{1.098815in}}%
\pgfpathlineto{\pgfqpoint{2.423713in}{2.700447in}}%
\pgfusepath{stroke}%
\end{pgfscope}%
\begin{pgfscope}%
\pgfpathrectangle{\pgfqpoint{0.567708in}{0.400278in}}{\pgfqpoint{3.519792in}{2.801944in}}%
\pgfusepath{clip}%
\pgfsetroundcap%
\pgfsetroundjoin%
\pgfsetlinewidth{1.505625pt}%
\definecolor{currentstroke}{rgb}{0.000000,0.000000,0.000000}%
\pgfsetstrokecolor{currentstroke}%
\pgfsetdash{}{0pt}%
\pgfpathmoveto{\pgfqpoint{0.567708in}{0.723452in}}%
\pgfpathlineto{\pgfqpoint{2.858693in}{2.700447in}}%
\pgfusepath{stroke}%
\end{pgfscope}%
\begin{pgfscope}%
\pgfpathrectangle{\pgfqpoint{0.567708in}{0.400278in}}{\pgfqpoint{3.519792in}{2.801944in}}%
\pgfusepath{clip}%
\pgfsetroundcap%
\pgfsetroundjoin%
\pgfsetlinewidth{1.505625pt}%
\definecolor{currentstroke}{rgb}{0.000000,0.000000,0.000000}%
\pgfsetstrokecolor{currentstroke}%
\pgfsetdash{}{0pt}%
\pgfpathmoveto{\pgfqpoint{1.672439in}{2.052138in}}%
\pgfpathlineto{\pgfqpoint{4.087500in}{2.052138in}}%
\pgfusepath{stroke}%
\end{pgfscope}%
\begin{pgfscope}%
\pgfpathrectangle{\pgfqpoint{0.567708in}{0.400278in}}{\pgfqpoint{3.519792in}{2.801944in}}%
\pgfusepath{clip}%
\pgfsetroundcap%
\pgfsetroundjoin%
\pgfsetlinewidth{1.505625pt}%
\definecolor{currentstroke}{rgb}{0.000000,0.000000,0.000000}%
\pgfsetstrokecolor{currentstroke}%
\pgfsetdash{}{0pt}%
\pgfpathmoveto{\pgfqpoint{2.423713in}{2.700447in}}%
\pgfpathlineto{\pgfqpoint{4.087500in}{2.700447in}}%
\pgfusepath{stroke}%
\end{pgfscope}%
\begin{pgfscope}%
\pgfsetrectcap%
\pgfsetmiterjoin%
\pgfsetlinewidth{0.803000pt}%
\definecolor{currentstroke}{rgb}{0.000000,0.000000,0.000000}%
\pgfsetstrokecolor{currentstroke}%
\pgfsetdash{}{0pt}%
\pgfpathmoveto{\pgfqpoint{0.567708in}{0.400278in}}%
\pgfpathlineto{\pgfqpoint{0.567708in}{3.202222in}}%
\pgfusepath{stroke}%
\end{pgfscope}%
\begin{pgfscope}%
\pgfsetrectcap%
\pgfsetmiterjoin%
\pgfsetlinewidth{0.803000pt}%
\definecolor{currentstroke}{rgb}{0.000000,0.000000,0.000000}%
\pgfsetstrokecolor{currentstroke}%
\pgfsetdash{}{0pt}%
\pgfpathmoveto{\pgfqpoint{4.087500in}{0.400278in}}%
\pgfpathlineto{\pgfqpoint{4.087500in}{3.202222in}}%
\pgfusepath{stroke}%
\end{pgfscope}%
\begin{pgfscope}%
\pgfsetrectcap%
\pgfsetmiterjoin%
\pgfsetlinewidth{0.803000pt}%
\definecolor{currentstroke}{rgb}{0.000000,0.000000,0.000000}%
\pgfsetstrokecolor{currentstroke}%
\pgfsetdash{}{0pt}%
\pgfpathmoveto{\pgfqpoint{0.567708in}{0.400278in}}%
\pgfpathlineto{\pgfqpoint{4.087500in}{0.400278in}}%
\pgfusepath{stroke}%
\end{pgfscope}%
\begin{pgfscope}%
\pgfsetrectcap%
\pgfsetmiterjoin%
\pgfsetlinewidth{0.803000pt}%
\definecolor{currentstroke}{rgb}{0.000000,0.000000,0.000000}%
\pgfsetstrokecolor{currentstroke}%
\pgfsetdash{}{0pt}%
\pgfpathmoveto{\pgfqpoint{0.567708in}{3.202222in}}%
\pgfpathlineto{\pgfqpoint{4.087500in}{3.202222in}}%
\pgfusepath{stroke}%
\end{pgfscope}%
\begin{pgfscope}%
\definecolor{textcolor}{rgb}{0.000000,0.000000,0.000000}%
\pgfsetstrokecolor{textcolor}%
\pgfsetfillcolor{textcolor}%
\pgftext[x=0.734375in,y=1.307149in,left,bottom,rotate=40.975915]{\color{textcolor}\rmfamily\fontsize{8.000000}{9.600000}\selectfont Peak cache bandwidth}%
\end{pgfscope}%
\begin{pgfscope}%
\definecolor{textcolor}{rgb}{0.000000,0.000000,0.000000}%
\pgfsetstrokecolor{textcolor}%
\pgfsetfillcolor{textcolor}%
\pgftext[x=0.734375in,y=0.890119in,left,bottom,rotate=40.975915]{\color{textcolor}\rmfamily\fontsize{8.000000}{9.600000}\selectfont Peak main memory bandwidth}%
\end{pgfscope}%
\begin{pgfscope}%
\definecolor{textcolor}{rgb}{0.000000,0.000000,0.000000}%
\pgfsetstrokecolor{textcolor}%
\pgfsetfillcolor{textcolor}%
\pgftext[x=2.837500in,y=2.079916in,left,bottom]{\color{textcolor}\rmfamily\fontsize{8.000000}{9.600000}\selectfont Peak scalar throughput}%
\end{pgfscope}%
\begin{pgfscope}%
\definecolor{textcolor}{rgb}{0.000000,0.000000,0.000000}%
\pgfsetstrokecolor{textcolor}%
\pgfsetfillcolor{textcolor}%
\pgftext[x=2.837500in,y=2.728224in,left,bottom]{\color{textcolor}\rmfamily\fontsize{8.000000}{9.600000}\selectfont Peak vector throughput}%
\end{pgfscope}%
\begin{pgfscope}%
\definecolor{textcolor}{rgb}{0.000000,0.000000,0.000000}%
\pgfsetstrokecolor{textcolor}%
\pgfsetfillcolor{textcolor}%
\pgftext[x=1.179624in,y=0.804831in,,]{\color{textcolor}\rmfamily\fontsize{8.000000}{9.600000}\selectfont A}%
\end{pgfscope}%
\begin{pgfscope}%
\definecolor{textcolor}{rgb}{0.000000,0.000000,0.000000}%
\pgfsetstrokecolor{textcolor}%
\pgfsetfillcolor{textcolor}%
\pgftext[x=3.183023in,y=1.090504in,,]{\color{textcolor}\rmfamily\fontsize{8.000000}{9.600000}\selectfont B}%
\end{pgfscope}%
\begin{pgfscope}%
\definecolor{textcolor}{rgb}{0.000000,0.000000,0.000000}%
\pgfsetstrokecolor{textcolor}%
\pgfsetfillcolor{textcolor}%
\pgftext[x=3.764491in,y=2.423843in,,]{\color{textcolor}\rmfamily\fontsize{8.000000}{9.600000}\selectfont C}%
\end{pgfscope}%
\end{pgfpicture}%
\makeatother%
\endgroup%

\end{figure}

\subsection{Optimisations for mesh computations}

  \section{Implementation}
\label{sec:impl}

As discussed in Section~\ref{sec:stencillang}, existing stencil languages may be classified according to whether or not they are aware of the mesh topology.
A library that is not `mesh-aware', for example \pyop2, can be more challenging to program in because responsibility for reasoning about the topology, including orientations, is passed to the user who has to construct the appropriate indirection maps to represent their mesh.
By contrast, a `mesh-aware' library does not have these problems but it has to use its own custom mesh implementation.
This increases the burden on the maintainers of the software and the mesh implementations will, without substantial development effort, suffer from both lack of features (e.g. I/O, parallel decomposition, adaptive refinement) and poor interoperability with other packages.

In this work we attempt to bridge this gap by writing a new stencil language, \pyop3, that combines the advantages provided by `mesh-aware' frameworks with a mature, external mesh implementation (DMPlex).
\pyop3 is, somewhat obviously, heavily inspired by and based upon \pyop2, and hence much of its design represents either an incremental improvement on \pyop2, or is in fact directly lifted from it.


In \pyop3, users declare the iterations to be performed, the local operations to apply, and the stencil patterns for each data structure in a manner that is close to the mathematics/pseudocode.
As an example, the syntax for a typical \gls{fem} residual assembly, where one loops over cells and computes using data in the cell's closure, would look something like:

\begin{minted}[xleftmargin=4em]{python}
do_loop(
  c := mesh.cells.index,
  kernel(dat0[closure(c)], dat1[closure(c)])
)
\end{minted}

Unlike \pyop2, \pyop3 permits arbitrary composition of DMPlex restriction operations.
One can, for example, easily describe stencils over interior facets where the stencil is composed of \textit{the closure of the cells incident on the facet}, or, in DMPlex terminology, $\closure(\support(p))$:

\begin{minted}[xleftmargin=4em]{python}
do_loop(
  f := mesh.interior_facets.index,
  kernel(
    dat0[closure(support(f))],
    dat1[closure(support(f))]
  )
)
\end{minted}

\subsection{Data types}

Like \pyop2, \pyop3 has three main data types: \py{Globals}, \py{Dats} and \py{Mats}.
These may be understand as data types aware of zero, one or two meshes respectively.

A \py{Global} represents some, possibly vector-valued, quantity that is shared across all processes.
It has no awareness of the mesh whatsoever.

By contrast, a \py{Dat} is a vector associated with a given mesh, storing some set of values at each mesh point.
It is uniquely associated with a single mesh and, since the mesh is distributed between processes, only a subset of the overall data structure is present on any single process.

Lastly, a \py{Mat} can be thought of as representing the `overlap' between two (possibly the same) meshes.
Abstractly, they are used to compute quantities that depend on the interaction between different mesh points.
This can include the test and trial functions from \gls{fem}, or an interpolation matrix to interpolate values between different meshes.
Since solving \glspl{pde} typically makes use of functions with \textit{local support} (i.e. functions whose value are zero except in some small region of the mesh), these matrices are typically very \textit{sparse} as non-zeros are only present in the regions where these supports overlap.

For the \gls{fem}, where the cell integrals are defined using the closure operation, the overlap between test and trial functions is assumed to be the combination of the closures of each cell.

\subsection{Just-in-time compilation}

The execution pipeline is shown in Figure~\ref{fig:codegenproc}.
First, and this is the main work done by \pyop3, an input \textit{loop expression} is converted to a loopy intermediate representation.
Then, this intermediate representation is lowered to a low-level language such as C which is then just-in-time compiled to a shared library that can be executed using data provided by the original loop expression (e.g. here that means \py{mesh}, \py{dat0} and \py{dat1}).

Note that in this example, by using \py{do_loop}, we declare the loop expression and then \textit{immediately} execute it.
It is frequently desirable, for reasons of efficiency, to have a \textit{persistent} loop expression which one can create by running \py{expr = loop(...)}.
This is discussed in more detail in Section~\ref{sec:impl_overhead}.

\begin{figure}
  \centering
  \begin{tikzpicture}
    \tikzstyle{node} = [
      rectangle,rounded corners,minimum width=1.5cm,minimum height=1cm,text centered,
      draw=black,fill=red!80,align=center,anchor=west,font=\footnotesize
    ];
    \tikzstyle{input} = [trapezium, trapezium left angle=70, trapezium right angle=110, minimum width=1.5cm, minimum height=1cm, text centered, draw=black, fill=blue!30,align=center,anchor=west,font=\footnotesize];

    \node (loop) [input] at (0,0) {Input loop\\expression};
    \node (loopy) [node,at={(loop.east)},xshift=.7cm] {Generate\\loopy kernel};
    \node (c) [node,at={(loopy.east)},xshift=.7cm] {Generate\\source code};
    \node (exe) [node,at={(c.east)},xshift=.7cm] {Compile\\executable};
    \node (do) [node,at={(exe.east)},xshift=.7cm] {Execute};

    \draw [-{stealth}] (loop) -- (loopy);
    \draw [-{stealth}] (loopy) -- (c);
    \draw [-{stealth}] (c) -- (exe);
    \draw [-{stealth}] (exe) -- (do);
    \draw [-{stealth},densely dashed] (loop) to [bend left=35]
      node[midway,above,align=center,font=\footnotesize] {Pass in data from\\loop expression} (do);
  \end{tikzpicture}
  \caption{}
  \label{fig:codegenproc}
\end{figure}

In the example above, \mintinline{python}{kernel} is an externally provided loopy kernel augmented with \textit{access descriptors} (\py{READ}, \py{WRITE}, \py{RW}, \py{INC}, \py{MIN} or \py{MAX}) that allows \pyop3 to emit the correct packing/unpacking code.
In this case, if we assume the access descriptors for \py{kernel} are \py{READ} and \py{INC}, then we would generate C code resembling that shown in Listing~\ref{lst:basicloop}.

\begin{listing}
  \begin{minted}{c}
void do_loop(int ncells, double *dat0, double *dat1, int *map0) {
  double t0[CLOSURE_SIZE], t1[CLOSURE_SIZE];

  for (int c=0; c<ncells; ++c) {
    // Pack temporaries
    for (int p=0; p<CLOSURE_SIZE; ++p) {
      t0[p] = dat0[map0[c*CLOSURE_SIZE+p]];
      t1[p] = 0.0;
    }
    // Do the local computation
    kernel(t0, t1);
    // Now scatter the results
    for (int p=0; p<CLOSURE_SIZE; ++p) {
      dat1[map0[c*CLOSURE_SIZE+p]] += t1[p];
    }
  }
}
  \end{minted}
  \caption{
    Simplified version of code that would be generated by \pyop3 where \mintinline{c}{kernel} has access descriptors \py{READ} and \py{INC}.
    \mintinline{c}{CLOSURE_SIZE} is an integer constant and would be known at compile-time.
  }
  \label{lst:basicloop}
\end{listing}

\subsection{A new abstraction for mesh data layouts}

The additional flexibility \pyop3 has over its precursor \pyop2 is thanks to its novel abstraction for describing data layouts.
In \pyop2, data layouts are prescribed by associating data (i.e. \glspl{dof}) with \textit{sets}.
More precisely, they are described with a \py{DataSet}, which is formed by associating a \py{Set} with some \textit{local shape} (a tuple), termed its \py{dim}.
To demonstrate, consider a \pyop2 \py{Dat} originating from some vector element discretisation applied over a mesh.
At each node in the discretisation, recalling that there need not be just one node per topological entity, this \py{Dat} would store \glspl{dof} with some non-scalar \py{dim}, say, \py{(3,)}.
Such a layout is shown in Figure~\ref{fig:vdat_pyop2}.

\begin{figure}
  \centering
  \begin{subfigure}{0.48\textwidth}
    \centering
    \begin{tikzpicture}[y=-1cm,scale=.75]
      \begin{scope}[yshift=0cm]
        \fill[lightgray] (0,0) rectangle(6,1);
        \filldraw[draw=black, fill=white] (0.5,0) rectangle ++ (1,1);
        \filldraw[draw=black, fill=white] (1.5,0) rectangle ++ (1,1);
        \filldraw[draw=black, fill=white] (2.5,0) rectangle ++ (1,1);
        \filldraw[draw=black, fill=white] (3.5,0) rectangle ++ (1,1);
        \filldraw[draw=black, fill=white] (4.5,0) rectangle ++ (1,1);
        \node[at={(1,.5)}, ptlabel] {$i_0$};
        \node[at={(2,.5)}, ptlabel] {$i_1$};
        \node[at={(3,.5)}, ptlabel] {$i_4$};
        \node[at={(4,.5)}, ptlabel] {$i_5$};
        \node[at={(5,.5)}, ptlabel] {$i_9$};
        \draw (0,0) -- (6,0);
        \draw (0,1) -- (6,1);
      \end{scope}

      \begin{scope}[xshift=1.5cm,yshift=-2cm]
        \filldraw[draw=black, fill=white] (0,0) rectangle ++ (1,1);
        \filldraw[draw=black, fill=white] (1,0) rectangle ++ (1,1);
        \filldraw[draw=black, fill=white] (2,0) rectangle ++ (1,1);
        \node[at={(0.5,.5)}, ptlabel] {$d_0$};
        \node[at={(1.5,.5)}, ptlabel] {$d_1$};
        \node[at={(2.5,.5)}, ptlabel] {$d_2$};

        \draw (1,-1) -- (0,0);
        \draw (2,-1) -- (3,0);
      \end{scope}

      \node (nodeslabel) [ptlabel] at (7,.5) {Nodes};
      \node (dofslabel) [ptlabel] at (7,2.5) {\glspl{dof}};
      % \draw [-{stealth},shorten <=0pt,shorten >=8pt] (nodeslabel) -- (6,.5);
      % \draw [-{stealth},shorten <=0pt,shorten >=8pt] (dofslabel) -- (4.5,2.5);
    \end{tikzpicture}
    \caption{
      A typical \pyop2 \py{Dat} data layout for a \py{DataSet} with \py{dim} \py{(3,)}.
      Note that the nodes are unordered, since we are on an unstructured mesh, but that the \glspl{dof} are ordered.
    }
    \label{fig:vdat_pyop2}
  \end{subfigure}
  %
  \begin{subfigure}{0.48\textwidth}
    \centering
    \begin{tikzpicture}[y=-1cm,scale=.75]
      \begin{scope}[yshift=0cm]
        \fill[lightgray] (0,0) rectangle(6,1);
        \filldraw[draw=black, fill=white] (0.5,0) rectangle ++ (1,1);
        \filldraw[draw=black, fill=white] (1.5,0) rectangle ++ (1,1);
        \filldraw[draw=black, fill=white] (2.5,0) rectangle ++ (1,1);
        \filldraw[draw=black, fill=white] (3.5,0) rectangle ++ (1,1);
        \filldraw[draw=black, fill=white] (4.5,0) rectangle ++ (1,1);
        \node[at={(1,.5)}, ptlabel] {$c_5$};
        \node[at={(2,.5)}, ptlabel] {$v_1$};
        \node[at={(3,.5)}, ptlabel] {$e_4$};
        \node[at={(4,.5)}, ptlabel] {$e_5$};
        \node[at={(5,.5)}, ptlabel] {$c_9$};
        \draw (0,0) -- (6,0);
        \draw (0,1) -- (6,1);
      \end{scope}

      \begin{scope}[xshift=2cm,yshift=-2cm]
        \filldraw[draw=black, fill=white] (0,0) rectangle ++ (1,1);
        \filldraw[draw=black, fill=white] (1,0) rectangle ++ (1,1);
        \node[at={(0.5,.5)}, ptlabel] {$i_0$};
        \node[at={(1.5,.5)}, ptlabel] {$i_1$};

        \draw (.5,-1) -- (0,0);
        \draw (1.5,-1) -- (2,0);
      \end{scope}

      \begin{scope}[xshift=1cm,yshift=-4cm]
        \filldraw[draw=black, fill=white] (0,0) rectangle ++ (1,1);
        \filldraw[draw=black, fill=white] (1,0) rectangle ++ (1,1);
        \filldraw[draw=black, fill=white] (2,0) rectangle ++ (1,1);
        \node[at={(0.5,.5)}, ptlabel] {$d_0$};
        \node[at={(1.5,.5)}, ptlabel] {$d_1$};
        \node[at={(2.5,.5)}, ptlabel] {$d_2$};

        \draw (1,-1) -- (0,0);
        \draw (2,-1) -- (3,0);
      \end{scope}

      \node (pointslabel) [ptlabel] at (7,.5) {Points};
      \node (nodeslabel) [ptlabel] at (7,2.5) {Nodes};
      \node (dofslabel) [ptlabel] at (7,4.5) {\glspl{dof}};
      % \draw [-{stealth},shorten <=0pt,shorten >=8pt] (pointslabel) -- (6,.5);
      % \draw [-{stealth},shorten <=0pt,shorten >=8pt] (nodeslabel) -- (4.5,2.5);
      % \draw [-{stealth},shorten <=0pt,shorten >=8pt] (dofslabel) -- (4.5,4.5);
    \end{tikzpicture}
    \caption{
      An equivalent data layout that is aware of the topology of the mesh.
      Note that the number of nodes per entity is not constant - here we indicate that there are 2 nodes per edge, leaving cells and vertices unspecified.
    }
    \label{fig:vdat_pyop3}
  \end{subfigure}
  \caption{}
  \label{fig:vdat_comparison}
\end{figure}

There are a number of advantages to this approach:
Firstly, code generation is very straightforward as packing/unpacking is as straightforward as iterating over the nodes with a fixed size inner loop over the \glspl{dof}.
Also, this approach maps naturally to PETSc's notion of a \textit{blocked} matrix (\clang{Mat}) or vector (\clang{Vec}), since a block describes contiguous data that can be addressed all together.

Despite these advantages, however, this simple data layout model \textit{loses topological information}.
\pyop2 \py{DataSets} store data per node, but they do not know which topological entities the nodes originated from.
To counter this shortcoming, in \pyop3 we introduce a hierarchical model for describing data layout that can record, among other things, the topological entities that a given node comes from.
This can be seen in Figure~\ref{fig:vdat_pyop3}.
Rather than just having nodes and \glspl{dof} per node, we can represent the same vector-valued \py{Dat} using a 3-level data structure containing: mesh points, nodes per point, and \glspl{dof} per node.

In \pyop3, we describe these multi-dimensional, inhomogeneous data structures with a \py{MultiAxis}.
A \py{MultiAxis} represents a single layer of the hierarchy and is simply a container for some positive number of \py{AxisPart} objects.
Each \py{AxisPart} describes one particular class of entities in a given layer of the hierarchy and stores information such as the number of entries and how they might be addressed.
The `type' property of the typed multi-index is simply used as an identifier for the appropriate \py{AxisPart}.
This is similar to the interface presented by Taichi~\cite{huTaichiLanguageHighperformance2019}. % though we can do unstructured/interleaved stuff

To construct a hierarchical data layout, each \py{AxisPart} can optionally be given a sub-axis (\py{MultiAxis}).
To demonstrate, the hierarchical layout shown in Figure~\ref{fig:vdat_pyop3} could be constructed using the code in Figure~\ref{lis:demotreecode}.
The tree itself is shown in Figure~\ref{fig:vdat_tree}.

\begin{figure}
  \centering
  \begin{subfigure}{.9\textwidth}
    \centering
    \begin{tikzpicture}
      \tikzstyle{node} = [minimum width=1.5cm,text centered,align=center];

      \node [node] (root) at (0,0) {\textit{root}};
      \node [node] (cells) at (-2,-1) {cells};
      \node [node] (edges) at (0,-1) {edges};
      \node [node] (verts) at (2,-1) {vertices};
      \node [node] (cnodes) at (-2,-2) {nodes};
      \node [node] (enodes) at (0,-2) {nodes};
      \node [node] (vnodes) at (2,-2) {nodes};
      \node [node] (cdofs) at (-2,-3) {\glspl{dof}};
      \node [node] (edofs) at (0,-3) {\glspl{dof}};
      \node [node] (vdofs) at (2,-3) {\glspl{dof}};

      \draw [-{stealth}] (root) -- (cells);
      \draw [-{stealth}] (root) -- (edges);
      \draw [-{stealth}] (root) -- (verts);
      \draw [-{stealth}] (cells) -- (cnodes);
      \draw [-{stealth}] (edges) -- (enodes);
      \draw [-{stealth}] (verts) -- (vnodes);
      \draw [-{stealth}] (cnodes) -- (cdofs);
      \draw [-{stealth}] (enodes) -- (edofs);
      \draw [-{stealth}] (vnodes) -- (vdofs);
    \end{tikzpicture}
    \caption{}
    \label{fig:vdat_tree}
  \end{subfigure}

  \begin{subfigure}{.9\textwidth}
    \begin{minted}[xleftmargin=4em,fontsize=\footnotesize]{python}
root = (
  MultiAxis()
  .add_part(AxisPart(ncells, id="cells"))
  .add_part(AxisPart(nedges, id="edges"))
  .add_part(AxisPart(nverts, id="verts"))
  .add_subaxis("cells", AxisPart(ncnodes, id="cnodes"))
  .add_subaxis("edges", AxisPart(nenodes, id="enodes"))
  .add_subaxis("verts", AxisPart(nvnodes, id="vnodes"))
  .add_subaxis("cnodes", AxisPart(ncdofs))
  .add_subaxis("enodes", AxisPart(nedofs))
  .add_subaxis("vnodes", AxisPart(nvdofs))
)
    \end{minted}
    \caption{}
    \label{lis:demotreecode}
  \end{subfigure}
  \caption{}
  \label{fig:demotree}
\end{figure}

Having constructed such a data layout, we address it via the use of a \textit{typed multi-index}.
This is an object of the form $((t_1, i_1), (t_2, i_2), \dots, (t_n, i_n))$ where $t_x$ (the `type') indicates the correct \py{AxisPart} to in the hierarchy.
For each \py{AxisPart}, and given an index into it ($i_x$), we use \textit{layout functions} to determine its location in memory and these offsets are added together to get the final address.
A layout function is a function that takes in an index and returns an offset.
In the case of axes with constant stride this simply takes the form \py{off = iX * stride}, but for non-constant strides it would resemble \py{off = offsets[iX]}.

The process of actually determining the address of a particular multi-index simply requires a \textit{pre-order tree visitor} algorithm to traverse the tree and accumulate the outputs of the layout functions at each level, before dispatching to the right child depending upon the `type' argument of the multi-index.

One major challenge presented by this new layout is that axes are no longer homogeneous.
In Figure~\ref{fig:vdat_pyop3} for instance, not all points have the same number of nodes per point.
This means that one can no longer stride over the axis by a constant value, and instead a lookup table must be used.

% TODO: This bit should probably be moved/broken up
This approach is advantageous because it is much more natural to reason about mesh operations at the level of mesh points.
DMPlex restrictions naturally map points to points instead of points to nodes, making map composition tractable.
The approach also facilitates: orienting \glspl{dof} (Section~\ref{sec:impl_orientation}), data layout optimisations (Section~\ref{sec:impl_datalayoutopt}), and extruded and other partially-structured meshes (Section~\ref{sec:future_partialstructure}).

% TODO: Mention that this lets us do mixed

\subsubsection{Maps}

When we address some data, the provided data structure is associated with a particular multi-index.
When we directly address data structures, for example by doing the following:

\begin{minted}[xleftmargin=4em]{python}
loop(c := mesh.cells.index, kernel(dat0[c]))
\end{minted}

Then the multi-index getting used, \py{c}, is simply $[(C, i)]$.
This is a multi-index with only a single entry which targets all entries in the selected \py{AxisPart}, in this case all cells in the mesh.
We remark that only the outermost axes need be indexed - the inner axes (here nodes-per-cell and \glspl{dof}-per-node) are automatically included as full slices.

Since computing stencils requires the addressing of adjacent mesh points, we use \textit{maps} to describe which multi-indexes are required.
To make things clear, a map is defined as a \textit{function that accepts a multi-index and returns multiple multi-indexes}:

\vspace{1em}
\begin{equation*}
  ((t_1, i_1),) \to ((u^1_1, j^1_1),) ,\ ((u^2_1, j^2_1),) ,\ \dots ,\ ((u^m_1, j^m_1),).
\end{equation*}
\vspace{1em}

As an example, assuming a triangular mesh, the code

\begin{minted}[xleftmargin=4em]{python}
loop(c := mesh.cells.index, kernel(dat[closure(c)]))
\end{minted}

uses the $\closure$ DMPlex restriction operation to yield a map of the form

\vspace{1em}
\begin{equation*}
  ((C, c_0),)
  \to ((C, c_0),)
  ,\ ((E, e_0),) ,\ ((E, e_1),) ,\ ((E, e_2),)
  ,\ ((V, v_0),) ,\ ((V, v_1),) ,\ ((V, v_2),).
\end{equation*}
\vspace{1em}

In an analogous way to layout functions, maps can be implemented either using index functions (i.e. $e_0 = f(c_0)$), or lookup tables (\py{e0 = map[c0]}) depending on whether or not there is structure to exploit in the data layout.
This enables, for example, the use of structured meshes without needing to incur the memory bandwidth cost of tabulating a lookup table.

The approach just described enables arbitrary map composition because maps now work in line with how DMPlex handles restrictions, namely functions between mesh points, rather than between points and nodes.

Note that at present we restrict maps to only apply to multi-indexes of length 1.
This is sufficient for unstructured meshes but a necessity for partially-structured meshes.
This is discussed in detail in Section~\ref{sec:future_partialstructure}.

\subsubsection{Raggedness and sparsity}
\label{sec:impl_datalayout_ragged}

There are occasions where one needs a data structure where the extent of an inner dimension depends on an outer one.
This occurs for example in variable layer extruded meshes - the extent of the inner dimension (the columns) is dependent upon the mesh point in the base mesh.
To get this to work, \pyop3 needs to generate code that resembles:

\begin{minted}[xleftmargin=4em]{c}
for (int i=0; i<N; ++i)
  for (int j=0; j<nlayers[i]; ++j)
    ...
\end{minted}

Note how the inner loop extent is dependent upon the outer one via the \clang{nlayers} array.

In \pyop3, such a `ragged' data structure can be initialised in the following way:

\begin{minted}[xleftmargin=4em]{python}
nlayers = Dat(MultiAxis(AxisPart(N)), dtype=int)
root = (
  MultiAxis()
  .add_part(AxisPart(N, id="outer"))
  .add_subaxis("outer", AxisPart(nlayers))
)
\end{minted}

Instead of using a constant integer value to prescribe the extent of the inner \py{AxisPart}, another \py{MultiAxis}-using data structure is used instead.
Having extents also use \py{MultiAxes} is advantageous as the code generation procedure can be shared.

We can also use the same technique to generate code for maps with \textit{variable arity}.
An example of this would be for $\plexstar(p)$ for $p$ a vertex since the number of incident edges on a vertex is variable.
This is useful for patch-based computations (see Section~\ref{sec:future_patch}).

Although not yet implemented, it should be entirely possible to implement sparse data structures by making small extensions to the existing abstraction.
If we consider a sparse matrix compressed with compressed-sparse-row (CSR) format, the data layout is described using two arrays: the row and column indices.
This is very similar to our existing solution for ragged arrays except that we assume that the internal dimension is logically dense, and hence do not need to specify column indices.

It should be noted however that we are assuming that the matrix is local to a single processor.
Parallel sparse matrices are considerably more challenging to implement and so we defer the work to PETSc (see Section~\ref{sec:impl_parallel}).

\subsubsection{Orienting degrees-of-freedom}
\label{sec:impl_orientation}

Having a hierarchical, `mesh-aware' data layout makes it much more straightforward to correctly handle orientations when packing stencils.
The way it works is as follows:
1) The DMPlex restrictions return orientation information of the mesh entities as well as the multi-index,
2) Each axis below the one selected by the map is transformed according to some predefined rule, using the orientation as a selector for the transformation.

As an example, we refer back to Figure~\ref{fig:orient_vector_flip}.
Noting that the data layout will decompose into points, nodes-per-point and \glspl{dof}-per-node, the transformation to canonical layout is done in two steps: permuting the nodes on the edge and flipping the individual \glspl{dof} such that they point in the right direction.
These operations map naturally to the different sub-axes of the data layout - the permutation applies to nodes and so can apply to the node axis, and the reflection applies to each \glspl{dof} individually and so can be applied to the \glspl{dof} axis.

\subsubsection{Data layout transformations}
\label{sec:impl_datalayoutopt}

\begin{figure}
  \centering
  \begin{subfigure}{.65\textwidth}
    \centering
    \begin{tikzpicture}[y=-1cm,scale=.75]
      \begin{scope}[xshift=3.25cm, yshift=0cm]
        \filldraw[draw=black, fill=blue!60] (0,0) rectangle (1,1);
        \filldraw[draw=black, fill=red!60] (1,0) rectangle (2,1);
        \node[at={(.5,.5)}, ptlabel] {$V_0$};
        \node[at={(1.5,.5)}, ptlabel] {$V_1$};
      \end{scope}

      \begin{scope}[yshift=-2cm]
        \begin{scope}[xshift=0cm]
          \fill[lightgray] (0,0) rectangle (4,1);
          \filldraw[draw=black, fill=white] (0.5,0) rectangle (1.5,1);
          \filldraw[draw=black, fill=white] (1.5,0) rectangle (2.5,1);
          \filldraw[draw=black, fill=white] (2.5,0) rectangle (3.5,1);
          \node[at={(1,.5)}, ptlabel] {$c_0$};
          \node[at={(2,.5)}, ptlabel] {$v_1$};
          \node[at={(3,.5)}, ptlabel] {$c_4$};
          \draw (0,0) -- (4,0);
          \draw (0,1) -- (4,1);
        \end{scope}

        \begin{scope}[xshift=4.5cm]
          \fill[lightgray] (0,0) rectangle (4,1);
          \filldraw[draw=black, fill=white] (0.5,0) rectangle (1.5,1);
          \filldraw[draw=black, fill=white] (1.5,0) rectangle (2.5,1);
          \filldraw[draw=black, fill=white] (2.5,0) rectangle (3.5,1);
          \node[at={(1,.5)}, ptlabel] {$c_0$};
          \node[at={(2,.5)}, ptlabel] {$v_1$};
          \node[at={(3,.5)}, ptlabel] {$c_4$};
          \draw (0,0) -- (4,0);
          \draw (0,1) -- (4,1);
        \end{scope}
      \end{scope}

      \draw (3.25,1) -- (0,2);
      \draw (4.25,1) -- (4,2);
      \draw (4.25,1) -- ({0+4.5},2);
      \draw (5.25,1) -- ({4+4.5},2);
    \end{tikzpicture}
    \caption{}
    \label{fig:mixedreorder_outer}
  \end{subfigure}
  %
  \begin{subfigure}{.3\textwidth}
    \centering
    \begin{tikzpicture}[x=.8cm,y=-1cm]
      \draw (0,0) .. controls (-.2,0) and (-.2,3) .. (0,3);
      \draw (1,0) .. controls (1.2,0) and (1.2,3) .. (1,3);
      \filldraw [fill=blue!60,rounded corners,draw=none]
        (.1,.05) -- (.9,.05) -- (.9,1.45) -- (.1,1.45) -- cycle;
      \filldraw [fill=red!60,rounded corners,draw=none]
        (.1,1.55) -- (.9,1.55) -- (.9,2.95) -- (.1,2.95) -- cycle;
    \end{tikzpicture}
    \caption{}
    \label{fig:mixedreorder_outer_vec}
  \end{subfigure}

  \vspace{1em}
 
  \begin{subfigure}{.65\textwidth}
    \centering
    \begin{tikzpicture}[y=-1cm,scale=.75]
      \begin{scope}[xshift=0cm,yshift=0cm]
        \fill[lightgray] (0,0) rectangle (4,1);
        \filldraw[draw=black, fill=white] (0.5,0) rectangle (1.5,1);
        \filldraw[draw=black, fill=white] (1.5,0) rectangle (2.5,1);
        \filldraw[draw=black, fill=white] (2.5,0) rectangle (3.5,1);
        \node[at={(1,.5)}, ptlabel] {$c_0$};
        \node[at={(2,.5)}, ptlabel] {$v_1$};
        \node[at={(3,.5)}, ptlabel] {$c_4$};
        \draw (0,0) -- (4,0);
        \draw (0,1) -- (4,1);
      \end{scope}

      \begin{scope}[xshift=1cm, yshift=-2cm]
        \filldraw[draw=black, fill=blue!60] (0,0) rectangle (1,1);
        \filldraw[draw=black, fill=red!60] (1,0) rectangle (2,1);
        \node[at={(.5,.5)}, ptlabel] {$V_0$};
        \node[at={(1.5,.5)}, ptlabel] {$V_1$};
      \end{scope}

      \draw (1.5,1) -- (1,2);
      \draw (2.5,1) -- (3,2);
    \end{tikzpicture}
    \caption{}
    \label{fig:mixedreorder_inner}
  \end{subfigure}
  %
  \begin{subfigure}{.3\textwidth}
    \centering
    \begin{tikzpicture}[x=.8cm,y=-1cm]
      \tikzstyle{entry} = [rounded corners,draw=none];
      \tikzstyle{blue} = [entry,blue!60];
      \tikzstyle{red} = [entry,red!60];
      \draw (0,0) .. controls (-.2,0) and (-.2,3) .. (0,3);
      \draw (1,0) .. controls (1.2,0) and (1.2,3) .. (1,3);
      \filldraw [blue] (.1,.05) -- (.9,.05) -- (.9,.45) -- (.1,.45) -- cycle;
      \filldraw [red] (.1,.55) -- (.9,.55) -- (.9,.95) -- (.1,.95) -- cycle;
      \filldraw [blue] (.1,1.05) -- (.9,1.05) -- (.9,1.45) -- (.1,1.45) -- cycle;
      \filldraw [red] (.1,1.55) -- (.9,1.55) -- (.9,1.95) -- (.1,1.95) -- cycle;
      \filldraw [blue] (.1,2.05) -- (.9,2.05) -- (.9,2.45) -- (.1,2.45) -- cycle;
      \filldraw [red] (.1,2.55) -- (.9,2.55) -- (.9,2.95) -- (.1,2.95) -- cycle;
    \end{tikzpicture}
    \caption{}
    \label{fig:mixedreorder_inner_vec}
  \end{subfigure}
  \caption{}
  \label{fig:mixedreorder}
\end{figure}


With this decomposition of data layouts in a flexible, declarative hierarchy, it is now relatively straightforward to reason about making \textit{data layout transformations} to improve properties such as the effective working-set size (Section~\ref{sec:background_opt_locality}).

Some of the possible optimisations include:

\begin{paragraph}{Swapping axes}
\pyop3 makes it straightforward to swap a pair of \py{MultiAxis} such that the `inner' axis becomes the `outer' and vice versa.

An example of this is shown in Figure~\ref{fig:mixedreorder}.
Typically a `mixed' system like this - here formed of $V_0$ and $V_1$ - stores data in a blocked format (Figures~\ref{fig:mixedreorder_outer} and~\ref{fig:mixedreorder_outer_vec}).
This means that the \glspl{dof} corresponding to the same mesh point for $V_0$ and $V_1$ are very far apart in memory.
If they are both used in the local computation then this constitutes poor data locality.

To rectify the situation, we can, when applicable, permute the axes such that the mixed components are stored per mesh point, adjacent in memory.
An example of this is shown in Figures~\ref{fig:mixedreorder_inner} and~\ref{fig:mixedreorder_inner_vec}.
\end{paragraph}

\begin{paragraph}{Reordering data within axes}
Once the axes have been ordered in the most advantageous way, we can now begin to rearrange the entries in a \py{MultiAxis} to maximise locality.
In the context of meshes, these reorderings could correspond to, for example, an \gls{rcm} renumbering of the mesh entities or a different ordering of the elements up the columns of an extruded mesh.
In addition to improving locality, one can also perform these reorderings in order to allow for efficient subset queries.
Certain preconditioners require access to subsets of the mesh entities and the data layout can be modified so that the relevant \glspl{dof} are contiguous in memory.

To implement this reordering, \pyop3 simply requires that a different layout function be given to the respective \py{AxisParts}.
\end{paragraph}

\subsubsection{Other locality optimisations}

\pyop3's abstraction also enables code transformations such as vectorisation and tiling (Section~\ref{sec:background_opt_locality}).
Such optimisations are not data layout transformations, but transformations of the iteration set (i.e. the multi-index).
In both cases, the flat iteration over some axis needs to be transformed to a set of nested loops of the form

\begin{minipage}{\textwidth}
\begin{minted}[xleftmargin=4em]{c}
for (int i=0; i<NOUTER; ++i) {
  for (int j=0; j<NINNER; ++j) {
    int k = i*NINNER + j;
    ...
  }
}
\end{minted}
\end{minipage}

In the case of vectorisation, \clang{NINNER} would correspond to the length of the vector lanes of the CPU, and if tiling it would be tailored to its cache sizes.

It is valuable to note that, for unstructured meshes, tiling on its own is redundant as the amount of data shared between successive loops and prefetched by the hardware is already maximised by having an appropriate mesh numbering.
The optimisation only becomes valuable when combined with kernel fusion to produce time tiling.
Then the size of tiles should be chosen such that data required for both loops remains in cache between kernel invocations.

\subsubsection{Enabling new research}

In addition to the performance benefits espoused above, this new data layout abstraction should enable one to implement a number of new mathematical methods heretofore impossible to implement in \pyop2:

\begin{itemize}
  \item
    \textbf{p-adaptivity}
    In order to reduce the errors in a simulation, one may vary the polynomial degree of particular cells in a process known as p-adaptivity.
    It is tricky to automate a stencil code for looping over the mesh because:
    a) multiple local kernels are needed, one for each degree, and b) there are `hanging' \glspl{dof} at the boundaries between cells of differing degrees.
    Problem (a) is trivial to resolve in \pyop3.
    Rather than having a \py{MultiAxis} that is composed only of cells, edges and vertices (each a distinct \py{AxisPart}), additional \py{AxisParts} can be added such that mesh points of different degree are associated with a unique \py{AxisPart}.
    Problem (b) is more challenging to solve and requires the addition of \textit{constraints} to the abstraction (Section~\ref{sec:future_constraints}).

  \item
    \textbf{Mixed meshes}
    A mixed mesh is a mesh composed of multiple different types of polytope (e.g. triangles and squares).
    Iterating over such a mesh poses the same fundamental problem as p-adaptivity: different local kernels are required depending on the polytope type.
    Since \pyop3 is `mesh-aware' and can reason about the different classes of mesh points, this problem becomes trivial.

  \item
    \textbf{Particle-in-cell methods}
    Particle-in-cell methods are a type of numerical method where the cells of a mesh are associated with a number of, possibly advecting, particles.
    Since the number of particles differs between cells, a variable arity map is required to address them (Section~\ref{sec:impl_datalayout_ragged}).
\end{itemize}

\subsection{Parallel design}
\label{sec:impl_parallel}

\begin{figure}
  \centering
  \begin{tikzpicture}[scale=1.3]
    % define styles
    \tkzSetUpStyle[draw=white,line width=5]{cell}
    \tkzSetUpStyle[line width=2,shorten >=.2cm,shorten <=.2cm]{edge}

    \tkzSetUpStyle[cell,fill=red!50]{p1cell}
    \tkzSetUpStyle[cell,fill=red!25]{p1cellhalo}
    \tkzSetUpStyle[edge,draw=red!80]{p1edge}
    \tkzSetUpStyle[draw=red!80,fill=red!80]{p1vert}
    \tkzSetUpStyle[cell,fill=blue!50]{p2cell}
    \tkzSetUpStyle[cell,fill=blue!25]{p2cellhalo}
    \tkzSetUpStyle[edge,draw=blue!80]{p2edge}
    \tkzSetUpStyle[draw=blue!80,fill=blue!80]{p2vert}

    \tkzSetUpStyle[densely dashed,shorten >=.1cm,shorten <=.1cm,line width=.5]{connector}

    % process 1
    \begin{scope}
      % define nodes
      \tkzDefPoint(0,0){p1v0}
      \tkzDefPoint(.1,1.1){p1v1}
      \tkzDefPoint(0,1.9){p1v2}
      \tkzDefPoint(.2,3.1){p1v3}
      \tkzDefPoint(1.1,0){p1v4}
      \tkzDefPoint(1,1){p1v5}
      \tkzDefPoint(.9,2){p1v6}
      \tkzDefPoint(1,3){p1v7}
      \tkzDefPoint(2,0){p1v8}
      \tkzDefPoint(2.1,1){p1v9}
      \tkzDefPoint(2,2.1){p1v10}
      \tkzDefPoint(1.9,3.2){p1v11}
      \tkzDefPoint(3,-.1){p1v12}
      \tkzDefPoint(3.1,.9){p1v13}
      \tkzDefPoint(3,2.1){p1v14}
      \tkzDefPoint(3.1,3.1){p1v15}

      % cells
      \tkzDrawPolygon[p1cell](p1v0,p1v1,p1v4)
      \tkzDrawPolygon[p1cell](p1v1,p1v4,p1v5)
      \tkzDrawPolygon[p1cell](p1v1,p1v5,p1v6)
      \tkzDrawPolygon[p1cell](p1v1,p1v2,p1v6)
      \tkzDrawPolygon[p1cell](p1v2,p1v3,p1v6)
      \tkzDrawPolygon[p1cell](p1v3,p1v6,p1v7)
      \tkzDrawPolygon[p1cell](p1v4,p1v8,p1v9)
      \tkzDrawPolygon[p1cell](p1v4,p1v5,p1v9)
      \tkzDrawPolygon[p1cell](p1v5,p1v9,p1v10)
      \tkzDrawPolygon[p1cell](p1v5,p1v6,p1v10)
      \tkzDrawPolygon[p1cell](p1v6,p1v7,p1v10)
      \tkzDrawPolygon[p1cell](p1v7,p1v10,p1v11)
      \tkzDrawPolygon[p1cell](p1v8,p1v9,p1v12)
      \tkzDrawPolygon[p2cellhalo](p1v9,p1v12,p1v13)
      \tkzDrawPolygon[p2cellhalo](p1v9,p1v13,p1v14)
      \tkzDrawPolygon[p1cell](p1v9,p1v10,p1v14)
      \tkzDrawPolygon[p2cellhalo](p1v10,p1v14,p1v15)
      \tkzDrawPolygon[p1cell](p1v10,p1v11,p1v15)

      % edges
      \tkzDrawSegments[p1edge](p1v4,p1v5 p1v5,p1v6 p1v6,p1v7)
      \tkzDrawSegments[p1edge](p1v8,p1v9 p1v9,p1v10 p1v10,p1v11)
      \tkzDrawSegments[p2edge,opacity=.5](p1v12,p1v13 p1v13,p1v14 p1v14,p1v15)

      \tkzDrawSegments[p1edge](p1v0,p1v4 p1v4,p1v8)
      \tkzDrawSegments[p1edge](p1v1,p1v5 p1v5,p1v9)
      \tkzDrawSegments[p1edge](p1v2,p1v6 p1v6,p1v10)
      \tkzDrawSegments[p1edge](p1v3,p1v7 p1v7,p1v11 p1v11,p1v15)
      \tkzDrawSegments[p2edge,opacity=.5](p1v8,p1v12 p1v9,p1v13 p1v10,p1v14)

      \tkzDrawSegments[p1edge](p1v1,p1v4 p1v1,p1v6 p1v3,p1v6)
      \tkzDrawSegments[p1edge](p1v4,p1v9 p1v5,p1v10 p1v7,p1v10)
      \tkzDrawSegments[p2edge,opacity=.5](p1v9,p1v12 p1v9,p1v14 p1v10,p1v15)  % mimics process 2

      % vertices
      % \tkzDrawPoints[p1vert](p1v0,p1v1,p1v2,p1v3,p1v4,p1v5,p1v6,p1v7)  % core
      \tkzDrawPoints[p1vert,size=5](p1v4,p1v5,p1v6,p1v7)  % core
      % \tkzDrawPoints[p1vert,diamond,size=6](p1v8,p1v9,p1v10,p1v11)  % owned
      \tkzDrawPoints[p1vert,diamond,size=6](p1v8,p1v9,p1v11)  % owned
      \tkzDrawPoints[p1vert,diamond,size=6,draw=black](p1v10)  % owned
      \tkzDrawPoints[p2vert,diamond,opacity=.5,size=6](p1v12,p1v13,p1v14,p1v15)

      % debugging
      % \tkzLabelPoints[anchor=south,font=\tiny](p1v0,p1v1,p1v2,p1v3,p1v4,p1v5,p1v6,p1v7,p1v8,p1v9,p1v10,p1v11,p1v12,p1v13,p1v14,p1v15)

      % draw a sample patch
      \tkzDefShiftPoint[p1v7](-.2,.2){p1v7patch}
      \tkzDefShiftPoint[p1v11](0,.2){p1v11patch}
      \tkzDefShiftPoint[p1v15](.2,.2){p1v15patch}
      \tkzDefShiftPoint[p1v14](.2,-.1){p1v14patch}
      \tkzDefShiftPoint[p1v9](.1,-.2){p1v9patch}
      \tkzDefShiftPoint[p1v5](-.2,-.2){p1v5patch}
      \tkzDefShiftPoint[p1v6](-.2,0){p1v6patch}
      \filldraw[draw=black,fill=black,fill opacity=.1,rounded corners=3]
      % \filldraw[draw=none,fill=blue,fill opacity=.4,rounded corners=2]
      % \filldraw[pattern={Hatch[distance=3mm,angle=45]},draw=black,rounded corners=2]
        (p1v7patch) -- (p1v11patch) -- (p1v15patch) -- (p1v14patch) -- (p1v9patch) --
        (p1v5patch) -- (p1v6patch) -- cycle;
      % \draw (p1v10) circle [radius=5pt];
      % \tkzDrawPoint[size=1pt](p1v10)

      % label "core" and "owned"
      \node (p1core) [inner sep=0pt,xshift=-20pt,yshift=20pt] at (p1v7) {\footnotesize core};
      \node (p1owned) [inner sep=0pt,xshift=-10pt,yshift=20pt] at (p1v11) {\footnotesize owned};
      \draw [-{stealth},shorten >=4pt,shorten <=2pt] (p1core.south) -- (p1v7.north);
      \draw [-{stealth},shorten >=4pt,shorten <=2pt] (p1owned.south) -- (p1v11.north);
    \end{scope}

    % process 2
    \begin{scope}[xshift=5cm]
      % define nodes
      \tkzDefPoint(0,0){p2v0}
      \tkzDefPoint(.1,1){p2v1}
      \tkzDefPoint(0,2.1){p2v2}
      \tkzDefPoint(-.1,3.2){p2v3}
      \tkzDefPoint(1,-.1){p2v4}
      \tkzDefPoint(1.1,.9){p2v5}
      \tkzDefPoint(1,2.1){p2v6}
      \tkzDefPoint(1.1,3.1){p2v7}
      \tkzDefPoint(2,-.1){p2v8}
      \tkzDefPoint(2,1.1){p2v9}
      \tkzDefPoint(2.1,2){p2v10}
      \tkzDefPoint(2,2.9){p2v11}
      \tkzDefPoint(3,.1){p2v12}
      \tkzDefPoint(3.1,1){p2v13}
      \tkzDefPoint(2.9,2){p2v14}
      \tkzDefPoint(2.9,3.1){p2v15}

      % cells
      \tkzDrawPolygon[p1cellhalo](p2v0,p2v1,p2v4)
      \tkzDrawPolygon[p2cell](p2v1,p2v4,p2v5)
      \tkzDrawPolygon[p2cell](p2v1,p2v5,p2v6)
      \tkzDrawPolygon[p1cellhalo](p2v1,p2v2,p2v6)
      \tkzDrawPolygon[p2cell](p2v2,p2v6,p2v7)
      \tkzDrawPolygon[p1cellhalo](p2v2,p2v3,p2v7)
      \tkzDrawPolygon[p2cell](p2v4,p2v5,p2v8)
      \tkzDrawPolygon[p2cell](p2v5,p2v8,p2v9)
      \tkzDrawPolygon[p2cell](p2v5,p2v6,p2v9)
      \tkzDrawPolygon[p2cell](p2v6,p2v9,p2v10)
      \tkzDrawPolygon[p2cell](p2v6,p2v10,p2v11)
      \tkzDrawPolygon[p2cell](p2v6,p2v7,p2v11)
      \tkzDrawPolygon[p2cell](p2v8,p2v9,p2v12)
      \tkzDrawPolygon[p2cell](p2v9,p2v12,p2v13)
      \tkzDrawPolygon[p2cell](p2v9,p2v10,p2v13)
      \tkzDrawPolygon[p2cell](p2v10,p2v13,p2v14)
      \tkzDrawPolygon[p2cell](p2v10,p2v14,p2v15)
      \tkzDrawPolygon[p2cell](p2v10,p2v11,p2v15)

      % edges
      \tkzDrawSegments[p1edge,opacity=.5](p2v0,p2v1 p2v1,p2v2 p2v2,p2v3)
      \tkzDrawSegments[p2edge](p2v4,p2v5 p2v5,p2v6 p2v6,p2v7)
      \tkzDrawSegments[p2edge](p2v8,p2v9 p2v9,p2v10 p2v10,p2v11)

      \tkzDrawSegments[p2edge](p2v0,p2v4 p2v4,p2v8 p2v8,p2v12)
      \tkzDrawSegments[p2edge](p2v1,p2v5 p2v5,p2v9 p2v9,p2v13)
      \tkzDrawSegments[p2edge](p2v2,p2v6 p2v6,p2v10 p2v10,p2v14)
      \tkzDrawSegments[p2edge](p2v7,p2v11 p2v11,p2v15)
      \tkzDrawSegments[p1edge,opacity=.5](p2v3,p2v7)

      \tkzDrawSegments[p2edge](p2v1,p2v4 p2v1,p2v6 p2v2,p2v7)
      \tkzDrawSegments[p2edge](p2v5,p2v8)
      \tkzDrawSegments[p2edge](p2v6,p2v9)
      \tkzDrawSegments[p2edge](p2v6,p2v11)
      \tkzDrawSegments[p2edge](p2v9,p2v12 p2v10,p2v13 p2v10,p2v15)

      % vertices
      \tkzDrawPoints[p2vert,size=5](p2v8,p2v9,p2v10,p2v11)  % core
      \tkzDrawPoints[p2vert,diamond,size=6](p2v4,p2v5,p2v6,p2v7)  % owned
      \tkzDrawPoints[p1vert,diamond,size=6,opacity=.5](p2v0,p2v1,p2v2,p2v3)  % halo

      % label "core" and "owned"
      \node (p2core) [inner sep=0pt,xshift=15pt,yshift=20pt] at (p2v11) {\footnotesize core};
      \node (p2owned) [inner sep=0pt,xshift=10pt,yshift=20pt] at (p2v7) {\footnotesize owned};
      \draw [-{stealth},shorten >=4pt,shorten <=2pt] (p2core.south) -- (p2v11.north);
      \draw [-{stealth},shorten >=4pt,shorten <=2pt] (p2owned.south) -- (p2v7.north);

      % debugging
      % \tkzLabelPoints[anchor=south,font=\tiny](p2v0,p2v1,p2v2,p2v3,p2v4,p2v5,p2v6,p2v7,p2v8,p2v9,p2v10,p2v11,p2v12,p2v13,p2v14,p2v15)
    \end{scope}

    % connect (sample of) equivalent points
    \draw [-{stealth},connector,shorten >=4pt,shorten <=4pt] (p1v11) to [bend left=45] (p2v3);
    \draw [{stealth}-,connector,shorten >=4pt,shorten <=4pt] (p1v15) to [bend left=45] (p2v7);
    \draw [-{stealth},connector,shorten >=4pt,shorten <=4pt] (p1v8) to [bend right=45] (p2v0);
    \draw [{stealth}-,connector,shorten >=4pt,shorten <=4pt] (p1v12) to [bend right=45] (p2v4);

    % label processes
    \node (p1name) at (1.5,4.2) {Process 1};
    \node (p2name) at (6.5,4.2) {Process 2};
  \end{tikzpicture}
  \caption{
    An example mesh distributed between two processes (red and blue).
    The mesh is intended for vertex patches (shaded) and so the overlap is chosen such that all required \glspl{dof} are stored locally.
    `Core' vertices are stored as circles and `owned' as diamonds.
    The direction of halo exchanges is indicated by the arrows.
  }
  \label{fig:halos}
\end{figure}

% same fundamental data types

% how to do insertion into off-diagonal parts in PETSc? how would that work with my stuff?

%\pyop3 has the same parallel design as \pyop2.
% ghost/halo values for overlaps
% globals with reductions
% can do variable halo size to replace exchanges with redundant communication - used in sparse tiling
% clever layout means we can exchange core, then owned, interleaving communication
% PETSc Mat for parallel matrices

% we assume ragged only works in serial

% PETSc is scalable (cite?)
% PyOP2 (via Firedrake assembly) has good weak scalability

% mention strong scaling

\subsection{Avoiding Python overhead}
\label{sec:impl_overhead}

We have chosen to write \pyop3 in Python... %why?

% halo exchanges in C
% do_loop(...) vs expr = loop(...) - easy to swap out data structures via **kwargs, weakref

% also strong-scaling (but shown otherwise)
% key is to minimise amount of Python in the hot loops via judicious caching, persistent objects and expanded codegen

% use DG advection as an example - caching and persistent objects can be shown to be key
% but now things like halo exchanges for valid halos (i.e. noops) have non-trivial cost -> move out of Python to the generated code.


  \section{Future work}
\label{sec:future}

\subsection{Direct addressing for partially-structured meshes}
\label{sec:future_partialstructure}

% stand to make a small memory saving - affects prefetch, loop unrolling, working set size, bandwidth stuff
% not necessarily worth it... but permitted by abstraction

% below are unified by "mesh transformation" approach in DMPlex - should be possible to reconstruct
% data hierarchy by inspecting labels.

% relax constraint that a map cannot have parents
Under such a prescription, a map is therefore a \textit{function between multi-indexes} of the form

\vspace{1em}

\begin{equation*}
  \begin{pmatrix} (t_1, i_1) \\ (t_2, i_2) \\ \dots \\ (t_n, i_n) \end{pmatrix}
  \to
  \begin{pmatrix} (u^1_1, j^1_1) \\ (u^1_2, j^1_2) \\ \dots \\ (u^1_{k_1}, j^1_{k_1}) \end{pmatrix}
  ,
  \begin{pmatrix} (u^2_1, j^2_1) \\ (u^2_2, j^2_2) \\ \dots \\ (u^2_{k_2}, j^2_{k_2}) \end{pmatrix}
  , \dots ,
  \begin{pmatrix} (u^m_1, j^m_1) \\ (u^m_2, j^m_2) \\ \dots \\ (u^m_{k_m}, j^m_{k_m}) \end{pmatrix}.
\end{equation*}

\vspace{1em}

\subsubsection{Extruded meshes}

\subsubsection{Regular mesh refinement}

% refinement 1
\begin{tikzpicture}

  \tkzDefPoint(0,0){v0}
  \tkzDefShiftPoint[v0](60:4){v1}
  \tkzDefShiftPoint[v0](0:4){v2}
  \tkzDrawPolygon(v0,v1,v2)

  \tkzLabelPoint[below left](v0){$v_0$}
  \tkzLabelPoint[above](v1){$v_1$}
  \tkzLabelPoint[below right](v2){$v_2$}

  \tkzLabelSegment(v0,v1){$e_0$}
  \tkzLabelSegment(v1,v2){$e_1$}
  \tkzLabelSegment(v2,v0){$e_2$}

  \tkzDefBarycentricPoint(v0=1,v1=1,v2=1) \tkzGetPoint{c0}
  \tkzLabelPoint[centered](c0){$c_0$}

\end{tikzpicture}

\vspace{1em}

% refinement 2
\begin{tikzpicture}

  \tkzDefPoint(0,0){v0}
  \tkzDefShiftPoint[v0](60:4){v1}
  \tkzDefShiftPoint[v0](0:4){v2}
  \tkzDrawPolygon(v0,v1,v2)

  \tkzDefMidPoint(v0,v1) \tkzGetPoint{e0v0}
  \tkzDefMidPoint(v1,v2) \tkzGetPoint{e1v0}
  \tkzDefMidPoint(v2,v0) \tkzGetPoint{e2v0}

  \tkzDrawSegment(e0v0,e1v0)
  \tkzDrawSegment(e1v0,e2v0)
  \tkzDrawSegment(e2v0,e0v0)

  % note: this method is outdated - bisect instead
  \begin{pgfonlayer}{background}
    \tkzDefBarycentricPoint(e0v0=25,v1=1,e1v0=1,e2v0=1) \tkzGetPoint{e0v0inner}
    \tkzDefBarycentricPoint(e0v0=1,v1=25,e1v0=1,e2v0=1) \tkzGetPoint{v1inner}
    \tkzDefBarycentricPoint(e0v0=1,v1=1,e1v0=25,e2v0=1) \tkzGetPoint{e1v0inner}
    \tkzDefBarycentricPoint(e0v0=1,v1=1,e1v0=1,e2v0=25) \tkzGetPoint{e2v0inner}


    \draw[rounded corners] (e0v0inner) -- (v1inner) -- (e1v0inner) -- (e2v0inner) -- cycle;
  \end{pgfonlayer}

\end{tikzpicture}

% \vspace{1em}
\pagebreak

% refinement 3
\begin{tikzpicture}

  \tkzDefPoint(0,0){v0}
  \tkzDefShiftPoint[v0](60:4){v1}
  \tkzDefShiftPoint[v0](0:4){v2}
  \tkzDrawPolygon(v0,v1,v2)

  \tkzDefMidPoint(v0,v1) \tkzGetPoint{e0v0}
  \tkzDefMidPoint(v1,v2) \tkzGetPoint{e1v0}
  \tkzDefMidPoint(v2,v0) \tkzGetPoint{e2v0}

  \tkzDrawSegment(e0v0,e1v0)
  \tkzDrawSegment(e1v0,e2v0)
  \tkzDrawSegment(e2v0,e0v0)

  \tkzDefMidPoint(e1v0,v2) \tkzGetPoint{e1e1v0}
  \tkzDefMidPoint(v2,e2v0) \tkzGetPoint{e2e0v0}
  \tkzDefMidPoint(e1v0,e2v0) \tkzGetPoint{c0e2v0}

  \tkzDrawSegment(e1e1v0,e2e0v0)
  \tkzDrawSegment(e2e0v0,c0e2v0)
  \tkzDrawSegment(c0e2v0,e1e1v0)

  % patch
  \begin{pgfonlayer}{background}
    % find points by bisecting the angles
    \tkzDefShiftPoint[e0v0](-30:0.15){e0v0inner}
    \tkzDefShiftPoint[e1v0](-120:0.1){e1v0inner}
    \tkzDefShiftPoint[e1e1v0](150:0.15){e1e1v0inner}
    \tkzDefShiftPoint[c0e2v0](120:0.1){c0e2v0inner}
    \tkzDefShiftPoint[e2v0](90:0.15){e2v0inner}

    % source: https://tikz.dev/base-paths#sec-102.12
    \pgfsetcornersarced{\pgfpoint{1mm}{1mm}}
    \filldraw[color=blue!20] (e0v0inner) -- (e1v0inner) -- (e1e1v0inner) -- (c0e2v0inner) -- (e2v0inner) -- cycle;
    \pgfsetcornersarced{\pgfpointorigin}
  \end{pgfonlayer}

  % add labels
  \tkzDefBarycentricPoint(e0v0=1,e1v0=1,e2v0=1) \tkzGetPoint{c0c1}
  \node [xshift=-2cm,yshift=.8cm] (c0c1label) at (c0c1) {$(c_i,c_1)$};
  \draw (c0c1label) -- (c0c1);

  \tkzDefMidPoint(e1v0,c0e2v0) \tkzGetPoint{c0e2}
  \node [xshift=2cm,yshift=.8cm] (c0e2label) at (c0e2) {$(c_i,e_2,e_0)$};
  \draw (c0e2label) -- (c0e2);

  \tkzDefBarycentricPoint(e1v0=1,e1e1v0=1,c0e2v0=1) \tkzGetPoint{c0c3c2}
  \node [xshift=2cm,yshift=-.2cm] (c0c3c2label) at (c0c3c2) {$(c_i,c_3,c_2)$};
  \draw (c0c3c2label) -- (c0c3c2);

\end{tikzpicture}

\vspace{2em}

% data layout for patch
\begin{tikzpicture}[y=-1cm]
  \begin{scope}[xshift=1.5cm, yshift=0cm]
    \fill[lightgray] (0,0) rectangle (4,1);
    \filldraw[draw=black, fill=white] (1.5,0) rectangle ++ (1,1);
    \node[at={(2,.5)}, ptlabel] {$c_i$};
    \draw (0,0) -- (4,0);
    \draw (0,1) -- (4,1);
  \end{scope}

  \begin{scope}[xshift=.5cm, yshift=-2cm]
    \filldraw[draw=black, fill=white] (0,0) rectangle ++ (1,1);
    \filldraw[draw=black, fill=blue!20] (1,0) rectangle ++ (1,1);
    \filldraw[draw=black, fill=white] (2,0) rectangle ++ (1,1);
    \filldraw[draw=black, fill=white] (3,0) rectangle ++ (1,1);
    \filldraw[draw=black, fill=white] (4,0) rectangle ++ (1,1);
    \filldraw[draw=black, fill=white] (5,0) rectangle ++ (1,1);
    \filldraw[draw=black, fill=white] (6,0) rectangle ++ (1,1);
    \node[at={(.5,.5)}, ptlabel] {$c_0$};
    \node[at={(1.5,.5)}, ptlabel] {$c_1$};
    \node[at={(2.5,.5)}, ptlabel] {$c_2$};
    \node[at={(3.5,.5)}, ptlabel] {$c_3$};
    \node[at={(4.5,.5)}, ptlabel] {$e_0$};
    \node[at={(5.5,.5)}, ptlabel] {$e_1$};
    \node[at={(6.5,.5)}, ptlabel] {$e_2$};

    % \draw[->] (2.8,-1) .. controls ([yshift=-.4cm] 2.6,-1) and ([yshift=.6cm] 1,0) .. (.8,0);
    % \draw[->] (3,-1) .. controls ([yshift=-.6cm] 3,-1) and ([yshift=1cm] 3.5,0) .. (3.5,0);
    % \draw[->] (3.2,-1) .. controls ([yshift=-.4cm] 3,-1) and ([yshift=.6cm] 6,0) .. (6.2,0);
  \end{scope}

  \begin{scope}[xshift=0cm, yshift=-4cm]
    % c3
    \begin{scope}[xshift=0cm]
      \fill[lightgray] (0,0) rectangle (4,1);
      \filldraw[draw=black, fill=white] (.5,0) rectangle ++ (1,1);
      \filldraw[draw=black, fill=blue!20] (1.5,0) rectangle ++ (1,1);
      \filldraw[draw=black, fill=white] (2.5,0) rectangle ++ (1,1);
      \node[at={(1,.5)}, ptlabel] {$c_1$};
      \node[at={(2,.5)}, ptlabel] {$c_2$};
      \node[at={(3,.5)}, ptlabel] {$c_3$};
      \draw (0,0) -- (4,0);
      \draw (0,1) -- (4,1);
      % \draw[->] (4,-1) .. controls ([yshift=-.7cm] 4,-1) and ([yshift=1cm] 2,0) .. (2,0);
    \end{scope}

    % e2
    \begin{scope}[xshift=5cm]
      \filldraw[draw=black, fill=blue!20] (0,0) rectangle ++ (1,1);
      \filldraw[draw=black, fill=white] (1,0) rectangle ++ (1,1);
      \filldraw[draw=black, fill=white] (2,0) rectangle ++ (1,1);
      \node[at={(.5,.5)}, ptlabel] {$e_0$};
      \node[at={(1.5,.5)}, ptlabel] {$e_1$};
      \node[at={(2.5,.5)}, ptlabel] {$v_0$};
      % \draw[->] (2,-1) .. controls ([yshift=-.7cm] 2,-1) and ([yshift=1cm] .5,0) .. (.5,0);
    \end{scope}
  \end{scope}

  \draw ({1.5+1.5},1) -- ({0+.5},2);
  \draw ({2.5+1.5},1) -- ({7+.5},2);

  \draw ({2+1.5},3) -- ({0+0},4);
  \draw ({3+1.5},3) -- ({4+0},4);

  \draw ({5+1.5},3) -- ({0+5},4);
  \draw ({6+1.5},3) -- ({3+5},4);
\end{tikzpicture}

\vspace{2em}


\vspace{2em}

% alt refinement 3
\begin{tikzpicture}

  \tkzDefPoint(0,0){v0}
  \tkzDefShiftPoint[v0](60:4){v1}
  \tkzDefShiftPoint[v0](0:4){v2}
  \tkzDrawPolygon(v0,v1,v2)

  \tkzDefMidPoint(v0,v1) \tkzGetPoint{e0v0}
  \tkzDefMidPoint(v1,v2) \tkzGetPoint{e1v0}
  \tkzDefMidPoint(v2,v0) \tkzGetPoint{e2v0}

  \tkzDrawSegment(e0v0,e1v0)
  \tkzDrawSegment(e1v0,e2v0)
  \tkzDrawSegment(e2v0,e0v0)

  \tkzDefMidPoint(e1v0,v2) \tkzGetPoint{e1e1v0}
  \tkzDefMidPoint(v2,e2v0) \tkzGetPoint{e2e0v0}
  \tkzDefMidPoint(e1v0,e2v0) \tkzGetPoint{c0e2v0}

  \tkzDrawSegment(e1e1v0,e2e0v0)
  \tkzDrawSegment(e2e0v0,c0e2v0)
  \tkzDrawSegment(c0e2v0,e1e1v0)

  \tkzDrawSegment(e0v0,c0e2v0)
\end{tikzpicture}

\vspace{1em}
% vertex refinement
\begin{tikzpicture}

  \tkzDefPoint(0,0){cvert}
  \tkzDrawPoint(cvert) \tkzLabelPoint[above](cvert){$(v_i,)$}

  \draw[->] (1,0) -> (2,0);

  \tkzDefPoint(3,0){fvert}
  \tkzDrawPoint(fvert) \tkzLabelPoint[above](fvert){$(v_i,v_0)$}


\end{tikzpicture}

\vspace{2em}

% edge refinement
\begin{tikzpicture}

  \tkzDefPoint(0,0){cstart}
  \tkzDefPoint(3,0){cend}
  \tkzDrawSegment(cstart,cend)
  \tkzLabelSegment(cstart,cend){$(e_i,)$}

  \draw[->] (4,0) -> (5,0);

  \tkzDefPoint(6,0){fstart}
  \tkzDefPoint(11,0){fend}
  \tkzDefMidPoint(fstart,fend) \tkzGetPoint{fmid}

  \tkzDrawSegment(fstart,fmid) \tkzLabelSegment(fstart,fmid){$(e_i,e_0)$}
  \tkzDrawSegment(fmid,fend) \tkzLabelSegment(fmid,fend){$(e_i,e_1)$}
  \tkzDrawPoint(fmid) \tkzLabelPoint[below](fmid){$(e_i,v_0)$}

\end{tikzpicture}

\vspace{2em}

% cell refinement
\begin{tikzpicture}

  \tkzDefPoint(0,0){v0}
  \tkzDefShiftPoint[v0](60:5){v1}
  \tkzDefShiftPoint[v0](0:5){v2}
  \tkzDrawPolygon[style=dashed](v0,v1,v2)

  \tkzDefBarycentricPoint(v0=1,v1=1,v2=1) \tkzGetPoint{c0}
  \tkzLabelPoint[centered](c0){$c_i$}

  \draw[->] (5,{5/4*sqrt(3)}) -- (6,{5/4*sqrt(3)});

  \tkzDefPoint(7,0){fv0}
  \tkzDefShiftPoint[fv0](60:5){fv1}
  \tkzDefShiftPoint[fv0](0:5){fv2}
  \tkzDrawPolygon[style=dashed](fv0,fv1,fv2)

  \tkzDefMidPoint(fv0,fv1) \tkzGetPoint{fe0}
  \tkzDefMidPoint(fv1,fv2) \tkzGetPoint{fe1}
  \tkzDefMidPoint(fv2,fv0) \tkzGetPoint{fe2}

  \tkzDrawSegment(fe0,fe1)
  \tkzDrawSegment(fe1,fe2)
  \tkzDrawSegment(fe2,fe0)

  \tkzDefBarycentricPoint(fv0=1,fe0=1,fe2=1) \tkzGetPoint{fc0}
  \tkzLabelPoint[centered](fc0){$(c_i,c_0)$}

  \tkzDefBarycentricPoint(fe0=1,fe1=1,fe2=1) \tkzGetPoint{fc1}
  \tkzLabelPoint[centered](fc1){$(c_i,c_1)$}

  \tkzDefBarycentricPoint(fv1=1,fe0=1,fe1=1) \tkzGetPoint{fc2}
  \tkzLabelPoint[centered](fc2){$(c_i,c_2)$}

  \tkzDefBarycentricPoint(fv2=1,fe1=1,fe2=1) \tkzGetPoint{fc3}
  \tkzLabelPoint[centered](fc3){$(c_i,c_3)$}

  % edges (2/3 along)
  \tkzDefBarycentricPoint(fe2=1,fe0=2) \tkzGetPoint{e0labeldest}
  \tkzDefBarycentricPoint(fe0=1,fe1=2) \tkzGetPoint{e1labeldest}
  \tkzDefBarycentricPoint(fe1=2,fe2=1) \tkzGetPoint{e2labeldest}

  \node [xshift=-1.6cm,yshift=.2cm] (e0labelsrc) at (e0labeldest) {$(c_i,e_0)$};
  \draw (e0labelsrc) -- (e0labeldest);

  \node [xshift=1.2cm,yshift=.9cm] (e1labelsrc) at (e1labeldest) {$(c_i,e_1)$};
  \draw (e1labelsrc) -- (e1labeldest);

  \node [xshift=1.5cm,yshift=.4cm] (e2labelsrc) at (e2labeldest) {$(c_i,e_2)$};
  \draw (e2labelsrc) -- (e2labeldest);



\end{tikzpicture}

% one could introduce a proper tensor language to express things (like Simit) - currently only
% focussed on the packing part

\subsection{Patch preconditioners}
\label{sec:future_patch}


\begin{minted}{python}
loop(v := mesh.vertices, [
  loop(p := star(v), [
    kernel(dat1[closure(p)], dat2[closure(p)], "mat"),
    kernel(dat3[closure(p)], "vec"),
  ]),
  solve("mat", "vec", dat4[v]),
])
\end{minted}

\subsection{Constraints}
\label{sec:future_constraints}

  \section{Conclusions}
\label{sec:conclusions}

We have presented \pyop3, a new library for automating the application of stencil computations over a mesh.
Its interface takes inspiration from DMPlex to provide the user with a concise and composable way to express stencils and it has a novel data layout abstraction permitting a range of possible optimisations and mathematical methods.
Basing much of its design on PETSc, and through code generation, \pyop3 achieves significant flexibility in only a very small amount of code.



  \printbibliography
\end{document}
