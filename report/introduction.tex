\section{Introduction}

In scientific computing, the composition of appropriate (software) abstractions is essential for scientists to write portable and performant simulations in a productive way (the three P's).
Having suitable abstraction layers allows for a separation of concerns whereby numericists can reason about their problem from a purely mathematical point-of-view, and computer scientists can focus on low-level performance.
Each discipline is presented with a particular interface from which the problems of interest can be expressed in the clearest possible way.
This makes the application of particular optimisations straightforward as problem-specific information can be explicitly enumerated, rather than requiring inference to determine.

When it comes to writing software there are effectively three choices of approach.
For many problems, generic library interfaces introduce too much overhead to be viable options for writing programs.
Similarly, hand-written codes, though extremely fast, require a substantial effort to maintain and extend and the codebase can be very large.
Code generation is an appealing solution to these problems.
Given an appropriate abstraction, high-performance code can be automatically generated, compiled and run.
This offers an advantage over library interfaces because problem-specific information can be exploited to generate faster code (e.g. commonly used operations can be memoized for fast lookups), and the task of actually writing the code is offloaded to a compiler rather than being hand-written.
With a code generation framework, the key questions now become:
What is an appropriate abstraction for capturing all of the behaviour I wish to model?
What performance optimisations are nicely expressed at this layer of abstraction?

In this work, we present \projectname, a library for mesh computations.
In accordance with the principles described above, \projectname deals mainly in 3 abstractions:
Firstly, the user interface is motivated by the fact that many operations relating to the solution of \gls{pde}s can be expressed as the operation of some `local' kernel over a set of entities in the mesh where only functions with non-zero support on this entity are considered in the calculation.
A classic example of this sort of calculation occurs with finite element assembly where the cells of the mesh are the iteration set and the kernel uses \gls{dofs} from the cell and enclosing edges and vertices.
This first abstraction layer therefore presents an interface to the user where they may straightforwardly express the operation of local kernels within loops over mesh entities, specifying the requisite restrictions for the data.
The second abstraction, intended as an internal representation for the developer, describes the data layout and the third is a polyhedral loop model that is the target for the code generation.

Having motivated the need for a mesh traversal abstraction, we now move on to motivating the need for a \textit{high-performance} mesh traversal abstraction.
In other words, are there reasonable use-cases where this traversal constitutes a substantial fraction of the overall program runtime?

To begin, the wall-clock time of a program, assuming good algorithmic/parallel design, is usually limited by some combination of the maximum throughput of the processor and the cost of moving data to and from said processor.
Given that the principle role for \projectname is to marshall data for the kernel we focus on codes where the cost of moving data is the bottleneck.
Any effort on minimising the number of \gls{flops} would be wasted as such optimisations would only be useful inside the `hot' loops of the kernel.
Also, hardware developments have seen a general trend where computing power is increasing at a more rapid rate than memory access speed.
As such, the bottleneck in an increasing number of codes is going to be the memory accesses.

% FIXME: Merge with 'in this work' bit above
In this work, we present \projectname\dots

% applications:
% - multigrid
% - explicit DG
% - patch smoothers

% also supports:
% - p-adaptivity
% - mixed geometry meshes
% - hexahedral meshes (just Lagrange elements for now)

The rest of this paper is laid out as follows: \dots
