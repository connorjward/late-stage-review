\section{Future work}
\label{sec:future}

\subsection{Roadmap}
\label{sec:future_roadmap}

The approximate plan of action for the development and rollout of \pyop3 is as follows:

\begin{enumerate}
  \item
    Complete the code for the data layout abstraction, ensuring that the components are fully composable and tested.
  \item
    Add support for sparse matrices (via PETSc) and parallelism.
  \item
    Write the mesh `layer' of the code (e.g. automatically computing the right indirection maps from calls to \py{closure(...)} or \py{support(...)}).
  \item
    Ensure that the code is sufficiently optimised for its performance to be comparable to \pyop2.
    This will involve activities such as modifying algorithms for better complexity, or moving performance critical code to Cython.
  \item
    Produce a (mostly) working branch of Firedrake that uses \pyop3.
    Extruded meshes can natively use DMPlex, incurring a small performance penalty (see Section~\ref{sec:background_structure}).
  \item
    Incrementally add new features including: orientation support, inter-element vectorisation, data layout transformations and support for semi-structured meshes.
\end{enumerate}

\subsection{Direct addressing for partially-structured meshes}
\label{sec:future_partialstructure}

Depending upon the application, certain simulations use meshes that possess `partial structure'.
That is, meshes that possess both unstructured components and structured components.
In general, the structure found in these meshes can be classified as either \textit{refinement} or \textit{extrusion}.

A refined mesh is a mesh where some unstructured `coarse' mesh is refined by replacing cells of the mesh with multiple, smaller cells.
Edges and vertices are also inserted to keep appropriate connectivity, though \textit{hanging nodes} may occur if the refinement is non-conforming.
An example refined mesh is shown in Figure~\ref{fig:refined_patch}.

By contrast, an extruded mesh is created by taking an unstructured `base' mesh and extruding it into some number of layers.
This results in a mesh composed of columns (e.g. Figure~\ref{fig:extruded}).

For refined meshes the partial structure comes from having a finite number of possible refinement patterns.
Given a point in the refined mesh and a refinement pattern, it should be possible to address stencils without needing a lookup table for every single point as one can reason about the connectivity.
For extruded meshes the partial structure exists within the columns - each layer can be addressed directly using offsets given a starting point at the bottom cell.

The benefit to using partially structured meshes is that the memory volume of the simulation can be reduced which, in a memory-bound computation, will directly lead to speedup.
This is discussed in detail in Section~\ref{sec:background_structure} where we observed that the savings in memory volume are actually not that great, with a best case of 25\%.
Since the potential benefits are limited, we have not implemented meshes with partial structure in \pyop3 yet.
This section exists to demonstrate that our implementation does not prohibit making such optimisations in the future, and that in fact they would be relatively simple to implement as a consequence of our mesh-aware data layout.

As an aside, it is important to note that, in order to be able to extract any memory savings from this approach, both the stencils (a.k.a. maps) \textit{and} the layout functions must be expressible without the need for lookup tables.

\subsubsection{A unifying abstraction: mesh transformations}

To handle refinement and extrusion, DMPlex has a convenient way of unifying the two.
Termed \textit{mesh transformations}, the points in the input mesh are modified via some \textit{production rule}, resulting in a new, transformed, mesh.
Some example production rules are shown in Figures~\ref{fig:transform_refinement} (refinement) and~\ref{fig:transform_extrusion} (extrusion).
It is important to note that the production rule does not produce a `complete' cell - frequently only cell interiors without edges or edges without vertices are produced in the transformation.
This is necessary because it constrains each point in the transformed mesh to only have a single parent, making reasoning about structure much easier.

Note that there are a great many more production rules that are not shown here.
We have not included refinement rules for 3D polytopes (e.g. tetrahedra) and quadrilaterals are skipped.
Also, in some cases there are multiple ways to refine a point, for example a triangle can be `green' refined by connecting one vertex with the midpoint of the opposite edge~\cite{bankRefinementAlgorithmDynamic}.
Lastly, it should also be remarked that some transformations are naturally parametric.
For example extruding a mesh requires the number of layers to insert.
Likewise we could consider refining an edge, say, into 3 or more segments rather than just 2 (Figure~\ref{fig:transform_refinement_edge}).

% refinement transformation examples
\begin{figure}
  \centering

  % vertex refinement
  \begin{subfigure}{\textwidth}
    \centering
    \begin{tikzpicture}
      \tkzDefPoint(2,0){cvert}
      \tkzDrawPoint(cvert) 
      \draw[->] (5,0) -> (6,0);
      \tkzDefPoint(9,0){fvert}
      \tkzDrawPoint(fvert)

      \begin{scope}[overlay]
        \tkzLabelPoint[above](cvert){$(v_i,)$}
        \tkzLabelPoint[above](fvert){$(v_i,v_0)$}
      \end{scope}
    \end{tikzpicture}
    \caption{Vertex refinement - a no-op.}
  \end{subfigure}

  \vspace{1em}

  % edge refinement
  \begin{subfigure}{\textwidth}
    \centering
    \begin{tikzpicture}
      \tkzDefPoint(0,0){cstart}
      \tkzDefPoint(4,0){cend}
      \tkzDrawSegment(cstart,cend)
      \tkzLabelSegment(cstart,cend){$(e_i,)$}

      \draw[->] (5,0) -> (6,0);

      \tkzDefPoint(7,0){fstart}
      \tkzDefPoint(11,0){fend}
      \tkzDefMidPoint(fstart,fend) \tkzGetPoint{fmid}

      \tkzDrawSegment(fstart,fmid) \tkzLabelSegment(fstart,fmid){$(e_i,e_0)$}
      \tkzDrawSegment(fmid,fend) \tkzLabelSegment(fmid,fend){$(e_i,e_1)$}
      \tkzDrawPoint(fmid) \tkzLabelPoint[below](fmid){$(e_i,v_0)$}
    \end{tikzpicture}
    \caption{Edge refinement.}
    \label{fig:transform_refinement_edge}
  \end{subfigure}

  \vspace{1em}

  % cell refinement
  \begin{subfigure}{\textwidth}
    \centering
    \begin{tikzpicture}
      \tkzDefPoint(0,0){v0}
      \tkzDefShiftPoint[v0](60:4){v1}
      \tkzDefShiftPoint[v0](0:4){v2}
      \tkzDrawPolygon[style=dashed](v0,v1,v2)

      \tkzDefBarycentricPoint(v0=1,v1=1,v2=1) \tkzGetPoint{c0}
      \tkzLabelPoint[centered](c0){$c_i$}

      \draw[->] (5,{sqrt(3)}) -- (6,{sqrt(3)});

      \tkzDefPoint(7,0){fv0}
      \tkzDefShiftPoint[fv0](60:4){fv1}
      \tkzDefShiftPoint[fv0](0:4){fv2}
      \tkzDrawPolygon[style=dashed](fv0,fv1,fv2)

      \tkzDefMidPoint(fv0,fv1) \tkzGetPoint{fe0}
      \tkzDefMidPoint(fv1,fv2) \tkzGetPoint{fe1}
      \tkzDefMidPoint(fv2,fv0) \tkzGetPoint{fe2}

      \tkzDrawSegment(fe0,fe1)
      \tkzDrawSegment(fe1,fe2)
      \tkzDrawSegment(fe2,fe0)

      \tkzDefBarycentricPoint(fv0=1,fe0=1,fe2=1) \tkzGetPoint{fc0}
      \tkzLabelPoint[centered](fc0){$(c_i,c_0)$}

      \tkzDefBarycentricPoint(fe0=1,fe1=1,fe2=1) \tkzGetPoint{fc1}
      \tkzLabelPoint[centered](fc1){$(c_i,c_1)$}

      \tkzDefBarycentricPoint(fv1=1,fe0=1,fe1=1) \tkzGetPoint{fc2}
      \tkzLabelPoint[centered](fc2){$(c_i,c_2)$}

      \tkzDefBarycentricPoint(fv2=1,fe1=1,fe2=1) \tkzGetPoint{fc3}
      \tkzLabelPoint[centered](fc3){$(c_i,c_3)$}

      % edges (2/3 along)
      \begin{scope}[overlay]
        \tkzDefBarycentricPoint(fe2=1,fe0=2) \tkzGetPoint{e0labeldest}
        \tkzDefBarycentricPoint(fe0=1,fe1=2) \tkzGetPoint{e1labeldest}
        \tkzDefBarycentricPoint(fe1=2,fe2=1) \tkzGetPoint{e2labeldest}

        \node [xshift=-1.6cm,yshift=.2cm] (e0labelsrc) at (e0labeldest) {$(c_i,e_0)$};
        \draw (e0labelsrc) -- (e0labeldest);

        \node [xshift=1.2cm,yshift=.9cm] (e1labelsrc) at (e1labeldest) {$(c_i,e_1)$};
        \draw (e1labelsrc) -- (e1labeldest);

        \node [xshift=1.5cm,yshift=.4cm] (e2labelsrc) at (e2labeldest) {$(c_i,e_2)$};
        \draw (e2labelsrc) -- (e2labeldest);
      \end{scope}
    \end{tikzpicture}
    \caption{
      A possible refinement of a triangle.
      Note that no vertices are produced by the transformation and that the dashed lines indicate the production of a cell interior but not all its edges.
    }
  \end{subfigure}

  \caption{Example refinement transformations.}
  \label{fig:transform_refinement}
\end{figure}

% extrusion transformation examples
\begin{figure}
  \centering

  % vertex extrusion
  \begin{subfigure}{\textwidth}
    \centering
    \begin{tikzpicture}
      \tkzDefPoint(2,1){cvert}
      \tkzDrawPoint(cvert) 
      \draw[->] (5,1) -> (6,1);
      \tkzDefPoint(9,0){fvert0}
      \tkzDefPoint(9,2){fvert1}
      \tkzDrawPoints(fvert0,fvert1)
      \tkzDrawSegment(fvert0,fvert1)
      \tkzDefShiftPoint[fvert1](0,1){up}
      \tkzDrawSegment[dotted](fvert1,up)

      \begin{scope}[overlay]
        \tkzLabelPoint[above](cvert){$(v_i,)$}
        \tkzLabelPoint[right](fvert0){$(v_i,v_0)$}
        \tkzLabelPoint[right](fvert1){$(v_i,v_1)$}
        \tkzLabelSegment[left](fvert0,fvert1){$(v_i,e_0)$}
      \end{scope}
    \end{tikzpicture}
    \caption{Vertex extrusion.}
    \label{fig:transform_extrusion_vertex}
  \end{subfigure}

  \vspace{1em}

  % edge extrusion
  \begin{subfigure}{\textwidth}
    \centering
    \begin{tikzpicture}
      \tkzDefPoint(0,1){cstart}
      \tkzDefPoint(4,1){cend}
      \tkzDrawSegment(cstart,cend)
      \tkzLabelSegment(cstart,cend){$(e_i,)$}

      \draw[->] (5,1) -> (6,1);

      \tkzDefPoint(7,0){f0}
      \tkzDefPoint(11,0){f1}
      \tkzDefShiftPoint[f0](0,2){f2}
      \tkzDefShiftPoint[f1](0,2){f3}
      \tkzDrawSegment(f0,f1)
      \tkzDrawSegment(f2,f3)
      \tkzDrawSegment[densely dashed](f0,f2)
      \tkzDrawSegment[densely dashed](f1,f3)
      \tkzDefShiftPoint[f2](0,1){upl}
      \tkzDefShiftPoint[f3](0,1){upr}
      \tkzDrawSegments[dotted](f2,upl f3,upr)

      % labels
      \begin{scope}[overlay]
        \tkzLabelSegment[above](f0,f1){$(e_i,e_0)$}
        \tkzLabelSegment[above](f2,f3){$(e_i,e_1)$}
        \tkzDefBarycentricPoint(f0=1,f1=1,f2=1,f3=1) \tkzGetPoint{fmid}
        \tkzLabelPoint[centered](fmid){$(e_i,c_0)$}
      \end{scope}
    \end{tikzpicture}
    \caption{
      Edge extrusion.
      Note that no vertices are produced in the transformation as they would be produced by the vertices incident on the initial edge.
    }
    \label{fig:transform_extrusion_edge}
  \end{subfigure}
  \caption{
    Example extrusion transformations.
    The dotted lines indicate that the transformation may produce more than a single layer.
  }
  \label{fig:transform_extrusion}
\end{figure}

\subsubsection{Implementation: Overview}

% refined patch example
\begin{figure}
  \centering
  \begin{subfigure}{\textwidth}
    \centering
    \begin{tikzpicture}

      \tkzDefPoint(0,0){v0}
      \tkzDefShiftPoint[v0](60:4){v1}
      \tkzDefShiftPoint[v0](0:4){v2}
      \tkzDrawPolygon(v0,v1,v2)

      \tkzDefMidPoint(v0,v1) \tkzGetPoint{e0v0}
      \tkzDefMidPoint(v1,v2) \tkzGetPoint{e1v0}
      \tkzDefMidPoint(v2,v0) \tkzGetPoint{e2v0}

      \tkzDrawSegment(e0v0,e1v0)
      \tkzDrawSegment(e1v0,e2v0)
      \tkzDrawSegment(e2v0,e0v0)

      \tkzDefMidPoint(e1v0,v2) \tkzGetPoint{e1e1v0}
      \tkzDefMidPoint(v2,e2v0) \tkzGetPoint{e2e0v0}
      \tkzDefMidPoint(e1v0,e2v0) \tkzGetPoint{c0e2v0}

      \tkzDrawSegment(e1e1v0,e2e0v0)
      \tkzDrawSegment(e2e0v0,c0e2v0)
      \tkzDrawSegment(c0e2v0,e1e1v0)

      % patch
      \begin{pgfonlayer}{background}
        % find points by bisecting the angles
        \tkzDefShiftPoint[e0v0](-30:0.15){e0v0inner}
        \tkzDefShiftPoint[e1v0](-120:0.1){e1v0inner}
        \tkzDefShiftPoint[e1e1v0](150:0.15){e1e1v0inner}
        \tkzDefShiftPoint[c0e2v0](120:0.1){c0e2v0inner}
        \tkzDefShiftPoint[e2v0](90:0.15){e2v0inner}

        % source: https://tikz.dev/base-paths#sec-102.12
        \pgfsetcornersarced{\pgfpoint{1mm}{1mm}}
        \filldraw[color=blue!20] (e0v0inner) -- (e1v0inner) -- (e1e1v0inner) -- (c0e2v0inner) -- (e2v0inner) -- cycle;
        \pgfsetcornersarced{\pgfpointorigin}
      \end{pgfonlayer}

      % add labels
      \tkzDefBarycentricPoint(e0v0=1,e1v0=1,e2v0=1) \tkzGetPoint{c0c1}
      \node [xshift=-2cm,yshift=.8cm] (c0c1label) at (c0c1) {$(c_1,c_0)$};
      \draw (c0c1label) -- (c0c1);

      \tkzDefMidPoint(e1v0,c0e2v0) \tkzGetPoint{c0e2}
      \node [xshift=2cm,yshift=.8cm] (c0e2label) at (c0e2) {$(e_2,e_0)$};
      \draw (c0e2label) -- (c0e2);

      \tkzDefBarycentricPoint(e1v0=1,e1e1v0=1,c0e2v0=1) \tkzGetPoint{c0c3c2}
      \node [xshift=2cm,yshift=-.2cm] (c0c3c2label) at (c0c3c2) {$(c_3,c_2)$};
      \draw (c0c3c2label) -- (c0c3c2);
    \end{tikzpicture}
    \caption{
      An example of a stencil - $\plexstar((e_2,e_0))$ - over a refined mesh.
      Note that the unrefined cell $(c_1,c_0)$ is still indexed with two indices.
      We say that it has been refined using the identity transformation.
    }
    \label{fig:refined_patch}
  \end{subfigure}

  \vspace{1em}

  % data layout for patch
  \begin{subfigure}{\textwidth}
    \centering
    \begin{tikzpicture}[y=-1cm]
      \begin{scope}[xshift=.5cm, yshift=0cm]
        \filldraw[draw=black, fill=white] (0,0) rectangle ++ (1,1);
        \filldraw[draw=black, fill=blue!20] (1,0) rectangle ++ (1,1);
        \filldraw[draw=black, fill=white] (2,0) rectangle ++ (1,1);
        \filldraw[draw=black, fill=blue!20] (3,0) rectangle ++ (1,1);
        \filldraw[draw=black, fill=white] (4,0) rectangle ++ (1,1);
        \filldraw[draw=black, fill=blue!20] (5,0) rectangle ++ (1,1);
        \filldraw[draw=black, fill=white] (6,0) rectangle ++ (1,1);
        \node[at={(.5,.5)}, ptlabel] {$c_0$};
        \node[at={(1.5,.5)}, ptlabel] {$c_3$};
        \node[at={(2.5,.5)}, ptlabel] {$e_0$};
        \node[at={(3.5,.5)}, ptlabel] {$c_1$};
        \node[at={(4.5,.5)}, ptlabel] {$c_2$};
        \node[at={(5.5,.5)}, ptlabel] {$e_2$};
        \node[at={(6.5,.5)}, ptlabel] {$e_1$};

        % \draw[->] (2.8,-1) .. controls ([yshift=-.4cm] 2.6,-1) and ([yshift=.6cm] 1,0) .. (.8,0);
        % \draw[->] (3,-1) .. controls ([yshift=-.6cm] 3,-1) and ([yshift=1cm] 3.5,0) .. (3.5,0);
        % \draw[->] (3.2,-1) .. controls ([yshift=-.4cm] 3,-1) and ([yshift=.6cm] 6,0) .. (6.2,0);
      \end{scope}

      \begin{scope}[xshift=0cm, yshift=-2cm]
        % c3
        \begin{scope}[xshift=-1cm]
          \fill[lightgray] (0,0) rectangle (4,1);
          \filldraw[draw=black, fill=white] (.5,0) rectangle ++ (1,1);
          \filldraw[draw=black, fill=blue!20] (1.5,0) rectangle ++ (1,1);
          \filldraw[draw=black, fill=white] (2.5,0) rectangle ++ (1,1);
          \node[at={(1,.5)}, ptlabel] {$c_1$};
          \node[at={(2,.5)}, ptlabel] {$c_2$};
          \node[at={(3,.5)}, ptlabel] {$c_3$};
          \draw (0,0) -- (4,0);
          \draw (0,1) -- (4,1);
          % \draw[->] (4,-1) .. controls ([yshift=-.7cm] 4,-1) and ([yshift=1cm] 2,0) .. (2,0);
        \end{scope}

        % c1c0
        \begin{scope}[xshift=3.5cm]
          \filldraw[draw=black, fill=blue!20] (0,0) rectangle ++ (1,1);
          \node[at={(.5,.5)}, ptlabel] {$c_0$};
        \end{scope}

        % e2
        \begin{scope}[xshift=5cm]
          \filldraw[draw=black, fill=blue!20] (0,0) rectangle ++ (1,1);
          \filldraw[draw=black, fill=white] (1,0) rectangle ++ (1,1);
          \filldraw[draw=black, fill=white] (2,0) rectangle ++ (1,1);
          \node[at={(.5,.5)}, ptlabel] {$e_0$};
          \node[at={(1.5,.5)}, ptlabel] {$e_1$};
          \node[at={(2.5,.5)}, ptlabel] {$v_0$};
          % \draw[->] (2,-1) .. controls ([yshift=-.7cm] 2,-1) and ([yshift=1cm] .5,0) .. (.5,0);
        \end{scope}
      \end{scope}

      \draw [densely dashed] ({1+.5},1) -- ({0-1},2);
      \draw [densely dashed] ({2+.5},1) -- ({4-1},2);

      \draw ({3+.5},1) -- ({0+3.5},2);
      \draw ({4+.5},1) -- ({1+3.5},2);

      \draw ({5+.5},1) -- ({0+5},2);
      \draw ({6+.5},1) -- ({3+5},2);
    \end{tikzpicture}
    \caption{
      Example data layout for the refined mesh shown above.
      Note that the base mesh in unstructured which is why the top axis is unordered.
    }
    \label{fig:refined_data}
  \end{subfigure}
  \caption{Example data layout and stencil for a refined mesh.}
  \label{fig:refined_patch_and_data}
\end{figure}

% extruded patch example
\begin{figure}
  \centering
  \begin{subfigure}{\textwidth}
    \centering
    \begin{tikzpicture}
      \tkzDefPoint(0,0){v0v0}
      \tkzDefShiftPoint[v0v0](2,0){v1v0}
      \tkzDefShiftPoint[v1v0](2,0){v2v0}
      \tkzDefShiftPoint[v0v0](0,2){v0v1}
      \tkzDefShiftPoint[v1v0](0,2){v1v1}
      \tkzDefShiftPoint[v2v0](0,2){v2v1}
      \tkzDefShiftPoint[v0v1](0,2){v0v2}
      \tkzDefShiftPoint[v1v1](0,2){v1v2}
      \tkzDefShiftPoint[v2v1](0,2){v2v2}
      \tkzDefShiftPoint[v0v2](0,2){v0v3}
      \tkzDefShiftPoint[v1v2](0,2){v1v3}

      % horiz edges
      \tkzDrawSegments(v0v0,v1v0 v1v0,v2v0)
      \tkzDrawSegments(v0v1,v1v1 v1v1,v2v1)
      \tkzDrawSegments(v0v2,v1v2 v1v2,v2v2)
      \tkzDrawSegments(v0v3,v1v3)

      % vert edges
      \tkzDrawSegments(v0v0,v0v1 v0v1,v0v2 v0v2,v0v3)
      \tkzDrawSegments(v1v0,v1v1 v1v1,v1v2 v1v2,v1v3)
      \tkzDrawSegments(v2v0,v2v1 v2v1,v2v2)

      % verts
      \tkzDrawPoints(v0v0,v0v1,v0v2,v0v3)
      \tkzDrawPoints(v1v0,v1v1,v1v2,v1v3)
      \tkzDrawPoints(v2v0,v2v1,v2v2)

      % patch
      \begin{pgfonlayer}{background}
        % find points by bisecting the angles
        \tkzDefShiftPoint[v0v0](45:0.15){v0v0inner}
        \tkzDefShiftPoint[v0v1](-45:0.15){v0v1inner}
        \tkzDefShiftPoint[v2v1](-135:0.15){v2v1inner}
        \tkzDefShiftPoint[v2v0](135:0.15){v2v0inner}

        % source: https://tikz.dev/base-paths#sec-102.12
        \pgfsetcornersarced{\pgfpoint{1mm}{1mm}}
        \filldraw[color=blue!20] (v0v0inner) -- (v0v1inner) -- (v2v1inner) -- (v2v0inner) -- cycle;
        \pgfsetcornersarced{\pgfpointorigin}
      \end{pgfonlayer}

      % add labels
      \begin{scope}[overlay]
        \tkzDefBarycentricPoint(v0v0=1,v0v1=1,v1v0=1,v1v1=1) \tkzGetPoint{c0c0}
        \node [xshift=-2cm,yshift=.8cm] (c0c0label) at (c0c0) {$(e_0,c_0)$};
        \draw (c0c0label) -- (c0c0);

        \tkzDefBarycentricPoint(v1v0=1,v1v1=1,v2v0=1,v2v1=1) \tkzGetPoint{c1c0}
        \node [xshift=2cm,yshift=.4cm] (c1c0label) at (c1c0) {$(e_1,c_0)$};
        \draw (c1c0label) -- (c1c0);

        \tkzDefBarycentricPoint(v1v0=1,v1v1=1) \tkzGetPoint{v1e0}
        \node [xshift=-.9cm,yshift=1.7cm] (v1e0label) at (v1e0) {$(v_1,e_0)$};
        \draw (v1e0label) -- (v1e0);
      \end{scope}

      % bottom labels need to be included in bounding box
      \tkzDefBarycentricPoint(v0v0=1,v1v0=1) \tkzGetPoint{e0}
      \tkzLabelPoint[below](e0){$e_0$}
      \tkzDefBarycentricPoint(v1v0=1,v2v0=1) \tkzGetPoint{e1}
      \tkzLabelPoint[below](e1){$e_1$}

      \tkzLabelPoint[below](v0v0){$v_0$}
      \tkzLabelPoint[below](v1v0){$v_1$}
      \tkzLabelPoint[below](v2v0){$v_2$}
    \end{tikzpicture}
    \caption{
      An example of a stencil - $\plexstar((v_1, e_0))$ - applied to an extruded mesh formed by extruding a `base' mesh consisting of 2 edges ($e_0$ and $e_1$) and 3 vertices ($v_0$, $v_1$ and $v_2$).
      Note that the mesh shown here has `variable layers' to emphasise that such a mesh would be supported by our abstraction.
    }
    \label{fig:extruded_patch}
  \end{subfigure}

  \vspace{1em}

  % data layout for patch
  \begin{subfigure}{\textwidth}
    \centering
    \begin{tikzpicture}[y=-1cm]
      \begin{scope}[xshift=3.25cm, yshift=0cm]
        \filldraw[draw=black, fill=blue!20] (0,0) rectangle ++ (1,1);
        \filldraw[draw=black, fill=white] (1,0) rectangle ++ (1,1);
        \filldraw[draw=black, fill=blue!20] (2,0) rectangle ++ (1,1);
        \filldraw[draw=black, fill=white] (3,0) rectangle ++ (1,1);
        \filldraw[draw=black, fill=blue!20] (4,0) rectangle ++ (1,1);

        \node[at={(.5,.5)}, ptlabel] {$v_1$};
        \node[at={(1.5,.5)}, ptlabel] {$v_0$};
        \node[at={(2.5,.5)}, ptlabel] {$e_0$};
        \node[at={(3.5,.5)}, ptlabel] {$v_2$};
        \node[at={(4.5,.5)}, ptlabel] {$e_1$};
      \end{scope}

      \begin{scope}[yshift=-2cm]
        \begin{scope}[xshift=0cm]
          \fill[lightgray] (0,0) rectangle (3.5,1);
          \filldraw[draw=black, fill=white] (0,0) rectangle ++ (1,1);
          \filldraw[draw=black, fill=blue!20] (1,0) rectangle ++ (1,1);
          \filldraw[draw=black, fill=white] (2,0) rectangle ++ (1,1);
          \node[at={(.5,.5)}, ptlabel] {$v_0$};
          \node[at={(1.5,.5)}, ptlabel] {$e_0$};
          \node[at={(2.5,.5)}, ptlabel] {$v_1$};
          \draw (0,0) -- (3.5,0);
          \draw (0,1) -- (3.5,1);
        \end{scope}

        \begin{scope}[xshift=4cm]
          \fill[lightgray] (0,0) rectangle (3.5,1);
          \filldraw[draw=black, fill=white] (0,0) rectangle ++ (1,1);
          \filldraw[draw=black, fill=blue!20] (1,0) rectangle ++ (1,1);
          \filldraw[draw=black, fill=white] (2,0) rectangle ++ (1,1);
          \node[at={(.5,.5)}, ptlabel] {$e_0$};
          \node[at={(1.5,.5)}, ptlabel] {$c_0$};
          \node[at={(2.5,.5)}, ptlabel] {$e_1$};
          \draw (0,0) -- (3.5,0);
          \draw (0,1) -- (3.5,1);
        \end{scope}

        \begin{scope}[xshift=8cm]
          \fill[lightgray] (0,0) rectangle (3.5,1);
          \filldraw[draw=black, fill=white] (0,0) rectangle ++ (1,1);
          \filldraw[draw=black, fill=blue!20] (1,0) rectangle ++ (1,1);
          \filldraw[draw=black, fill=white] (2,0) rectangle ++ (1,1);
          \node[at={(.5,.5)}, ptlabel] {$e_0$};
          \node[at={(1.5,.5)}, ptlabel] {$c_0$};
          \node[at={(2.5,.5)}, ptlabel] {$e_1$};
          \draw (0,0) -- (3.5,0);
          \draw (0,1) -- (3.5,1);
        \end{scope}
      \end{scope}

      \draw ({3.25+0},1) -- ({0+0},2);
      \draw [densely dashed] ({3.25+1},1) -- ({0+3.5},2);
      \draw ({3.25+2},1) -- ({4+0},2);
      \draw [densely dashed] ({3.25+3},1) -- ({4+3.5},2);
      \draw ({3.25+4},1) -- ({8+0},2);
      \draw [densely dashed] ({3.25+5},1) -- ({8+3.5},2);
    \end{tikzpicture}
    \caption{
      Example data layout for the extruded mesh shown above.
      Points in the stencil are highlighted in blue.
      Note that the points in the `base' mesh are not ordered since it represents an unstructured mesh.
    }
    \label{fig:extruded_data}
  \end{subfigure}
  \caption{Example data layout and stencil for an extruded mesh.}
  \label{fig:extruded_patch_and_data}
\end{figure}

To implement mesh transformations in a structure preserving way, we simply require that the mesh points produced from the transformation produce a new subaxis in the data layout.
This is most simply demonstrated for extruded meshes.
If we consider the extruded mesh and data layout shown in Figure~\ref{fig:extruded_patch_and_data}, the `base' mesh is formed of 2 edges ($e_0$ and $e_1$) and 3 vertices ($v_0$, $v_1$ and $v_2$).
From Figure~\ref{fig:transform_extrusion} we see that, under extrusion, vertices produce points like $(v_0, e_0, v_1, \dots)$ and that edges produce points like $(e_0, c_0, e_1, \dots)$.
These production rules exactly match the subaxes shown in Figure~\ref{fig:extruded_data}.

The principle benefit of codifying the distinction between the base points and those up each of the columns in separate axes is that we can now use separate layout functions (see Section~\ref{sec:impl_datalayout}) to handle the addressing for each.
The base mesh is unstructured - and so an indirection map is required to address its axis - but the points up each of the columns are structured and can be addressed using some affine indexing function (i.e. \clang{offset = start + i*step}).

\subsubsection{Implementation: Rethinking maps}
\label{sec:future_partialstructure_maps}

As described in Section~\ref{sec:impl_datalayout_maps}, a map is a function that accepts a multi-index and returns a collection of multi-indices.
For meshes without partial structure it is sufficient to limit the length of these multi-indices to 1 - any `parent' point, often non-existing, will always be the same for both the input point and all of the outputs.
Inconveniently, for partially structured meshes, this assumption no longer holds and parent points may differ.
This is illustrated in Figure~\ref{fig:extruded_patch_and_data}: $\plexstar((v_1,e_0))$ is $\big[ (v_1,e_0), (e_0,c_0), (e_1,c_0) \big]$ - the parent points, here corresponding to the `base' mesh points, are different.

As discussed above, in order to achieve any performance gains via memory volume reduction both the maps and the layout functions must be expressible without resorting to a global tabulation.
In the extruded case just described, this really means that one cannot store the full multi-indices in a lookup table.
Instead, the information available to \pyop3 is as follows:
1) the $\plexstar$ of a vertical edge contains cells `belonging' to adjacent base edges, and
2) the adjacency relationships between base entities (i.e. we know that $v_1$ is incident on $e_0$ and $e_1$).
Using these pieces of information it is possible to reconstruct the full set of complete multi-indices required to address the data correctly.

\subsubsection{Implementation: Interfacing with DMPlex}

One major shortcoming of the existing extruded mesh implementation in \pyop2 is that an extruded mesh is not in fact a DMPlex instance.
Instead it uses a DMPlex to represent the unstructured base mesh and then uses custom code to handle the extrusion up the columns.
With such an implementation, as far as DMPlex is concerned, an extruded mesh is merely a mesh with `very big' cells (i.e. storing all the \glspl{dof} for each column).

This approach means that extruded meshes are special-cased throughout \pyop2 and Firedrake, requiring specialised implementations for all manner of operations including: preconditioner application, multigrid and I/O.
Indeed there are places in Firedrake where, due to the extra burden of developing a custom implementation, there is no support for extruded meshes at all.
One example of such a case is the fact that one cannot have an extruded \py{VertexOnlyMesh}.

\pyop3 takes a different approach to describing partial structure in meshes that we believe should avoid the need for a proliferation of custom implementations.
Our approach revolves around the idea that a mesh should always be represented by a single DMPlex instance, from which any structure can be inferred.
By choosing to sit directly on top of DMPlex, all code for unstructured meshes should now work for both cases.

To obtain a hierarchical data layout similar to that shown in Figure~\ref{fig:extruded_data}, \pyop3 would inspect the \textit{labels} of the DMPlex.
These labels are integers associated with each mesh point.

The critical point here is that the \textit{labels of the input mesh points are automatically passed to the transformed points}.
This means that one can, for example, uniquely label each point in the input mesh and then extrude it and this will result in a mesh where all of the transformed points up the column `know' the base point to which it belongs.
The same approach naturally works for refined meshes, uniquely labelling the coarse points prior to refinement allows \pyop3 to reconstruct the right data layout by inspecting the labels.

Since labels persist when writing to disk, we believe that, with minimal code, the hierarchical data structures could be reconstructed via analysis of these labels.

\subsection{Patch-based multigrid smoothers}
\label{sec:future_patch}

It has been demonstrated that geometric multigrid with a smoother stage involving the direct solution of a `local' finite element problem is effective for many problem~\cite{vankaBlockimplicitMultigridSolution1986,arnoldPRECONDITIONINGDivAPPLICATIONS1997,farrellAugmentedLagrangianPreconditioner2019}.
These `local' problems, called \textit{patches}, are in fact subdomains of the entire mesh taken via some composition of DMPlex restrictions.
Examples include \textit{vertex-star} patches, the \glspl{dof} defined on a vertex and entities in its $\plexstar$, and \textit{Vanka} patches, the same but taking the $\closure$ of the vertex-star to capture a larger patch.
The idea behind these patches is that a local finite element problem is solved using them and this contributes an update, via either the additive or multiplicative Schwartz methods, to the current guess.

This abstraction has been implemented via contributions to Firedrake, PETSc and \pyop2 and is called PCPATCH~\cite{farrellPCPATCHSoftwareTopological2021}.
To run, the `outer loop' over patches and the updates (either additive or multiplicative) are performed by PETSc.
Callbacks registered in Firedrake are used to construct the local problem.
Since the problem is defined entirely using PETSc types, one can utilise any of the possible solver strategies provided by it.
In particular, matrix-free solver implementations are natively supported.

Firedrake also supports an alternative backend for applying patch preconditioners called \textit{TinyASM}~\cite{wechsungTinyASMBlockJacobiImplementation}.
TinyASM, at setup time, precomputes the matrix inverses for each patch so the local solve can be done very efficiently without needing to use PETSc objects, which are specialised towards much larger linear systems.

Both of these existing approaches have a number of drawbacks.
As just mentioned, solving linear systems in PETSc can be inefficient for patches as one needs to solve lots of small problems, rather than a single large one.
This is solved by TinyASM, but their approach is unsuitable for high order methods because it requires computing a large number of dense inverses which can cause a machine to run out of memory.
Also, both systems require a significant amount of hand-coding for specific patches and reasoning about numberings etc.

% Pablo's stuff
We also run into problems when dealing with sparsity-preserving discretisations at high-order.
Matrix-explicit implementations are unsuitable because the per-patch matrices, though sparse, are very large and can fill up a machine's memory.
Also, matrix-free implementations won't work because each cell returns a dense block and the sparsity is lost.
To resolve, we would like to be able to construct sparse matrices `on-the-fly' for each patch.
To make this efficient we would need to memoize the different potential sparsity patterns - you get different patterns depending on the number of incident edges on a vertex for example.

In \pyop3, we would like to simplify these implementation considerations by raising the level of abstraction.
The \pyop3 interface (Section~\ref{sec:impl}) is already flexible enough to permit the sorts of loops that patch smoothing requires.
For example, a Vanka patch (closure of a vertex-star) could be expressed as follows:

\begin{minted}{python}
loop(v := mesh.vertices.index, [
  loop(p := star(v).index, [
    assemble_jacobian(dat1[closure(p)], dat2[closure(p)], "mat"),
    assemble_residual(dat3[closure(p)], "vec"),
  ]),
  solve_and_update("mat", "vec", dat4[v]),
])
\end{minted}

Note that here we use the strings \py{"mat"} and \py{"vec"} to identify the loop temporaries.
This is syntactic sugar and if we were to want to specify non-default behaviour for these objects, for instance memoizing the sparsity patterns or using a pre-computed inverse, then we could instead instantiate specific \py{LoopTemporary} objects.
