\section{Implementation}

% pyop3 is inspired by Simit in that we have a mesh representation that lowers to an indexed tensor 
% representation for computing


% data layout example
\begin{tikzpicture}[y=-1cm]
  \fill[lightgray] (0,0) rectangle(7,1);
  \filldraw[draw=black, fill=white] (0.5,0) rectangle (1.5,1);
  \filldraw[draw=black, fill=white] (1.5,0) rectangle (2.5,1);
  \filldraw[draw=black, fill=white] (2.5,0) rectangle (3.5,1);
  \filldraw[draw=black, fill=white] (3.5,0) rectangle (4.5,1);
  \node[at={(1,.5)}, ptlabel] {$c_0$};
  \node[at={(2,.5)}, ptlabel] {$v_1$};
  \node[at={(3,.5)}, ptlabel] {$c_4$};
  \node[at={(4,.5)}, ptlabel] {$v_8$};
  \draw (0,0) -- (7,0);
  \draw (0,1) -- (7,1);

  \begin{scope}[yshift=-2cm]
    \filldraw[draw=black, fill=white] (2,0) rectangle (3,1);
    \filldraw[draw=black, fill=white] (3,0) rectangle (4,1);
    \node[at={(2.5,.5)}, ptlabel] {$d_0$};
    \node[at={(3.5,.5)}, ptlabel] {$d_1$};
    \draw (2,0) rectangle (4,1);

    \draw (2.5,-1) -- (2,0);
    \draw (3.5,-1) -- (4,0);
  \end{scope}

\end{tikzpicture}

\vspace{1em}
