\section{Introduction}

In scientific computing, the composition of appropriate software abstractions is essential for scientists to write portable and performant simulations in a productive way (the three P's).
Having suitable abstraction layers allows for a separation of concerns whereby numericists can reason about their problem from a purely mathematical point of view and computer scientists can focus on optimising performance.

In this work we present \pyop3, a library for the fast execution of stencil computations over a mesh.
The chief contributions of \pyop3 are two new abstraction layers:
1) a simple yet powerful interface for mathematicians to express a wide range of stencil computations,
and 2) a novel system for describing mesh-based data layouts, facilitating both performance optimisations and typically hard-to-implement mathematical methods.
Code generation and just-in-time compilation are leveraged to achieve these with only a very small codebase.

The rest of this report is laid out as follows:
In Section~\ref{sec:background} we review existing stencil libraries and mesh abstractions as well as strategies for performance optimisation.
Then, Section~\ref{sec:impl} discusses the design of \pyop3 and Section~\ref{sec:future} reviews some possible extensions that might be pursued in future.
Some concluding remarks are made in Section~\ref{sec:conclusions}.
